\ifx\wholebook\relax \else
% ------------------------

\documentclass{article}
%------------------- Other types of document example ------------------------
%
%\documentclass[twocolumn]{IEEEtran-new}
%\documentclass[12pt,twoside,draft]{IEEEtran}
%\documentstyle[9pt,twocolumn,technote,twoside]{IEEEtran}
%
%-----------------------------------------------------------------------------
%\input{../../../common.tex}
\input{../../../common-en.tex}

\setcounter{page}{1}

\begin{document}

\fi
%--------------------------

% ================================================================
%                 COVER PAGE
% ================================================================

\title{Trie and Patricia}

\author{Larry~LIU~Xinyu
\thanks{{\bfseries Larry LIU Xinyu } \newline
  Email: liuxinyu95@gmail.com \newline}
  }

\markboth{Trie and Patricia}{Elementary algorithms}

\maketitle

\ifx\wholebook\relax
\chapter{Trie and Patricia}
\numberwithin{Exercise}{chapter}
\fi

%{\bfseries Corresponding Author:} Larry LIU Xinyu


% ================================================================
%                 Introduction
% ================================================================
\section{Introduction}
\label{introduction}

The binary trees introduced so far store information in nodes. It's
possible to store the information in edges.
Trie and Patricia are important data structures in
information retrieving and manipulating.
They were invented in 1960s. And are widely used in
compiler design\cite{okasaki-int-map}, and bio-information area, such as
DNA pattern matching \cite{wiki-suffix-tree}.

\begin{figure}[htbp]
  \centering
  \includegraphics[scale=0.5]{img/radix-tree.ps}
  \caption{Radix tree.} \label{fig:radix-tree}
\end{figure}

Figure \ref{fig:radix-tree} shows a radix tree\cite{CLRS}. It contains the
strings of bit 1011, 10, 011, 100 and 0. When searching a key $k=b_0b_1...b_n$, we
take the first bit $b_0$ (MSB from left), check if it is 0 or 1, if it
is 0, we turn left; and turn right for 1. Then we take the second bit and
repeat this search till either meet a leaf or finish all n bits.

The radix tree needn't store keys in node at all. The
information is represented by edges. The nodes marked with keys
in the above figure are only for illustration purpose.

It is very natural to come to the idea `is it possible to represent
key in integer instead of string?' Because integer can be written in
binary format, such approach can save spaces. Another advantage is
that the speed is fast because we can use bit-wise manipulation in
most programming environment.

% ================================================================
%                 Int Trie
% ================================================================
\section{Integer Trie}
\label{int-trie}

The data structure shown in figure \ref{fig:radix-tree} is
called as \emph{binary trie}.
Trie is invented by Edward Fredkin. It comes from ``retrieval'', pronounce
as /'tri:/ by the inventor, while it is pronounced /'trai/ ``try''
by other authors \cite{wiki-trie}. Trie is also called prefix tree or
radix tree.
A binary
trie is a special binary tree in which the placement of each key is controlled by
its bits, each 0 means ``go left'' and each 1 means ``go
right''\cite{okasaki-int-map}.

Because integers can be represented in binary format, it is
possible to store integer keys rather than 0, 1 strings. When insert an
integer as the new key to the trie, we change it to binary form, then examine
the first bit, if it is 0, we recursively
insert the rest bits to the left sub tree; otherwise if it is 1, we insert
into the right sub tree.

There is a problem when treat the key as integer. Consider a binary
trie shown in figure \ref{fig:big-endian-trie}. If represented in
0, 1 strings, all the three keys are different. But they are identical when
turn into integers. Where should we insert decimal 3, for example, to the trie?

\begin{figure}[htbp]
  \centering
  \includegraphics[scale=0.5]{img/big-endian-trie.ps}
  \caption{a big-endian trie} \label{fig:big-endian-trie}
\end{figure}

One approach is to treat all the prefix zero as effective bits.
Suppose the integer is represented with 32-bits, If we want to insert key 1,
it ends up with a tree of 32 levels.
There are 31 nodes, each only has the left sub tree. the last node only has
the right sub tree. It is very inefficient in terms of space.

Okasaki shows a method to solve this problem in \cite{okasaki-int-map}. Instead of
using big-endian integer, we can use the little-endian integer to represent key.
Thus decimal integer 1 is represented as binary 1. Insert it to the empty binary
trie, the result is a trie with a root and a right leaf.
There is only 1 level. decimal 2 is represented as 01, and decimal 3 is 11
in little-eidan binary format. There is no need to add
any prefix 0, the position in the trie is uniquely determined.

%=========================================================================
%       Definition of integer trie
%=========================================================================
\subsection{Definition of integer Trie}
In order to define the little-endian binary trie, we can reuse the structure
of binary tree.
A binary trie node is either empty, or a branch node. The branch
node contains a left child, a right node, and optional value as
satellite data.
The left sub tree is encoded as 0 and the right sub tree
is encoded as 1.

The following example Haskell code defines the trie algebraic data type.

\lstset{language=Haskell}
\begin{lstlisting}
data IntTrie a = Empty
               | Branch (IntTrie a) (Maybe a) (IntTrie a)
\end{lstlisting}

Below is another example definition in Python.

\lstset{language=Python}
\begin{lstlisting}
class IntTrie:
    def __init__(self):
        self.left = self.right = None
        self.value = None
\end{lstlisting}


% ================================================================
%               Insertion of integer trie
% ================================================================
\subsection{Insertion}

Since the key is little-endian integer, when insert a key, we take the
bit one by one from the right most (LSB).
If it is 0, we go to the left, otherwise go to the right
for 1. If the child is empty, we need create a new node, and repeat this to
the last bit (MSB) of the key.

%\begin{algorithm}
\begin{algorithmic}[1]
\Function{Insert}{$T, k, v$}
  \If{$T =$ NIL}
    \State $T \gets$ \Call{EmptyNode}{}
  \EndIf
  \State $p \gets T$
  \While{$k \neq 0$}
    \If{\Call{Even?}{$k$}}
      \If{\Call{Left}{$p$} = NIL}
        \State \Call{Left}{$p$} $\gets$ \Call{EmptyNode}{}
      \EndIf
      \State $p \gets$ \Call{Left}{$p$}
    \Else
      \If{\Call{Right}{$p$} = NIL}
        \State \Call{Right}{$p$} $\gets$ \Call{EmptyNode}{}
      \EndIf
      \State $p \gets$ \Call{Right}{$p$}
    \EndIf
    \State $k \gets \lfloor k/2 \rfloor$
  \EndWhile
  \State \Call{Data}{$p$} $\gets v$
  \State \Return $T$
\EndFunction
\end{algorithmic}
%\end{algorithm}

This algorithm takes 3 arguments, a Trie $T$, a key $k$, and the satellite
date $v$. The following Python example code implements the insertion algorithm.
The satellite data is optional, it is empty by default.

\lstset{language=Python}
\begin{lstlisting}
def trie_insert(t, key, value = None):
    if t is None:
        t = IntTrie()
    p = t
    while key != 0:
        if key & 1 == 0:
            if p.left is None:
                p.left = IntTrie()
            p = p.left
        else:
            if p.right is None:
                p.right = IntTrie()
            p = p.right
        key = key>>1
    p.value = value
    return t
\end{lstlisting}

Figure \ref{int-trie} shows a trie which is created by inserting
pairs of key and value
\{$ 1 \rightarrow a, 4 \rightarrow b, 5 \rightarrow c, 9 \rightarrow d$\}
to the empty trie.

\begin{figure}[htbp]
  \centering
  \includegraphics[scale=0.5]{img/int-trie.ps}
  \caption{An little-endian integer binary trie for the map
          \{$ 1 \rightarrow a, 4 \rightarrow b, 5 \rightarrow c, 9 \rightarrow d$\}.}
  \label{fig:int-trie}
\end{figure}

Because the definition of the integer trie is recursive, it's nature to define
the insertion algorithm recursively. If the LSB is 0, it
means that the key to be inserted is even, we recursively insert
to the left child, we can divide the key by 2 to get
rid of the LSB. If the LSB is 1, the key is odd number, the recursive
insertion is applied to the right child. For trie $T$,
denote the left and right children as $T_l$ and $T_r$ respectively.
Thus $T = (T_l, d, T_r)$, where $d$ is the optional satellite data.
if $T$ is empty, $T_l$, $T_r$ and $d$ are defined as empty as well.

\be
insert(T, k, v) = \left \{
  \begin{array}
  {r@{\quad:\quad}l}
  (T_l, v, T_r) & k = 0 \\
  (insert(T_l, k / 2, v), d, T_r) & even(k) \\
  (T_l, d, insert(T_r, \lfloor k / 2 \rfloor, v)) & otherwise
  \end{array}
\right.
\ee

If the key to be inserted already exists, this algorithm just
overwrites the previous stored data. It can be replaced with
other alternatives, such as storing data as with linked-list etc.

The following Haskell example program implements the insertion
algorithm.

\lstset{language=Haskell}
\begin{lstlisting}
insert t 0 x = Branch (left t) (Just x) (right t)
insert t k x | even k = Branch (insert (left t) (k `div` 2) x) (value t) (right t)
             | otherwise = Branch (left t) (value t) (insert (right t) (k `div` 2) x)

left (Branch l _ _) = l
left Empty = Empty

right (Branch _ _ r) = r
right Empty = Empty

value (Branch _ v _) = v
value Empty = Nothing
\end{lstlisting}

For a given integer $k$ with $m$ bits in binary, the insertion algorithm
resurses $m$ levels. The performance is bound to $O(m)$ time.

% ================================================================
%               Look up in integer binary trie
% ================================================================
\subsection{Look up}

To look up key $k$ in the little-endian integer binary trie. We take each
bit of $k$ from the left (LSB), then go left if this bit is 0, otherwise,
we go right. The looking up completes when all bits are consumed.

\begin{algorithmic}[1]
\Function{Lookup}{$T, k$}
  \While{$x \neq 0 \land T \neq $NIL}
    \If{ \Call{Even?}{$x$} }
      \State $T \gets$ \Call{Left}{$T$}
    \Else
      \State $T \gets$ \Call{Right}{$T$}
    \EndIf
    \State $k \gets \lfloor k/2 \rfloor$
  \EndWhile
  \If{$T \neq $ NIL}
    \State \Return \Call{Data}{$T$}
  \Else
    \State \Return not found \EndIf
\EndFunction
\end{algorithmic}

Below Python example code uses bit-wise operation to implements the
looking up algorithm.

\lstset{language=Python}
\begin{lstlisting}
def lookup(t, key):
    while key != 0 and (t is not None):
        if key & 1 == 0:
            t = t.left
        else:
            t = t.right
        key = key>>1
    if t is not None:
        return t.value
    else:
        return None
\end{lstlisting}

Looking up can also be define in recusrive manner. If the tree is
empty, the lookup fails; If $k=0$, the satellite data is the result
to be found; If the last bit is 0, we recurisvely look up the
left child; otherwise, we look up the right child.

\be
lookup(T, k) =  \left \{
  \begin{array}
  {r@{\quad:\quad}l}
  \Phi & T = \Phi \\
  d & k = 0 \\
  lookup(T_l, k / 2) & even(k) \\
  lookup(T_r, \lfloor k / 2 \rfloor) & otherwise
  \end{array}
\right.
\ee

The following Haskell example program implements the recursive
look up algorithm.

\lstset{language=Haskell}
\begin{lstlisting}
search Empty k = Nothing
search t 0 = value t
search t k = if even k then search (left t) (k `div` 2)
             else search (right t) (k `div` 2)
\end{lstlisting}

The looking up algorithm is bound to $O(m)$ time, where $m$ is the
number of bits for a given key.

% ================================================================
%               Int Patricia
% ================================================================
\section{Integer Patricia}
\label{int-patricia}

Trie has some drawbacks. It wasts a lot of
spaces. Note in figure \ref{int-trie}, only leafs store the real data.
Typically, the integer binary trie contains many nodes only have one child.
One improvement idea is to compress the chained nodes together.
Patricia is such a data structure invented by
Donald R. Morrison in 1968. Patricia means practical algorithm to retrieve information coded
in alphanumeric\cite{patricia-morrison}. It is another kind of
prefix tree.

Okasaki gives implementation of integer Patricia in \cite{okasaki-int-map}.
If we merge the chained nodes which have only one child together in figure \ref{fig:int-trie},
We can get a patricia as shown in figure \ref{fig:little-endian-patricia}.

\begin{figure}[htbp]
  \centering
  \includegraphics[scale=0.5]{img/little-endian-patricia.ps}
  \caption{Little endian patricia for the map
     \{$ 1 \rightarrow a, 4 \rightarrow b, 5 \rightarrow c, 9 \rightarrow d$\}.}
  \label{fig:little-endian-patricia}
\end{figure}

From this figure, we can find that the key for the sibling nodes is the
longest common prefix for them.
They branches out at certain bit. Patricia saves a lot of spaces compare
to trie.

Different from integer trie, using the big-endian integer in Patricia
doesn't cause the padding zero problem mentioned
in section \ref{int-trie}. All zero bits before MSB are omitted to save
space. Okasaki lists some significant advantages
of big-endian Patricia\cite{okasaki-int-map}.

% ================================================================
%                 Definition of int patricia tree
% ================================================================
\subsection{Definition of Integer Patricia tree}
Integer Patricia tree is a special kind of binary tree, it is
\begin{itemize}
\item either a leaf node contains an integer key and a value
\item or a branch node, contains a left child and a right child. The
integer keys of two children shares the longest common prefix bits,
the next bit of the left child's key is zero while it is one for right
child's key.
\end{itemize}

\subsubsection*{Definition of big-endian integer Patricia tree in Haskell}
If we translate the above recursive definition to Haskell, we can get
below Integer Patrica Tree code.

\lstset{language=Haskell}
\begin{lstlisting}
data IntTree a = Empty
               | Leaf Key a
               | Branch Prefix Mask (IntTree a) (IntTree a) -- prefix, mask, left, right

type Key = Int
type Prefix = Int
type Mask = Int
\end{lstlisting}

In order to tell from which bit the left and right children differ, a
mask is recorded by the branch node. Typically, a mask is $2^n$, all
lower bits than n doesn't belong to common prefix

\subsubsection*{Definition of big-endian integer Patricia tree in Python}
Such definition can be represent in Python similarly. Some helper
functions are provided for easy operation later on.

\lstset{language=Python}
\begin{lstlisting}
class IntTree:
    def __init__(self, key = None, value = None):
        self.key = key
        self.value = value
        self.prefix = self.mask = None
        self.left = self.right = None

    def set_children(self, l, r):
        self.left = l
        self.right = r

    def replace_child(self, x, y):
        if self.left == x:
            self.left = y
        else:
            self.right = y

    def is_leaf(self):
        return self.left is None and self.right is None

    def get_prefix(self):
        if self.prefix is None:
            return self.key
        else:
            return self.prefix
\end{lstlisting}

Some helper member functions are provided in this definition. When
Initialized, prefix, mask and children are all set to invalid value.
Note the get\_prefix() function, in case the prefix hasn't been
initialized, which means it is a leaf node, the key itself is returned.

\subsubsection*{Definition of big-endian integer Patricia tree in C++}

With ISO C++, the type of the data stored in Patricia can be abstracted
as a template parameter. The definition is similar to the python version.

\lstset{language=C++}
\begin{lstlisting}
template<class T>
struct IntPatricia{
  IntPatricia(int k=0, T v=T()):
    key(k), value(v), prefix(k), mask(1), left(0), right(0){}

  ~IntPatricia(){
    delete left;
    delete right;
  }

  bool is_leaf(){
    return left == 0 && right == 0;
  }

  bool match(int x){
    return (!is_leaf()) && (maskbit(x, mask) == prefix);
  }

  void replace_child(IntPatricia<T>* x, IntPatricia<T>* y){
    if(left == x)
      left = y;
    else
      right = y;
  }

  void set_children(IntPatricia<T>* l, IntPatricia<T>* r){
    left = l;
    right = r;
  }

  int key;
  T value;
  int prefix;
  int mask;
  IntPatricia* left;
  IntPatricia* right;
};
\end{lstlisting}

In order to release the memory easily, the program just recursively
deletes the children in destructor. The default value of type T
is used for initialization. The prefix is initialized to be the same
value as key.

For the member function match(), I'll explain it in later part.

\subsubsection*{Definition of big-endian integer Patricia tree in
Scheme/Lisp}

In Scheme/Lisp program, the data structure behind is still list, we
provide creator functions and accessors to create Patricia and access
the children, key, value, prefix and mask.

\lstset{language=lisp}
\begin{lstlisting}
(define (make-leaf k v) ;; key and value
  (list k v))

(define (make-branch p m l r) ;; prefix, mask, left and right
  (list p m l r))

;; Helpers
(define (leaf? t)
  (= (length t) 2))

(define (branch? t)
  (= (length t) 4))

(define (key t)
  (if (leaf? t) (car t) '()))

(define (value t)
  (if (leaf? t) (cadr t) '()))

(define (prefix t)
  (if (branch? t) (car t) '()))

(define (mask t)
  (if (branch? t) (cadr t) '()))

(define (left t)
  (if (branch? t) (caddr t) '()))

(define (right t)
  (if (branch? t) (cadddr t) '()))
\end{lstlisting}

Function key and value are only applicable to leaf node while prefix,
mask, children accessors are only applicable to branch node. So we
test the node type in these functions.

% ================================================================
%                 Insertion of int patricia tree
% ================================================================
\subsection{Insertion of Integer Patricia tree}
When insert a key into a integer Patricia tree, if the tree is empty,
we can just create a leaf node with the given key and data. (as shown
in figure \ref{fig:int-patricia-insert-a}).

\begin{figure}[htbp]
       \begin{center}
	\includegraphics[scale=1]{img/int-patricia-insert-a.ps}
        \caption{(a). Insert key 12 to an empty patricia tree.}
        \label{fig:int-patricia-insert-a}
       \end{center}
\end{figure}

If the tree only contains a leaf node x, we can create a branch, put the new
key and data as a leaf y of the branch. To determine if the new leaf y
should be left node or right node. We need find the longest common prefix
of x and y, for example if key(x) is 12 (1100 in binary), key(y) is 15
(1111 in binary), then the longest common prefix is $11oo$. The $o$
denotes the bits we don't care about. we can use an integer to mask
the those bits. In this case, the mask number is 4 (100 in binary).
The next bit after the prefix presents $2^1$. It's 0 in key(x), while
it is 1 in key(y). So we put x as left child and y as right
child. Figure \ref{fig:int-patricia-insert-b} shows this case.

\begin{figure}[htbp]
       \begin{center}
	\includegraphics[scale=1]{img/int-patricia-insert-b.ps}
        \caption{(b). Insert key 15 to the result tree in (a).}
        \label{fig:int-patricia-insert-b}
       \end{center}
\end{figure}

If the tree is neither empty, nor a leaf node, we need firstly check
if the key to be inserted matches common prefix with root node. If it
does, then we can recursively insert the key to the left child or right child
according to the next bit. For instance, if we want to
insert key 14 (1110 in binary) to the result tree in figure
\ref{fig:int-patricia-insert-b}, since it has common prefix $11oo$,
and the next bit (the bit of $2^1$) is 1, so we tried to insert 14 to
the right child. Otherwise, if the key to be inserted doesn't match the
common prefix with the root node, we need branch a new leaf
node. Figure \ref{fig:int-patricia-insert-c} shows these 2 different cases.

\begin{figure}[htbp]
       \begin{center}
	\includegraphics[scale=0.5]{img/int-patricia-insert-c.ps}
	\includegraphics[scale=0.5]{img/int-patricia-insert-d.ps}
        \caption{(c). Insert key 14 to the result tree in (b);
	(d). Insert key 5 to the result tree in (b).}
        \label{fig:int-patricia-insert-c}
       \end{center}
\end{figure}

\subsubsection{Iterative insertion algorithm for integer Patricia}

Summarize the above cases, the insertion of integer patricia can be described
with the following algorithm.

\begin{algorithmic}[1]
\Function{INT-PATRICIA-INSERT}{$T, x, data$}
\If{$T = NIL$}
   \State $T \leftarrow CREATE-LEAF(x, data)$
   \State \Return $T$
\EndIf

\State $y \leftarrow T$
\State $p \leftarrow NIL$
\While{$y$ is not $LEAF$ and $MATCH(x, PREFIX(y), MASK(y))$}
  \State $p \leftarrow y$
  \If{$ZERO(x, MASK(y)) = TRUE$}
    \State $y \leftarrow LEFT(y)$
  \Else
    \State $y \leftarrow RIGHT(y)$
  \EndIf
\EndWhile

\If{$LEAF(y) = TRUE$ and $x = KEY(y)$}
  \State $DATA(y) \leftarrow data$
\Else
  \State $z \leftarrow BRANCH(y, CREATE-LEAF(x, data))$
  \If{$p = NIL$}
    \State $T \leftarrow z$
  \Else
    \If{$LEFT(p) = y$}
      \State $LEFT(p) \leftarrow z$
    \Else
      \State $RIGHT(p) \leftarrow z$
    \EndIf
  \EndIf
\EndIf
\State \Return $T$
\EndFunction
\end{algorithmic}

In the above algorithm, MATCH procedure test if an integer key $x$, has
the same prefix of node y above the mask bit. For instance,
Suppose the prefix of node y can be denoted as
$p(n), p(n-1), ..., p(i), ..., p(0)$ in binary, key x is
$k(n), k(n-1), ..., k(i), ..., k(0)$, and mask of node y is
$100...0=2^i$, if and only if $p(j)=k(j)$ for all $i \leq j \leq n$,
we say the key matches.

\subsubsection*{Insertion of big-endian integer Patricia tree in Python}

Based on the above algorithm, the main insertion program can be realized
as the following.

\begin{lstlisting}
def insert(t, key, value = None):
    if t is None:
        t = IntTree(key, value)
        return t

    node = t
    parent = None
    while(True):
        if match(key, node):
            parent = node
            if zero(key, node.mask):
                node = node.left
            else:
                node = node.right
        else:
            if node.is_leaf() and key == node.key:
                node.value = value
            else:
                new_node = branch(node, IntTree(key, value))
                if parent is None:
                    t = new_node
                else:
                    parent.replace_child(node, new_node)
            break
    return t
\end{lstlisting}

The sub procedure of match, branch, lcp etc. are given as below.

\begin{lstlisting}
def maskbit(x, mask):
    return x & (~(mask-1))

def match(key, tree):
    if tree.is_leaf():
        return False
    return maskbit(key, tree.mask) == tree.prefix

def zero(x, mask):
    return x & (mask>>1) == 0

def lcp(p1, p2):
    diff = (p1 ^ p2)
    mask=1
    while(diff!=0):
        diff>>=1
        mask<<=1
    return (maskbit(p1, mask), mask)

def branch(t1, t2):
    t = IntTree()
    (t.prefix, t.mask) = lcp(t1.get_prefix(), t2.get_prefix())
    if zero(t1.get_prefix(), t.mask):
        t.set_children(t1, t2)
    else:
        t.set_children(t2, t1)
    return t
\end{lstlisting}

Function maskbit() can clear all bits covered by a mask to 0. For instance,
$x = 101101(b)$, and $mask = 2^3 = 100(b)$, the lowest 2 bits will be cleared to 0,
which means $maskbit(x, mask) = 101100(b)$. This can be easily done by bit-wise
operation.

Function zero() is used to check if the bit next to mask bit is 0. For instance,
if $x = 101101(b), y = 101111(b)$, and $mask = 2^3 = 100(b)$, zero will check
if the 2nd lowest bit is 0. So $zero(x, mask) = true, zero(y, mask) = false$.

Function lcp can extract 'the Longest Common Prefix' of two integer. For the $x$
and $y$ in above example, because only the last 2 bits are different, so
$lcp(x, y) = 101100(b)$. And we set a mask to $2^3 = 100(b)$ to indicate that
the last 2 bits are not effective for the prefix value.

To convert a list or a map into a Patricia tree, we can repeatedly
insert the elements one by one. Since the program is same, except for
the insert function, we can abstract list\_to\_xxx and map\_to\_xxx to
utility functions

\begin{lstlisting}
# in trieutil.py
def from_list(l, insert_func):
    t = None
    for x in l:
        t = insert_func(t, x)
    return t

def from_map(m, insert_func):
    t = None
    for k, v in m.items():
        t = insert_func(t, k, v)
    return t
\end{lstlisting}

With this high level functions, we can provide list\_to\_patricia and
map\_to\_patricia as below.

\begin{lstlisting}
def list_to_patricia(l):
    return from_list(l, insert)

def map_to_patricia(m):
    return from_map(m, insert)
\end{lstlisting}

In order to have smoke test of the above insertion program, some test
cases and output helper are given.

\begin{lstlisting}
def to_string(t):
    to_str = lambda x: "%s" %x
    if t is None:
        return ""
    if t.is_leaf():
        str = to_str(t.key)
        if t.value is not None:
            str += ":"+to_str(t.value)
        return str
    str ="["+to_str(t.prefix)+"@"+to_str(t.mask)+"]"
    str+="("+to_string(t.left)+","+to_string(t.right)+")"
    return str

class IntTreeTest:
    def run(self):
        self.test_insert()

    def test_insert(self):
        print "test insert"
        t = list_to_patricia([6])
        print to_string(t)
        t = list_to_patricia([6, 7])
        print to_string(t)
        t = map_to_patricia({1:'x', 4:'y', 5:'z'})
        print to_string(t)

if __name__ == "__main__":
    IntTreeTest().run()
\end{lstlisting}

The program will output a result as the following.

\begin{verbatim}
test insert
6
[6@2](6,7)
[0@8](1:x,[4@2](4:y,5:z))
\end{verbatim}

This result means the program creates a Patrica tree shown in
Figure \ref{fig:int-patricia-haskell-insert}.

\begin{figure}[htbp]
       \begin{center}
	\includegraphics[scale=1]{img/int-patricia-haskell-insert.ps}
        \caption{Insert map $1 \rightarrow x, 4 \rightarrow y, 5 \rightarrow z$ into a big-endian integer Patricia tree.}
        \label{fig:int-patricia-haskell-insert}
       \end{center}
\end{figure}

\subsubsection*{Insertion of big-endian integer Patricia tree in C++}

In the below C++ program, the default value of data type is used if user
doesn't provide data. It is nearly strict translation of the pseudo code.

\lstset{language=C++}
\begin{lstlisting}
template<class T>
IntPatricia<T>* insert(IntPatricia<T>* t, int key, T value=T()){
  if(!t)
    return new IntPatricia<T>(key, value);

  IntPatricia<T>* node = t;
  IntPatricia<T>* parent(0);

  while( node->is_leaf()==false && node->match(key) ){
    parent = node;
    if(zero(key, node->mask))
      node = node->left;
    else
      node = node->right;
  }

  if(node->is_leaf() && key == node->key)
    node->value = value;
  else{
    IntPatricia<T>* p = branch(node, new IntPatricia<T>(key, value));
    if(!parent)
      return p;
    parent->replace_child(node, p);
  }
  return t;
}
\end{lstlisting}

Let's review the implementation of member function match()

\begin{lstlisting}
bool match(int x){
  return (!is_leaf()) && (maskbit(x, mask) == prefix);
}
\end{lstlisting}

if a node is not a leaf, and it has common prefix (in bit-wise) as the key
to be inserted, we say the node match the key. It is realized by a maskbit()
function as below.

\begin{lstlisting}
int maskbit(int x, int mask){
  return x & (~(mask-1));
}
\end{lstlisting}

Since mask is always $2^n$, minus 1 will flip it to $111...1(b)$, then
we reverse the it by bit-wise not, and clear all the lowest $n-1$ bits of
$x$ by bit-wise and.

The branch() function in above program is as the following.

\begin{lstlisting}
template<class T>
IntPatricia<T>* branch(IntPatricia<T>* t1, IntPatricia<T>* t2){
  IntPatricia<T>* t = new IntPatricia<T>();
  t->mask = lcp(t->prefix, t1->prefix, t2->prefix);
  if(zero(t1->prefix, t->mask))
    t->set_children(t1, t2);
  else
    t->set_children(t2, t1);
  return t;
}
\end{lstlisting}

It will extract the 'Longest Common Prefix', and create a new node, put
the 2 nodes to be merged as its children. Function lcp() is implemented
as below.

\begin{lstlisting}
int lcp(int& p, int p1, int p2){
  int diff = p1^p2;
  int mask = 1;
  while(diff){
    diff>>=1;
    mask<<=1;
  }
  p = maskbit(p1, mask);
  return mask;
}
\end{lstlisting}

Because we can only return one value in C++, we set the reference of
parameter p as the common prefix result and returns the mask value.

To decide which child is left and which one is right when branching,
we need test if the bit next to mask bit is zero.

\begin{lstlisting}
bool zero(int x, int mask){
  return (x & (mask>>1)) == 0;
}
\end{lstlisting}

To verify the C++ program, some simple test cases are provided.

\begin{lstlisting}
IntPatricia<int>* ti(0);
const int lst[] = {6, 7};
ti = std::accumulate(lst, lst+sizeof(lst)/sizeof(int), ti,
                     std::ptr_fun(insert_key<int>));
std::copy(lst, lst+sizeof(lst)/sizeof(int),
std::ostream_iterator<int>(std::cout, ", "));
std::cout<<"==>"<<patricia_to_str(ti)<<"\n";

const int keys[] = {1, 4, 5};
const char vals[] = "xyz";
IntPatricia<char>* tc(0);
for(unsigned int i=0; i<sizeof(keys)/sizeof(int); ++i)
  tc = insert(tc, keys[i], vals[i]);
std::copy(keys, keys+sizeof(keys)/sizeof(int),
   std::ostream_iterator<int>(std::cout, ", "));
std::cout<<"==>"<<patricia_to_str(tc);
\end{lstlisting}

To avoid repeating ourselves, we provide a different way instead of
write a list\_to\_patrica(), which is very similar to list\_to\_trie
in previous section.

In C++ STL, std::accumulate() plays a similar role of fold-left. But the
functor we provide to accumulate must take 2 parameters, so we provide a
wrapper function as below.

\begin{lstlisting}
template<class T>
IntPatricia<T>* insert_key(IntPatricia<T>* t, int key){
  return insert(t, key);
}
\end{lstlisting}

With all these code line, we can get the following result.
\begin{verbatim}
6, 7, ==>[6@2](6,7)
1, 4, 5, ==>[0@8](1:x,[4@2](4:y,5:z))
\end{verbatim}

\subsubsection{Recursive insertion algorithm for integer Patricia}

To implement insertion in recursive way, we treat the different cases
separately. If the tree is empty, we just create a leaf node and
return; if the tree is a leaf node, we need check the key of the
node is as same as the key to be inserted, we overwrite the data in
case they are same, else we need branch a new node and extract the
longest common prefix and mask bit; In other case, we need examine if
the key as common prefix with the branch node, and recursively perform
insertion either to left child or to right child according to the next
different bit is 0 or 1; Below recursive algorithm describes this approach.

\begin{algorithmic}[1]
\Function{INT-PATRICIA-INSERT'}{$T, x, data$}
\If{$T = NIL$ or ($T$ is a leaf and $x=KEY(T)$)}
   \State \Return $CREATE-LEAF(x, data)$
\ElsIf{$MATCH(x, PREFIX(T), MASK(T))$}
   \If{$ZERO(x, MASK(T))$}
     \State $LEFT(T) \leftarrow INT-PATRICIA-INSERT'(LEFT(T), x, data)$
   \Else
     \State $RIGHT(T) \leftarrow INT-PATRICIA-INSERT'(RIGHT(T), x, data)$
   \EndIf
   \State \Return $T$
\Else
   \State \Return $BRANCH(T, CREATE-LEAF(x, data))$
\EndIf
\EndFunction
\end{algorithmic}

\subsubsection*{Insertion of big-endian integer Patricia tree in Haskell}
Insertion of big-endian integer Patricia tree can be implemented in Haskell
by Change the above algorithm to recursive approach.

\lstset{language=Haskell}
\begin{lstlisting}
-- usage: insert tree key x
insert :: IntTree a -> Key -> a -> IntTree a
insert t k x
   = case t of
       Empty -> Leaf k x
       Leaf k' x' -> if k==k' then Leaf k x
                     else join k (Leaf k x) k' t -- t@(Leaf k' x')
       Branch p m l r
          | match k p m -> if zero k m
                           then Branch p m (insert l k x) r
                           else Branch p m l (insert r k x)
          | otherwise -> join k (Leaf k x) p t -- t@(Branch p m l r)
\end{lstlisting}

The match, zero and join functions in this program are defined as below.
\begin{lstlisting}
-- join 2 nodes together.
-- (prefix1, tree1) ++ (prefix2, tree2)
--  1. find the longest common prefix == lcp(prefix1, prefix2), where
--         prefix1 = a(n),a(n-1),...a(i+1),a(i),x...
--         prefix2 = a(n),a(n-1),...a(i+1),a(i),y...
--         prefix  = a(n),a(n-1),...a(i+1),a(i),00...0
--  2. mask bit = 100...0b (=2^i)
--         so mask is something like, 1,2,4,...,128,256,...
--  3. if      x=='0', y=='1' then (tree1=>left, tree2=>right),
--     else if x=='1', y=='0' then (tree2=>left, tree1=>right).
join :: Prefix -> IntTree a -> Prefix -> IntTree a -> IntTree a
join p1 t1 p2 t2 = if zero p1 m then Branch p m t1 t2
                                else Branch p m t2 t1
    where
      (p, m) = lcp p1 p2

-- 'lcp' means 'longest common prefix'
lcp :: Prefix -> Prefix -> (Prefix, Mask)
lcp p1 p2 = (p, m) where
    m = bit (highestBit (p1 `xor` p2))
    p = mask p1 m

-- get the order of highest bit of 1.
-- For a number x = 00...0,1,a(i-1)...a(1)
-- the result is i
highestBit :: Int -> Int
highestBit x = if x==0 then 0 else 1+highestBit (shiftR x 1)

-- For a number x = a(n),a(n-1)...a(i),a(i-1),...,a(0)
-- and a mask m = 100..0 (=2^i)
-- the result of mask x m is a(n),a(n-1)...a(i),00..0
mask :: Int -> Mask -> Int
mask x m = (x .&. complement (m-1)) -- complement means bit-wise not.

-- Test if the next bit after mask bit is zero
-- For a number x = a(n),a(n-1)...a(i),1,...a(0)
-- and a mask   m = 100..0 (=2^i)
-- because the bit next to a(i) is 1, so the result is False
-- For a number y = a(n),a(n-1)...a(i),0,...a(0) the result is True.
zero :: Int -> Mask -> Bool
zero x m = x .&. (shiftR m 1) == 0

-- Test if a key matches a prefix above of the mask bit
-- For a prefix: p(n),p(n-1)...p(i)...p(0)
--     a key:    k(n),k(n-1)...k(i)...k(0)
-- and a mask:                 100..0 = (2^i)
-- If and only if p(j)==k(j), i<=j<=n the result is True
match :: Key -> Prefix -> Mask -> Bool
match k p m = (mask k m) == p
\end{lstlisting}

In order to test the above insertion program, some test helper functions
are provided.

\begin{lstlisting}
-- Generate a Int Patricia tree from a list
-- Usage: fromList [(k1, x1), (k2, x2),..., (kn, xn)]
fromList :: [(Key, a)] -> IntTree a
fromList xs = foldl ins Empty xs where
    ins t (k, v) = insert t k v

toString :: (Show a)=>IntTree a -> String
toString t =
    case t of
      Empty -> "."
      Leaf k x -> (show k) ++ ":" ++ (show x)
      Branch p m l r -> "[" ++ (show p) ++ "@" ++ (show m) ++ "]" ++
                        "(" ++ (toString l) ++ ", " ++ (toString r) ++ ")"

\end{lstlisting}

With these helpers, insertion can be test as the following.

\begin{lstlisting}
testIntTree = "t=" ++ (toString t)
    where
      t = fromList [(1, 'x'), (4, 'y'), (5, 'z')]

main = do
  putStrLn testIntTree
\end{lstlisting}

This test will output:

\begin{verbatim}
t=[0@8](1:'x', [4@2](4:'y', 5:'z'))
\end{verbatim}

This result means the program creates a Patrica tree shown in
Figure \ref{fig:int-patricia-haskell-insert}.

\subsubsection*{Insertion of big-endian integer Patricia tree in
Scheme/Lisp}

In Scheme/Lisp, we use switch-case like condition to test if the node
is empty, or a leaf or a branch.

\lstset{language=lisp}
\begin{lstlisting}
(define (insert t k x) ;; t: patrica, k: key, x: value
  (cond ((null? t) (make-leaf k x))
	((leaf? t) (if (= (key t) k)
		       (make-leaf k x) ;; overwrite
		       (branch k (make-leaf k x) (key t) t)))
	((branch? t) (if (match? k (prefix t) (mask t))
			 (if (zero-bit? k (mask t))
			     (make-branch (prefix t)
					  (mask t)
					  (insert (left t) k x)
					  (right t))
			     (make-branch (prefix t)
					  (mask t)
					  (left t)
					  (insert (right t) k x)))
			 (branch k (make-leaf k x) (prefix t) t)))))
\end{lstlisting}

Where the function match?, zero-bit?, and branch are given as the
following. We use the scheme fix number bit-wise operations to mask the
number and test bit.

\begin{lstlisting}
(define (mask-bit x m)
  (fix:and x (fix:not (- m 1))))

(define (zero-bit? x m)
  (= (fix:and x (fix:lsh m -1)) 0))

(define (lcp x y) ;; get the longest common prefix
  (define (count-mask z)
    (if (= z 0) 1 (* 2 (count-mask (fix:lsh z -1)))))
  (let* ((m (count-mask (fix:xor x y)))
	 (p (mask-bit x m)))
    (cons p m)))

(define (match? k p m)
  (= (mask-bit k m) p))

(define (branch p1 t1 p2 t2) ;; pi: prefix i, ti: Patricia i
  (let* ((pm (lcp p1 p2))
	 (p (car pm))
	 (m (cdr pm)))
    (if (zero-bit? p1 m)
	(make-branch p m t1 t2)
	(make-branch p m t2 t1))))
\end{lstlisting}

We can use the very same list->trie function which is defined in
integer trie. Below is an example to create a integer Patricia tree.

\begin{lstlisting}
(define (test-int-patricia)
  (define t (list->trie (list '(1 "x") '(4 "y") '(5 "z"))))
  (display t) (newline))
\end{lstlisting}

Evaluate it will generate a Patricia tree like below.

\begin{lstlisting}
(test-int-patricia)
(0 8 (1 x) (4 2 (4 y) (5 z)))
\end{lstlisting}

It is identical to the insert result output by Hasekll insertion program.

% ================================================================
%                 Lookup in int patricia tree
% ================================================================
\subsection{Look up in Integer Patricia tree}
Consider the property of integer Patricia tree, to look up a
key, we test if the key has common prefix with the root, if yes, we
then check the next bit differs from common prefix is zero or one. If
it is zero, we then do look up in the left child, else we turn to
right.

\subsubsection{Iterative looking up in integer Patricia tree}

In case we reach a leaf node, we can directly check if the key of the
leaf is equal to what we are looking up. This algorithm can be
described with the following pseudo code.

\begin{algorithmic}[1]
\Function{INT-PATRICIA-LOOK-UP}{$T, x$}
  \If{$T = NIL$}
    \State \Return $NIL$ \EndIf

  \While{$T$ is not $LEAF$ and $MATCH(x, PREFIX(T), MASK(T))$}
    \If{$ZERO(x, MASK(T))$}
      \State $T \leftarrow LEFT(T)$
    \Else
      \State $T \leftarrow RIGHT(T)$
    \EndIf
  \EndWhile

  \If{$T$ is $LEAF$ and $KEY(T)=x$}
    \State \Return $DATA(T)$
  \Else
    \State \Return $NIL$
  \EndIf
\EndFunction
\end{algorithmic}

\subsubsection*{Look up in big-endian integer Patricia tree in Python}
With Python, we can directly translate the pseudo code into valid
program.

\lstset{language=Python}
\begin{lstlisting}
def lookup(t, key):
    if t is None:
        return None
    while (not t.is_leaf()) and match(key, t):
        if zero(key, t.mask):
            t = t.left
        else:
            t = t.right
    if t.is_leaf() and t.key == key:
        return t.value
    else:
        return None
\end{lstlisting}

We can verify this program by some simple smoke test cases.

\begin{lstlisting}
print "test look up"
t = map_to_patricia({1:'x', 4:'y', 5:'z'})
print "look up 4: ", lookup(t, 4)
print "look up 0: ", lookup(t, 0)
\end{lstlisting}

We can get similar output as below.

\begin{verbatim}
test look up
look up 4:  y
look up 0:  None
\end{verbatim}

\subsubsection*{Look up in big-endian integer Patricia tree in C++}

With C++ language, if the program doesn't find the key, we can either
raise exception to indicate a search failure or return a special
value.

\lstset{language=C++}
\begin{lstlisting}
template<class T>
T lookup(IntPatricia<T>* t, int key){
  if(!t)
    return T(); //or throw exception

  while( (!t->is_leaf()) && t->match(key)){
    if(zero(key, t->mask))
      t = t->left;
    else
      t = t->right;
  }
  if(t->is_leaf() && t->key == key)
    return t->value;
  else
    return T(); //or throw exception
}
\end{lstlisting}

We can try some test cases to search keys in a Patricia tree we
created when test insertion.

\begin{lstlisting}
std::cout<<"\nlook up 4: "<<lookup(tc, 4)
         <<"\nlook up 0: "<<lookup(tc, 0)<<"\n";
\end{lstlisting}

The output result is as the following.

\begin{verbatim}
look up 4: y
look up 0:
\end{verbatim}

\subsubsection{Recursive looking up in integer Patricia tree}

We can easily change the while-loop in above iterative algorithm into
recursive calls, so that we can have a functional approach.

\begin{algorithmic}[1]
\Function{INT-PATRICIA-LOOK-UP'}{$T, x$}
  \If{$T = NIL$}
    \State \Return $NIL$
  \ElsIf{$T$ is a leaf and $x = KEY(T)$}
    \State \Return $VALUE(T)$
  \ElsIf{$MATCH(x, PREFIX(T), MASK(T))$}
    \If{$ZERO(x, MASK(T))$}
      \State \Return $INT-PATRICIA-LOOK-UP'(LEFT(T), x)$
    \Else
      \State \Return $INT-PATRICIA-LOOK-UP'(RIGHT(T), x)$
    \EndIf
  \Else
    \State \Return $NIL$
  \EndIf
\EndFunction
\end{algorithmic}

\subsubsection*{Look up in big-endian integer Patricia tree in Haskell}
By changing the above if-then-else into pattern matching, we can get
Haskell version of looking up program.

\lstset{language=Haskell}
\begin{lstlisting}
-- look up a key
search :: IntTree a -> Key -> Maybe a
search t k
  = case t of
      Empty -> Nothing
      Leaf k' x -> if k==k' then Just x else Nothing
      Branch p m l r
             | match k p m -> if zero k m then search l k
                              else search r k
             | otherwise -> Nothing
\end{lstlisting}

And we can test this program with looking up some keys in the
previously created Patricia tree.

\begin{lstlisting}
testIntTree = "t=" ++ (toString t) ++ "\nsearch t 4: " ++ (show $ search t 4) ++
              "\nsearch t 0: " ++ (show $ search t 0)
    where
      t = fromList [(1, 'x'), (4, 'y'), (5, 'z')]

main = do
  putStrLn testIntTree
\end{lstlisting}

The output result is as the following.

\begin{verbatim}
t=[0@8](1:'x', [4@2](4:'y', 5:'z'))
search t 4: Just 'y'
search t 0: Nothing
\end{verbatim}

\subsubsection*{Look up in big-endian integer Patricia tree in
Scheme/Lisp}

Scheme/Lisp program for looking up is similar in case the tree is
empty, we just returns nothing; If it is a leaf node and the key is
equal to the number we are looking for, we find the result; If it is
branch, we need test if binary format of the prefix matches the
number, then we recursively search either in left child or in right
child according to the next bit after mask is zero or not.

\lstset{language=lisp}
\begin{lstlisting}
(define (lookup t k)
  (cond ((null? t) '())
	((leaf? t) (if (= (key t) k) (value t) '()))
	((branch? t) (if (match? k (prefix t) (mask t))
			 (if (zero-bit? k (mask t))
			     (lookup (left t) k)
			     (lookup (right t) k))
			 '()))))
\end{lstlisting}

We can test it with the Patricia tree we create in the insertion
program.

\begin{lstlisting}
(define (test-int-patricia)
  (define t (list->trie (list '(1 "x") '(4 "y") '(5 "z"))))
  (display t) (newline)
  (display "lookup 4: ") (display (lookup t 4)) (newline)
  (display "lookup 0: ") (display (lookup t 0)) (newline))
\end{lstlisting}

The result is like below.

\begin{lstlisting}
(test-int-patricia)
(0 8 (1 x) (4 2 (4 y) (5 z)))
lookup 4: y
lookup 0: ()
\end{lstlisting}

% ================================================================
%                 Alphabetic trie
% ================================================================
\section{Alphabetic Trie}
Integer based Trie and Patricia Tree can be a good start point. Such
technical plays important role in Compiler implementation. Okasaki
pointed that the widely used Haskell Compiler GHC (Glasgow Haskell
Compiler), utilizes a similar implementation for several years before
1998 \cite{okasaki-int-map}.

While if we extend the type of the key from integer to alphabetic
value, Trie and Patricia tree can be very useful in textual manipulation
engineering problems.

% ================================================================
%                 Definition of Alphabetic trie
% ================================================================
\subsection{Definition of alphabetic Trie}
If the key is alphabetic value, just left and right children can't
represent all values. For English, there are 26 letters and each can
be lower case or upper case. If we don't care about case, one solution
is to limit the number of branches (children) to 26. Some simplified
ANSI C implementation of Trie are defined by using an array of 26
letters. This can be illustrated as in Figure \ref{fig:trie-of-26}.

\begin{figure}[htbp]
  \begin{center}
    \includegraphics[scale=0.5]{img/trie-of-26.ps}
      \caption{A Trie with 26 branches, with key a, an, another, bool,
    boy and zoo inserted.}
      \label{fig:trie-of-26}
  \end{center}
\end{figure}

In each node, not all branches may contain data. for instance, in the
above figure, the root node only has its branches represent letter a,
b, and z have sub trees. Other branches such as for letter c, is
empty. For other nodes, empty branches (point to nil) are not shown.

I'll give such simplified implementation in ANSI C in later section,
however, before we go to the detailed source code, let's consider some
alternatives.

In case of language other than English, there may be more letters than 26,
and if we need
solve case sensitive problem. we face a problem of dynamic size of sub
branches. There are 2 typical method to represent children, one is by
using Hash table, the other is by using map. We'll show these two types
of method in Python and C++.

\subsubsection*{Definition of alphabetic Trie in ANSI C}
ANSI C implementation is to illustrate a simplified approach limited only
to case-insensitive English language. The program can't deal with letters
other than lower case 'a' to 'z' such as digits, space, tab etc.

\lstset{language=C}
\begin{lstlisting}
struct Trie{
  struct Trie* children[26];
  void* data;
};
\end{lstlisting}

In order to initialize/destroy the children and data, I also provide 2 helper
functions.

\begin{lstlisting}
struct Trie* create_node(){
  struct Trie* t = (struct Trie*)malloc(sizeof(struct Trie));
  int i;
  for(i=0; i<26; ++i)
    t->children[i]=0;
  t->data=0;
  return t;
}

void destroy(struct Trie* t){
  if(!t)
    return;

  int i;
  for(i=0; i<26; ++i)
    destroy(t->children[i]);

  if(t->data)
    free(t->data);
  free(t);
}
\end{lstlisting}

Note that, the destroy function uses recursive approach to free all
children nodes.

\subsubsection*{Definition of alphabetic Trie in C++}

With C++ and STL, we can abstract the language and characters as type
parameter. Since the number of characters of the undetermined language varies, we
can use std::map to store children of a node.

\lstset{language=C++}
\begin{lstlisting}
template<class Char, class Value>
struct Trie{
  typedef Trie<Char, Value> Self;
  typedef std::map<Char, Self*> Children;
  typedef Value ValueType;

  Trie():value(Value()){}

  virtual ~Trie(){
    for(typename Children::iterator it=children.begin();
        it!=children.end(); ++it)
      delete it->second;
  }

  Value value;
  Children children;
};
\end{lstlisting}

For simple illustration purpose, recursive destructor is used to
release the memory.

\subsubsection*{Definition of alphabetic Trie in Haskell}
We can use Haskell record syntax to get some ``free'' accessor
functions\cite{wiki-trie}.

\lstset{language=Haskell}
\begin{lstlisting}
data Trie a = Trie { value :: Maybe a
                   , children :: [(Char, Trie a)]}

empty = Trie Nothing []
\end{lstlisting}

Neither Map nor Hash table is used, just a list of pairs can realize
the same purpose. Function empty can help to create an empty Trie
node. This implementation doesn't constrain the key values to lower
case English letters, it can actually contains any values of 'Char' type.

\subsubsection*{Definition of alphabetic Trie in Python}
In Python version, we can use Hash table as the data structure to
represent children nodes.

\lstset{language=Python}
\begin{lstlisting}
class Trie:
    def __init__(self):
        self.value = None
        self.children = {}
\end{lstlisting}

\subsubsection*{Definition of alphabetic Trie in Scheme/Lisp}

The definition of alphabetic Trie in Scheme/Lisp is a list of two
elements, one is the value of the node, the other is a children
list. The children list is a list of pairs, one is the character
binding to the child, the other is a Trie node.

\lstset{language=lisp}
\begin{lstlisting}
(define (make-trie v lop) ;; v: value, lop: children, list of char-trie pairs
  (cons v lop))

(define (value t)
  (if (null? t) '() (car t)))

(define (children t)
  (if (null? t) '() (cdr t)))
\end{lstlisting}

In order to create the child and access it easily, we also provide
functions for such purpose.

\begin{lstlisting}
(define (make-child k t)
  (cons k t))

(define (key child)
  (if (null? child) '() (car child)))

(define (tree child)
  (if (null? child) '() (cdr child)))
\end{lstlisting}

% ================================================================
%                 Insertion of Alphabetic trie
% ================================================================
\subsection{Insertion of alphabetic trie}
To insert a key with type of string into a Trie, we pick the first letter
from the key string. Then check from the root node, we examine which branch
among the children represents this letter. If the branch is null, we then
create an empty node. After that, we pick the next letter from the key string
and pick the proper branch from the grand children of the root.

We repeat the above process till finishing all the letters of the key.
At this time point, we can finally set the data to be inserted as the value
of the node.

Note that the value of root node of Trie is always empty.

\subsubsection{Iterative algorithm of trie insertion}

The below pseudo code describes the above insertion algorithm.

\begin{algorithmic}[1]
\Function{TRIE-INSERT}{$T, key, data$}
  \If{$T = NIL$}
    \State $T \leftarrow EmptyNode$ \EndIf

  \State $p=T$
  \For{each $c$ in $key$}
    \If{$CHILDREN(p)[c] = NIL$}
      \State $CHILDREN(p)[c] \leftarrow EmptyNode$
    \EndIf
    \State $p \leftarrow CHILDREN(p)[c]$
  \EndFor
  \State $DATA(p) \leftarrow data$
  \State \Return $T$
\EndFunction
\end{algorithmic}

\subsubsection*{Simplified insertion of alphabetic trie in ANSI C}
Go on with the above ANSI C definition, because only lower case
English letter is supported, we can use plan array manipulation
to do the insertion.

\lstset{language=C}
\begin{lstlisting}
struct Trie* insert(struct Trie* t, const char* key, void* value){
  if(!t)
    t=create_node();

  struct Trie* p =t;
  while(*key){
    int c = *key - 'a';
    if(!p->children[c])
      p->children[c] = create_node();
    p = p->children[c];
    ++key;
  }
  p->data = value;
  return t;
}
\end{lstlisting}

In order to test the above program, some helper functions
to print content of the Trie is provided as the following.

\begin{lstlisting}
void print_trie(struct Trie* t, const char* prefix){
  printf("(%s", prefix);
  if(t->data)
    printf(":%s", (char*)(t->data));
  int i;
  for(i=0; i<26; ++i){
    if(t->children[i]){
      printf(", ");
      char* new_prefix=(char*)malloc(strlen(prefix+1)*sizeof(char));
      sprintf(new_prefix, "%s%c", prefix, i+'a');
      print_trie(t->children[i], new_prefix);
    }
  }
  printf(")");
}
\end{lstlisting}

After that, we can test the insertion program with such test
cases.

\begin{lstlisting}
struct Trie* test_insert(){
  struct Trie* t=0;
  t = insert(t, "a", 0);
  t = insert(t, "an", 0);
  t = insert(t, "another", 0);
  t = insert(t, "boy", 0);
  t = insert(t, "bool", 0);
  t = insert(t, "zoo", 0);
  print_trie(t, "");
  return t;
}

int main(int argc, char** argv){
  struct Trie* t = test_insert();
  destroy(t);
  return 0;
}
\end{lstlisting}

This program will output a Trie like this.

\begin{verbatim}
(, (a, (an, (ano, (anot, (anoth, (anothe, (another))))))),
(b, (bo, (boo, (bool)), (boy))), (z, (zo, (zoo))))
\end{verbatim}

It is exactly the Trie as shown in figure \ref{fig:trie-of-26}.

\subsubsection*{Insertion of alphabetic Trie in C++}

With above C++ definition, we can utilize STL provided search
function in std::map to locate a child quickly, the program is
implemented as the following, note that if user only provides key for
insert, we also insert a default value of that type.

\lstset{language=C++}
\begin{lstlisting}
template<class Char, class Value, class Key>
Trie<Char, Value>* insert(Trie<Char, Value>* t, Key key, Value value=Value()){
  if(!t)
    t = new Trie<Char, Value>();

  Trie<Char, Value>* p(t);
  for(typename Key::iterator it=key.begin(); it!=key.end(); ++it){
    if(p->children.find(*it) == p->children.end())
      p->children[*it] = new Trie<Char, Value>();
    p = p->children[*it];
  }
  p->value = value;
  return t;
}

template<class T, class K>
T* insert_key(T* t, K key){
  return insert(t, key);
}
\end{lstlisting}

Where insert\_key() acts as a adapter, we'll use similar accumulation
method to create trie from list later.

To test this program, we provide the helper functions to print the
trie on console.

\begin{lstlisting}
template<class T>
std::string trie_to_str(T* t, std::string prefix=""){
  std::ostringstream s;
  s<<"("<<prefix;
  if(t->value != typename T::ValueType())
    s<<":"<<t->value;
  for(typename T::Children::iterator it=t->children.begin();
      it!=t->children.end(); ++it)
    s<<", "<<trie_to_str(it->second, prefix+it->first);
  s<<")";
  return s.str();
}
\end{lstlisting}

After that, we can test our program with some simple test cases.

\begin{lstlisting}
typedef Trie<char, std::string> TrieType;
TrieType* t(0);
const char* lst[] = {"a", "an", "another", "b", "bob", "bool", "home"};
t = std::accumulate(lst, lst+sizeof(lst)/sizeof(char*), t,
                    std::ptr_fun(insert_key<TrieType, std::string>));
std::copy(lst, lst+sizeof(lst)/sizeof(char*),
          std::ostream_iterator<std::string>(std::cout, ", "));
std::cout<<"\n==>"<<trie_to_str(t)<<"\n";
delete t;

t=0;
const char* keys[] = {"001", "100", "101"};
const char* vals[] = {"y", "x", "z"};
for(unsigned int i=0; i<sizeof(keys)/sizeof(char*); ++i)
  t = insert(t, std::string(keys[i]), std::string(vals[i]));
std::copy(keys, keys+sizeof(keys)/sizeof(char*),
          std::ostream_iterator<std::string>(std::cout, ", "));
std::cout<<"==>"<<trie_to_str(t)<<"\n";
delete t;
\end{lstlisting}

It will output result like this.

\begin{verbatim}
a, an, another, b, bob, bool, home,
==>(, (a, (an, (ano, (anot, (anoth, (anothe, (another))))))), (b, (bo,
(bob), (boo, (bool)))), (h, (ho, (hom, (home)))))
001, 100, 101, ==>(, (0, (00, (001:y))), (1, (10, (100:x), (101:z))))
\end{verbatim}

\subsubsection*{Insertion of alphabetic trie in Python}
In python the implementation is very similar to the pseudo code.

\lstset{language=Python}
\begin{lstlisting}
def trie_insert(t, key, value = None):
    if t is None:
        t = Trie()

    p = t
    for c in key:
        if not c in p.children:
            p.children[c] = Trie()
        p = p.children[c]
    p.value = value
    return t
\end{lstlisting}

And we define the helper functions as the following.

\begin{lstlisting}
def trie_to_str(t, prefix=""):
    str="("+prefix
    if t.value is not None:
        str += ":"+t.value
    for k,v in sorted(t.children.items()):
        str += ", "+trie_to_str(v, prefix+k)
    str+=")"
    return str

def list_to_trie(l):
    return from_list(l, trie_insert)

def map_to_trie(m):
    return from_map(m, trie_insert)
\end{lstlisting}

With these helpers, we can test the insert program as below.

\begin{lstlisting}
class TrieTest:
    #...
    def test_insert(self):
        t = None
        t = trie_insert(t, "a")
        t = trie_insert(t, "an")
        t = trie_insert(t, "another")
        t = trie_insert(t, "b")
        t = trie_insert(t, "bob")
        t = trie_insert(t, "bool")
        t = trie_insert(t, "home")
        print trie_to_str(t)
\end{lstlisting}

It will print a trie in console.

\begin{verbatim}
(, (a, (an, (ano, (anot, (anoth, (anothe, (another))))))),
(b, (bo, (bob), (boo, (bool)))), (h, (ho, (hom, (home)))))
\end{verbatim}

\subsubsection {Recursive algorithm of Trie insertion}

The iterative algorithms can transform to recursive algorithm by such
approach. We take one character from
the key, and locate the child branch, then recursively insert the left
characters of the key to that branch. If the branch is empty, we
create a new node and add it to children before doing the recursively insertion.

\begin{algorithmic}[1]
\Function{TRIE-INSERT'}{$T, key, data$}
\If{$T = NIL$}
  \State $T \leftarrow EmptyNode$ \EndIf

\If{$key = NIL$}
  \State $VALUE(T) \leftarrow data$
\Else
  \State $p \leftarrow FIND(CHILDREN(T), FIRST(key))$
  \If{$p = NIL$}
    \State $p \leftarrow APPEND(CHILDREN(T), FIRST(key), EmptyNode)$
  \EndIf
  \State $TRIE-INSERT'(p, REST(key), data)$
\EndIf
\State \Return $T$
\EndFunction
\end{algorithmic}

\subsubsection*{Insertion of alphabetic trie in Haskell}
To realize the insertion in Haskell, The only thing we need do is
to translate the for-each loop into recursive call.

\lstset{language=Haskell}
\begin{lstlisting}
insert :: Trie a -> String -> a -> Trie a
insert t []     x = Trie (Just x)  (children t)
insert t (k:ks) x = Trie (value t) (ins (children t) k ks x) where
    ins [] k ks x = [(k, (insert empty ks x))]
    ins (p:ps) k ks x = if fst p == k
                        then (k, insert (snd p) ks x):ps
                        else p:(ins ps k ks x)
\end{lstlisting}

If the key is empty, the program reaches the trivial terminator case.
It just set the value. In other case, it examine the children recursively.
Each element of the children is a pair, contains a character and
a branch.

Some helper functions are provided as the following.

\begin{lstlisting}
fromList :: [(String, a)] -> Trie a
fromList xs = foldl ins empty xs where
    ins t (k, v) = insert t k v

toString :: (Show a)=> Trie a -> String
toString t = toStr t "" where
    toStr t prefix = "(" ++ prefix ++ showMaybe (value t) ++
                     (concat (map (\(k, v)-> ", " ++ toStr v (prefix++[k])))
                                 (sort (children t)))
                     ++ ")"
    showMaybe Nothing = ""
    showMaybe (Just x)  = ":" ++ show x

sort :: (Ord a)=>[(a, b)] -> [(a, b)]
sort [] = []
sort (p:ps) = sort xs ++ [p] ++ sort ys where
    xs = [x | x<-ps, fst x <= fst p ]
    ys = [y | y<-ps, fst y > fst p ]
\end{lstlisting}

The fromList function provide an easy way to repeatedly extract
key-value pairs from a list and insert them into a Trie.

Function toString can print the Trie in a modified pre-order way.
Because the children stored in a unsorted list, a sort function is
provided to sort the branches. The quick-sort algorithm is used.

We can test the above program with the below test cases.

\begin{lstlisting}
testTrie = "t=" ++ (toString t)
    where
      t = fromList [("a", 1), ("an", 2), ("another", 7), ("boy", 3),
                   ("bool", 4), ("zoo", 3)]

main = do
  putStrLn testTrie
\end{lstlisting}

The program outputs:

\begin{verbatim}
t=(, (a:1, (an:2, (ano, (anot, (anoth, (anothe, (another:7))))))),
(b, (bo, (boy:3), (boo, (bool:4)))), (z, (zo, (zoo:3))))
\end{verbatim}

It is identical to the ANSI C result except the values we inserted.

\subsubsection*{Insertion of alphabetic trie in Scheme/Lisp}

In order to manipulate string like list, we provide two helper
function to provide car, cdr like operations for string.

\lstset{language=lisp}
\begin{lstlisting}
(define (string-car s)
  (string-head s 1))

(define (string-cdr s)
  (string-tail s 1))
\end{lstlisting}

After that, we can implement insert program as the following.

\begin{lstlisting}
(define (insert t k x)
  (define (ins lst k ks x) ;; return list of child
    (if (null? lst)
	(list (make-child k (insert '() ks x)))
	(if (string=? (key (car lst)) k)
	    (cons (make-child k (insert (tree (car lst)) ks x)) (cdr lst))
	    (cons (car lst) (ins (cdr lst) k ks x)))))
  (if (string-null? k)
      (make-trie x (children t))
      (make-trie (value t)
		 (ins (children t) (string-car k) (string-cdr k) x))))
\end{lstlisting}

In order to print readable string for a Trie, we provide a pre-order
manner of Trie traverse function. It can convert a Trie to string.

\begin{lstlisting}
(define (trie->string t)
  (define (value->string x)
    (cond ((null? x) ".")
	  ((number? x) (number->string x))
	  ((string? x) x)
	  (else "unknown value")))
  (define (trie->str t prefix)
    (define (child->str c)
      (string-append ", " (trie->str (tree c) (string-append prefix (key c)))))
    (let ((lst (map child->str (sort-children (children t)))))
      (string-append "(" prefix (value->string (value t))
		     (fold-left string-append "" lst) ")")))
  (trie->str t ""))
\end{lstlisting}

Where sort-children is a quick sort algorithm to sort all children of
a node based on keys.

\begin{lstlisting}
(define (sort-children lst)
  (if (null? lst) '()
      (let ((xs (filter (lambda (c) (string<=? (key c) (key (car lst))))
			(cdr lst)))
	    (ys (filter (lambda (c) (string>?  (key c) (key (car lst))))
			(cdr lst))))
	(append (sort-children xs)
		(list (car lst))
		(sort-children ys)))))
\end{lstlisting}

Function filter is only available after $R^6RS$, for $R^5RS$, we define
the filter function manually.

\begin{lstlisting}
(define (filter pred lst)
  (keep-matching-items lst pred))
\end{lstlisting}

With all of these definition, we can test our insert program with some
simple test cases.

\begin{lstlisting}
(define (test-trie)
  (define t (list->trie (list '("a" 1) '("an" 2) '("another" 7)
                              '("boy" 3) '("bool" 4) '("zoo" 3))))
  (define t2 (list->trie (list '("zoo" 3) '("bool" 4) '("boy" 3)
                               '("another" 7) '("an" 2) '("a" 1))))
  (display (trie->string t)) (newline)
  (display (trie->string t2)) (newline))
\end{lstlisting}

In the above test program, function trie->string is reused, it is
previous defined for integer trie.

Evaluate test-trie function will output the following result.

\begin{lstlisting}
(test-trie)
(., (a1, (an2, (ano., (anot., (anoth., (anothe., (another7))))))),
    (b., (bo., (boo., (bool4)), (boy3))), (z., (zo., (zoo3))))
(., (a1, (an2, (ano., (anot., (anoth., (anothe., (another7))))))),
    (b., (bo., (boo., (bool4)), (boy3))), (z., (zo., (zoo3))))
\end{lstlisting}

% ================================================================
%                 Look up in Alphabetic trie
% ================================================================
\subsection{Look up in alphabetic trie}
To look up a key in a Trie, we also extract the character from the
key string one by one. For each character, we search among the children
branches to see if there is a branch represented by this character.
In case there is no such child, the look up process terminates
immediately to indicate a fail result. If we reach the last character,
The data stored in the current node is the result we are looking up.

\subsubsection{Iterative look up algorithm for alphabetic Trie}

This process can be described in pseudo code as below.

\begin{algorithmic}[1]
\Function{TRIE-LOOK-UP}{$T, key$}
  \If{$T = NIL$}
    \State \Return $NIL$
  \EndIf

  \State $p=T$
  \For{each $c$ in $key$}
    \If{$CHILDREN(p)[c] = NIL$}
      \State \Return $NIL$
    \EndIf
    \State $p \leftarrow CHILDREN(p)[c]$
  \EndFor
  \State \Return $DATA(p)$
\EndFunction
\end{algorithmic}

\subsubsection*{Look up in alphabetic Trie in C++}
We can easily translate the iterative algorithm to C++. If the key
specified can't be found in the Trie, our program returns a default
value of the data type. As alternative, it is also a choice to raise
exception.

\lstset{language=C++}
\begin{lstlisting}
template<class T, class Key>
typename T::ValueType lookup(T* t, Key key){
  if(!t)
    return typename T::ValueType(); //or throw exception

  T* p(t);
  for(typename Key::iterator it=key.begin(); it!=key.end(); ++it){
    if(p->children.find(*it) == p->children.end())
      return typename T::ValueType(); //or throw exception
    p = p->children[*it];
  }
  return p->value;
}
\end{lstlisting}

To verify the look up program, we can test it with some simple test
cases.

\begin{lstlisting}
Trie<char, int>* t(0);
const char* keys[] = {"a", "an", "another", "b", "bool", "bob", "home"};
const int vals[] = {1, 2, 7, 1, 4, 3, 4};
for(unsigned int i=0; i<sizeof(keys)/sizeof(char*); ++i)
  t = insert(t, std::string(keys[i]), vals[i]);
std::cout<<"\nlookup another: "<<lookup(t, std::string("another"))
         <<"\nlookup home: "<<lookup(t, std::string("home"))
	 <<"\nlookup the: "<<lookup(t, std::string("the"))<<"\n";
delete t;
\end{lstlisting}

We can get result as below.

\begin{verbatim}
lookup another: 7
lookup home: 4
lookup the: 0
\end{verbatim}

We see that key word ``the'' isn't contained in Trie, in our program,
the default value of integer, 0 is returned.

\subsubsection*{Look up in alphabetic trie in Python}
By translating the algorithm in Python language, we can get
a imperative program.

\lstset{language=Python}
\begin{lstlisting}
def lookup(t, key):
    if t is None:
        return None

    p = t
    for c in key:
        if not c in p.children:
            return None
        p = p.children[c]
    return p.value
\end{lstlisting}

We can use the similiar test cases to test looking up function.

\begin{lstlisting}
class TrieTest:
    #...
    def test_lookup(self):
        t = map_to_trie({"a":1, "an":2, "another":7, "b":1,
                         "bool":4, "bob":3, "home":4})
        print "find another: ", lookup(t, "another")
        print "find home: ", lookup(t, "home")
        print "find the: ", lookup(t, "the")
\end{lstlisting}

The result of these test cases are.

\begin{verbatim}
find another:  7
find home:  4
find the:  None
\end{verbatim}

\subsubsection{Recursive look up algorithm for alphabetic Trie}

In recursive algorithm, we take first character from the key to be
looked up. If it can be found in a child for the current node, we then
recursively search the rest characters of the key from that child
branch. else it means the key can't be found.

\begin{algorithmic}[1]
\Function{TRIE-LOOK-UP'}{$T, key$}
  \If{$key = NIL$}
    \State \Return $VALUE(T)$
  \EndIf
  \State $p \leftarrow FIND(CHILDREN(T), FIRST(key))$
  \If{$p = NIL$}
    \State \Return $NIL$
  \Else
    \State \Return $TRIE-LOOK-UP'(p, REST(key))$
  \EndIf
\EndFunction
\end{algorithmic}

\subsubsection*{Look up in alphabetic trie in Haskell}
To express this algorithm in Haskell, we can utilize 'lookup' function
in Haskell standard library\cite{wiki-trie}.

\lstset{language=Haskell}
\begin{lstlisting}
find :: Trie a -> String -> Maybe a
find t [] = value t
find t (k:ks) = case lookup k (children t) of
                  Nothing -> Nothing
                  Just t' -> find t' ks
\end{lstlisting}

We can append some search test cases right after insert.

\begin{lstlisting}
testTrie = "t=" ++ (toString t) ++
           "\nsearch t an: " ++ (show (find t "an")) ++
           "\nsearch t boy: " ++ (show (find t "boy")) ++
           "\nsearch t the: " ++ (show (find t "the"))
...
\end{lstlisting}

Here is the search result.

\begin{verbatim}
search t an: Just 2
search t boy: Just 3
search t the: Nothing
\end{verbatim}


\subsubsection*{Look up in alphabetic trie in Scheme/Lisp}

In Scheme/Lisp program, if the key is empty, we just return the value
of the current node, else we recursively find in children of the node
to see if there is a child binding to a character, which match the
first character of the key. We repeat this process till examine all
characters of the key.

\lstset{language=lisp}
\begin{lstlisting}
(define (lookup t k)
  (define (find k lst)
    (if (null? lst) '()
	(if (string=? k (key (car lst)))
	    (tree (car lst))
	    (find k (cdr lst)))))
  (if (string-null? k) (value t)
      (let ((child (find (string-car k) (children t))))
	(if (null? child) '()
	    (lookup child (string-cdr k))))))
\end{lstlisting}

we can test this look up with similar test cases as in Haskell program.

\begin{lstlisting}
(define (test-trie)
  (define t (list->trie (list '("a" 1) '("an" 2) '("another" 7)
                              '("boy" 3) '("bool" 4) '("zoo" 3))))
  (display (trie->string t)) (newline)
  (display "lookup an: ") (display (lookup t "an")) (newline)
  (display "lookup boy: ") (display (lookup t "boy")) (newline)
  (display "lookup the: ") (display (lookup t "the")) (newline))
\end{lstlisting}

This program will output the following result.

\begin{lstlisting}
(test-trie)
(., (a1, (an2, (ano., (anot., (anoth., (anothe., (another7))))))),
    (b., (bo., (boo., (bool4)), (boy3))), (z., (zo., (zoo3))))
lookup an: 2
lookup boy: 3
lookup the: ()
\end{lstlisting}

% ================================================================
%                 Alphabetic Patricia Tree
% ================================================================
\section{Alphabetic Patricia Tree}
Alphabetic Trie has the same problem as integer Trie. It is not memory
efficient. We can use the same method to compress alphabetic Trie into
Patricia.

% ================================================================
%                 Definition of Alphabetic Patricia Tree
% ================================================================
\subsection{Definition of alphabetic Patricia Tree}
Alphabetic patricia tree is a special tree, each node contains
multiple branches. All children of a node share the longest common
prefix string. There is no node has only one children, because it
is conflict with the longest common prefix property.

If we turn the Trie shown in figure \ref{fig:trie-of-26} into Patricia
tree by compressing those nodes which has only one child. we can get
a Patricia tree like in figure \ref{fig:patricia-tree}.

\begin{figure}[htbp]
  \begin{center}
    \includegraphics[scale=0.5]{img/patricia-tree.ps}
      \caption{A Patricia tree, with key a, an, another, bool,
    boy and zoo inserted.}
      \label{fig:patricia-tree}
  \end{center}
\end{figure}

Note that the root node always contains empty value.

\subsubsection*{Definition of alphabetic Patricia Tree in Haskell}
We can use a similar definition as Trie in Haskell, we need change
the type of the first element of children from single character to
string.

\lstset{language=Haskell}
\begin{lstlisting}
type Key = String

data Patricia a = Patricia { value :: Maybe a
                           , children :: [(Key, Patricia a)]}

empty = Patricia Nothing []

leaf :: a -> Patricia a
leaf x = Patricia (Just x) []
\end{lstlisting}

Besides the definition, helper functions to create a empty
Patricia node and to create a leaf node are provided.

\subsubsection*{Definition of alphabetic Patricia tree in Python}
The definition of Patricia tree is same as Trie in Python.

\lstset{language=Python}
\begin{lstlisting}
class Patricia:
    def __init__(self, value = None):
        self.value = value
        self.children = {}
\end{lstlisting}

\subsubsection*{Definition of alphabetic Patricia tree in C++}

With ISO C++, we abstract the key type of value type as type
parameters, and utilize STL provide map container to represent
children of a node.

\lstset{language=C++}
\begin{lstlisting}
template<class Key, class Value>
struct Patricia{
  typedef Patricia<Key, Value> Self;
  typedef std::map<Key, Self*> Children;
  typedef Key   KeyType;
  typedef Value ValueType;

  Patricia(Value v=Value()):value(v){}

  virtual ~Patricia(){
    for(typename Children::iterator it=children.begin();
        it!=children.end(); ++it)
      delete it->second;
  }

  Value value;
  Children children;
};
\end{lstlisting}

For illustration purpose, we simply release the memory in a recursive
way.

\subsubsection*{Definition of alphabetic Patricia tree in Scheme/Lisp}

We can fully reuse the definition of alphabetic Trie in Scheme/Lisp.
In order to provide a easy way to create a leaf node, we define an
extra helper function.

\lstset{language=lisp}
\begin{lstlisting}
(define (make-leaf x)
  (make-trie x '()))
\end{lstlisting}

% ================================================================
%                 Insertion of Alphabetic Patrica Tree
% ================================================================
\subsection{Insertion of alphabetic Patricia Tree}
When insert a key, $s$, into the Patricia tree, if the tree is empty, we
can just create an leaf node. Otherwise, we need check each child of the
Patricia tree. Every branch of the children is binding to a key, we denote
them as, $s_1, s_2, ..., s_n$, which means there are $n$ branches.
if $s$ and $s_i$ have common prefix, we then need branch out 2 new sub
branches. Branch itself is represent with the common prefix, each new
branches is represent with the different parts. Note there are 2 special
cases. One is that $s$ is the substring of $s_i$, the other is that $s_i$
is the substring of $s$. Figure \ref{fig:patricia-insert} shows these
different cases.

\begin{figure}[htbp]
       \begin{center}
	\includegraphics[scale=0.5]{img/patricia-insert-a.ps} (a)
	\includegraphics[scale=0.5]{img/patricia-insert-b.ps} (b)
	\includegraphics[scale=0.5]{img/patricia-insert-c.ps} (c)
	\includegraphics[scale=0.5]{img/patricia-insert-d.ps} (d)
        \caption{(a). Insert key, ``boy'' into an empty Patricia tree,
	the result is a leaf node; \newline
	(b). Insert key, ``bool'' into (a), result is a branch with
	common prefix ``bo''. \newline
        (c). Insert ``an'', with value $y$ into node $x$ with prefix
	``another''. \newline
        (d). Insert ``another'', into a node with prefix ``an'', the key to be inserted update to ``other'', and do further insertion.}
        \label{fig:patricia-insert}
       \end{center}
\end{figure}

\subsubsection{Iterative insertion algorithm for alphabetic Patricia}

The insertion algorithm can be described as below pseudo code.

\begin{algorithmic}[1]
\Function{PATRICIA-INSERT}{$T, key, value$}
\If{$T = NIL$}
   \State $T \leftarrow NewNode$ \EndIf

  \State $p=T$
  \Loop
    \State $match \leftarrow FALSE$
    \For{each $i$ in $CHILDREN(p)$}
      \If{$key = KEY(i)$}
        \State $VALUE(p) \leftarrow value$
        \State \Return $T$
      \EndIf

      \State $prefix \leftarrow LONGEST-COMMON-PREFIX(key, KEY(i))$
      \State $key1 \leftarrow key$ subtract $prefix$
      \State $key2 \leftarrow KEY(i)$ subtract $prefix$
      \If{$prefix \neq NIL$}
        \State $match \leftarrow TRUE$
        \If{$key2 = NIL$}
          \State $p \leftarrow TREE(i)$
          \State $key \leftarrow key$ substract $prefix$
          \State break
        \Else
          \State $CHILDREN(p)[prefix] \leftarrow BRANCH(key1, value, key2, TREE(i))$
          \State DELETE $CHILDREN(p)[KEY(i)]$
          \State \Return $T$
        \EndIf
      \EndIf

      \If{$match = FALSE$}
        \State $CHILDREN(p)[key] \leftarrow CREATE-LEAF(value)$
      \EndIf
    \EndFor

  \EndLoop
  \State \Return $T$
\EndFunction
\end{algorithmic}

In the above algorithm, LONGEST-COMMON-PREFIX function will find the longest
common prefix of two given string, for example, string ``bool'' and ``boy''
has longest common prefix ``bo''. BRANCH function will create a branch node
and update keys accordingly.

\subsubsection*{Insertion of alphabetic Patricia in C++}
in C++, to support implicit type conversion we utilize the KeyType and
ValueType as parameter types. If we define Patricia<std::string,
std::string>, we can directly provide char* parameters. the algorithm
is implemented as the following.

\lstset{language=C++}
\begin{lstlisting}
template<class K, class V>
Patricia<K, V>* insert(Patricia<K, V>* t,
                       typename Patricia<K, V>::KeyType key,
                       typename Patricia<K, V>::ValueType value=V()){
  if(!t)
    t = new Patricia<K, V>();

  Patricia<K, V>* p = t;
  typedef typename Patricia<K, V>::Children::iterator Iterator;
  for(;;){
    bool match(false);
    for(Iterator it = p->children.begin(); it!=p->children.end(); ++it){
      K k=it->first;
      if(key == k){
        p->value = value; //overwrite
        return t;
      }
      K prefix = lcp(key, k);
      if(!prefix.empty()){
        match=true;
        if(k.empty()){ //e.g. insert "another" into "an"
          p = it->second;
          break;
        }
        else{
          p->children[prefix] = branch(key, new Patricia<K, V>(value),
                                       k, it->second);
          p->children.erase(it);
          return t;
        }
      }
    }
    if(!match){
      p->children[key] = new Patricia<K, V>(value);
      break;
    }
  }
  return t;
}
\end{lstlisting}

Where the lcp and branch functions are defined like this.

\begin{lstlisting}
template<class K>
K lcp(K& s1, K& s2){
  typename K::iterator it1(s1.begin()), it2(s2.begin());
  for(; it1!=s1.end() && it2!=s2.end() && *it1 == *it2; ++it1, ++it2);
  K res(s1.begin(), it1);
  s1 = K(it1, s1.end());
  s2 = K(it2, s2.end());
  return res;
}

template<class T>
T* branch(typename T::KeyType k1, T* t1,
          typename T::KeyType k2, T* t2){
  if(k1.empty()){ //e.g. insert "an" into "another"
    t1->children[k2] = t2;
    return t1;
  }
  T* t = new T();
  t->children[k1] = t1;
  t->children[k2] = t2;
  return t;
}
\end{lstlisting}

Function lcp() will extract the longest common prefix and modify its
parameters. Function branch() will create a new node and set the 2
nodes to be merged as the children. There is a special case, if the
key of one node is sub-string of the other, it will chain them
together.

We find the implementation of patricia\_to\_str() will be very same
as trie\_to\_str(), so we can reuse it. Also the convert from a list
of keys to trie can be reused

\begin{lstlisting}
// list_to_trie
template<class Iterator, class T>
T* list_to_trie(Iterator first, Iterator last, T* t){
  typedef typename T::ValueType ValueType;
  return std::accumulate(first, last, t,
                         std::ptr_fun(insert_key<T, ValueType>));
}
\end{lstlisting}

We put all of the helper function templates to a utility header file,
and we can test patricia insertion program as below.

\begin{lstlisting}
template<class Iterator>
void test_list_to_patricia(Iterator first, Iterator last){
  typedef Patricia<std::string, std::string> PatriciaType;
  PatriciaType* t(0);
  t = list_to_trie(first, last, t);
  std::copy(first, last,
            std::ostream_iterator<std::string>(std::cout, ", "));
  std::cout<<"\n==>"<<trie_to_str(t)<<"\n";
  delete t;
}

void test_insert(){
  const char* lst1[] = {"a", "an", "another", "b", "bob", "bool", "home"};
  test_list_to_patricia(lst1, lst1+sizeof(lst1)/sizeof(char*));

  const char* lst2[] = {"home", "bool", "bob", "b", "another", "an", "a"};
  test_list_to_patricia(lst2, lst2+sizeof(lst2)/sizeof(char*));

  const char* lst3[] = {"romane", "romanus", "romulus"};
  test_list_to_patricia(lst3, lst3+sizeof(lst3)/sizeof(char*));

  typedef Patricia<std::string, std::string> PatriciaType;
  PatriciaType* t(0);
  const char* keys[] = {"001", "100", "101"};
  const char* vals[] = {"y", "x", "z"};
  for(unsigned int i=0; i<sizeof(keys)/sizeof(char*); ++i)
    t = insert(t, std::string(keys[i]), std::string(vals[i]));
  std::copy(keys, keys+sizeof(keys)/sizeof(char*),
	    std::ostream_iterator<std::string>(std::cout, ", "));
  std::cout<<"==>"<<trie_to_str(t)<<"\n";
  delete t;
}
\end{lstlisting}

Running test\_insert() function will generate the following output.

\begin{verbatim}
a, an, another, b, bob, bool, home,
==>(, (a, (an, (another))), (b, (bo, (bob), (bool))), (home))
home, bool, bob, b, another, an, a,
==>(, (a, (an, (another))), (b, (bo, (bob), (bool))), (home))
romane, romanus, romulus,
==>(, (rom, (roman, (romane), (romanus)), (romulus)))
001, 100, 101, ==>(, (001:y), (10, (100:x), (101:z)))
\end{verbatim}

\subsubsection*{Insertion of alphabetic Patrica Tree in Python}
By translate the insertion algorithm into Python language, we can get a
program as below.

\lstset{language=Python}
\begin{lstlisting}
def insert(t, key, value = None):
    if t is None:
        t = Patricia()

    node = t
    while(True):
        match = False
        for k, tr in node.children.items():
            if key == k: # just overwrite
                node.value = value
                return t
            (prefix, k1, k2)=lcp(key, k)
            if prefix != "":
                match = True
                if k2 == "":
                    # example: insert "another" into "an", go on traversing
                    node = tr
                    key = k1
                    break
                else: #branch out a new leaf
                    node.children[prefix] = branch(k1, Patricia(value), k2, tr)
                    del node.children[k]
                    return t
        if not match: # add a new leaf
            node.children[key] = Patricia(value)
            break
    return t
\end{lstlisting}

Where the longest common prefix finding and branching functions are implemented
as the following.

\begin{lstlisting}
# longest common prefix
# returns (p, s1', s2'), where p is lcp, s1'=s1-p, s2'=s2-p
def lcp(s1, s2):
    j=0
    while j<=len(s1) and j<=len(s2) and s1[0:j]==s2[0:j]:
        j+=1
    j-=1
    return (s1[0:j], s1[j:], s2[j:])

def branch(key1, tree1, key2, tree2):
    if key1 == "":
        #example: insert "an" into "another"
        tree1.children[key2] = tree2
        return tree1
    t = Patricia()
    t.children[key1] = tree1
    t.children[key2] = tree2
    return t
\end{lstlisting}

Function lcp check every characters of two strings are same one by one
till it met a different one or either of the string finished.

In order to test the insertion program, some helper functions are
provided.

\begin{lstlisting}
def to_string(t):
    return trie_to_str(t)

def list_to_patricia(l):
    return from_list(l, insert)

def map_to_patricia(m):
    return from_map(m, insert)
\end{lstlisting}

We can reuse the trie\_to\_str since their implementation are same.
to\_string function can turn a Patricia tree into string by traversing it
in pre-order. list\_to\_patricia helps to convert a list of object into
a Patricia tree by repeatedly insert every elements into the tree. While
map\_to\_string does similar thing except it can convert a list of key-value
pairs into a Patricia tree.

Then we can test the insertion program with below test cases.

\begin{lstlisting}
class PatriciaTest:
    #...
    def test_insert(self):
        print "test insert"
        t = list_to_patricia(["a", "an", "another", "b", "bob", "bool", "home"])
        print to_string(t)
        t = list_to_patricia(["romane", "romanus", "romulus"])
        print to_string(t)
        t = map_to_patricia({"001":'y', "100":'x', "101":'z'})
        print to_string(t)
        t = list_to_patricia(["home", "bool", "bob", "b", "another", "an", "a"]);
        print to_string(t)
\end{lstlisting}

These test cases will output a series of result like this.

\begin{verbatim}
(, (a, (an, (another))), (b, (bo, (bob), (bool))), (home))
(, (rom, (roman, (romane), (romanus)), (romulus)))
(, (001:y), (10, (100:x), (101:z)))
(, (a, (an, (another))), (b, (bo, (bob), (bool))), (home))
\end{verbatim}

\subsubsection{Recursive insertion algorithm for alphabetic Patricia}

The insertion can also be implemented recursively. When doing
insertion, the program check all the children of the Patricia node, to
see if there is a node can match the key. Match means they have common
prefix. One special case is that the keys are same, the program just
overwrite the value of that child. If there is no child can match the
key, the program create a new leaf, and add it as a new child.

\begin{algorithmic}[1]
\Function{PATRICIA-INSERT'}{$T, key, value$}
\If{$T = NIL$}
   \State $T \leftarrow EmptyNode$ \EndIf

\State $p \leftarrow FIND-MATCH(CHILDREN(T), key)$
\If{$p = NIL$}
  \State $ADD(CHILDREN(T), CREATE-LEAF(key, value))$
\ElsIf{$KEY(p) = key$}
  \State $VALUE(p) \leftarrow value$
\Else
  \State $q \leftarrow BRANCH(CREATE-LEAF(key, value), p)$
  \State $ADD(CHILDREN(T), q)$
  \State $DELETE(CHILDREN(T), p)$
\EndIf
\State \Return $T$
\EndFunction
\end{algorithmic}

The recursion happens inside call to BRANCH. The longest common prefix
of 2 nodes are extracted. If the key to be inserted is the sub-string of
the node, we just chain them together; If the prefix of the node is
the sub-string of the key, we recursively insert the rest of the key
to the node. In other case, we create a new node with the common
prefix and set its two children.

\begin{algorithmic}[1]
\Function{BRANCH}{$T1, T2$}
  \State $prefix \leftarrow LONGEST-COMMON-PREFIX(T1, T2)$

  \State $p \leftarrow EmptyNode$
  \If{$prefix = KEY(T1)$}
    \State $KEY(T2) \leftarrow KEY(T2)$ subtract $prefix$
    \State $p \leftarrow CREATE-LEAF(prefix, VALUE(T1))$
    \State $ADD(CHILDREN(p), T2)$
  \ElsIf{$prefix = KEY(T2)$}
    \State $KEY(T1) \leftarrow KEY(T1)$ subtract $prefix$
    \State $p \leftarrow PATRICIA-INSERT'(T2, KEY(T1), VALUE(T1))$
    \State $KEY(p) \leftarrow prefix$
  \Else
    \State $KEY(T2) \leftarrow KEY(T2)$ subtract $prefix$
    \State $KEY(T1) \leftarrow KEY(T1)$ subtract $prefix$
    \State $ADD(CHILDREN(p), T1, T2)$
    \State $KEY(p) \leftarrow prefix$
  \EndIf
  \State \Return $p$
\EndFunction
\end{algorithmic}

\subsubsection*{Insertion of alphabetic Patrica Tree in Haskell}
By implementing the above algorithm in Recursive way, we can get a
Haskell program of Patricia insertion.

\lstset{language=Haskell}
\begin{lstlisting}
insert :: Patricia a -> Key -> a -> Patricia a
insert t k x = Patricia (value t) (ins (children t) k x) where
    ins []     k x = [(k, Patricia (Just x) [])]
    ins (p:ps) k x
        | (fst p) == k
            = (k, Patricia (Just x) (children (snd p))):ps --overwrite
        | match (fst p) k
            = (branch k x (fst p) (snd p)):ps
        | otherwise
            = p:(ins ps k x)

\end{lstlisting}

Function insert takes a Patricia tree, a key and a value. It will call
an internal function ins to insert the data into the children of the tree.
If the tree has no children, it simply create a leaf node and put it
as the single child of the tree. In other case it will check each child
to see if any one has common prefix with the key. There is a special case,
the child has very same key, we can overwrite the data. If the child has common
prefix is located, we branch out a new node.

A function match is provided to determine if two keys have common prefix as
below.

\begin{lstlisting}
match :: Key -> Key -> Bool
match [] _ = False
match _ [] = False
match x y = head x == head y
\end{lstlisting}

This function is straightforward, only if the first character of the two keys
are identical, we say they have common prefix.

Branch out function and the longest common prefix function are implemented like
the following.

\begin{lstlisting}
branch :: Key -> a -> Key -> Patricia a -> (Key, Patricia a)
branch k1 x k2 t2
    | k1 == k
        -- ex: insert "an" into "another"
        = (k, Patricia (Just x) [(k2', t2)])
    | k2 == k
        -- ex: insert "another" into "an"
        = (k, insert t2 k1' x)
    | otherwise = (k, Patricia Nothing [(k1', leaf x), (k2', t2)])
   where
      k = lcp k1 k2
      k1' = drop (length k) k1
      k2' = drop (length k) k2

lcp :: Key -> Key -> Key
lcp [] _ = []
lcp _ [] = []
lcp (x:xs) (y:ys) = if x==y then x:(lcp xs ys) else []
\end{lstlisting}

Function take a key, k1, a value, another key, k2, and a Patricia tree t2. It will first
call lcp to get the longest common prefix, k, and the different part of the original key.
If k is just k1, which means k1 is a sub-string of k2, we create a new Patricia node,
put the new value in it, then set the left part of k2 and t2 as the single child of this
new node. If k is just k2, which means k2 is a sub-string of k1, we recursively insert
the update key and value to t2. Otherwise, we create a new node, along with the the longest
common prefix. the new node has 2 children, one is t2, the other is a leaf node of the data
to be inserted. Each of them are binding to updated keys.

In order to test the above program, we provided some helper functions.

\begin{lstlisting}
fromList :: [(Key, a)] -> Patricia a
fromList xs = foldl ins empty xs where
    ins t (k, v) = insert t k v

sort :: (Ord a)=>[(a, b)] -> [(a, b)]
sort [] = []
sort (p:ps) = sort xs ++ [p] ++ sort ys where
    xs = [x | x<-ps, fst x <= fst p ]
    ys = [y | y<-ps, fst y > fst p ]

toString :: (Show a)=>Patricia a -> String
toString t = toStr t "" where
    toStr t prefix = "(" ++ prefix ++ showMaybe (value t) ++
                     (concat $ map (\(k, v)->", " ++ toStr v (prefix++k))
                             (sort $children t))
                     ++ ")"
    showMaybe Nothing = ""
    showMaybe (Just x) = ":" ++ show x
\end{lstlisting}

Function fromList can recursively insert each key-value pair into a Patricia node.
Function sort helps to sort a list of key-value pairs based on keys by using quick sort
algorithm. toString turns a Patricia tree into a string by using modified pre-order.

After that, we can test our insert program with the following test cases.

\begin{lstlisting}
testPatricia = "t1=" ++ (toString t1) ++ "\n" ++
               "t2=" ++ (toString t2)
    where
      t1 = fromList [("a", 1), ("an", 2), ("another", 7),
                     ("boy", 3), ("bool", 4), ("zoo", 3)]
      t2 = fromList [("zoo", 3), ("bool", 4), ("boy", 3),
                     ("another", 7), ("an", 2), ("a", 1)]

main = do
  putStrLn testPatricia
\end{lstlisting}

No matter what's the insert order, the 2 test cases output an identical result.

\begin{verbatim}
t1=(, (a:1, (an:2, (another:7))), (bo, (bool:4), (boy:3)), (zoo:3))
t2=(, (a:1, (an:2, (another:7))), (bo, (bool:4), (boy:3)), (zoo:3))
\end{verbatim}


\subsubsection*{Insertion of alphabetic Patrica Tree in Scheme/Lisp}

In Scheme/Lisp, If the root doesn't has child, we create a leaf node
with the value, and bind the key with this node; if the key binding to
one child is equal to the string we want to insert, we just overwrite
the current value; If the key has common prefix of the string to be
inserted, we branch out a new node.

\lstset{language=lisp}
\begin{lstlisting}
(define (insert t k x)
  (define (ins lst k x) ;; lst: [(key patrica)]
    (if (null? lst) (list (make-child k (make-leaf x)))
	(cond ((string=? (key (car lst)) k)
	       (cons (make-child k (make-trie x (children (tree (car lst)))))
		     (cdr lst)))
	      ((match? (key (car lst)) k)
	       (cons (branch k x (key (car lst)) (tree (car lst)))
		     (cdr lst)))
	      (else (cons (car lst) (ins (cdr lst) k x))))))
  (make-trie (value t) (ins (children t) k x)))
\end{lstlisting}

The match? function just test if two strings have common prefix.

\begin{lstlisting}
(define (match? x y)
  (and (not (or (string-null? x) (string-null? y)))
       (string=? (string-car x) (string-car y))))
\end{lstlisting}

Function branch takes 4 parameters, the first key, the value to be
inserted, the second key, and the Patricia tree need to be branch
out. If will first find the longest common prefix of the two keys, If
it is equal to the first key, it means that the first key is the prefix of
the second key, we just create a new node with the value and chained
the Patricia tree by setting it as the only one child of this new
node; If the longest common prefix is equal to the second key, it
means that the second key is the prefix of the first key, we just
recursively insert the different part (the remove the prefix part)
into this Patricia tree; In other case, we just create a branch node
and set its two children, one is the leaf node with the value to be
inserted, the other is the Patricia tree passed as the fourth parameter.

\begin{lstlisting}
(define (branch k1 x k2 t2) ;; returns (key tree)
  (let* ((k (lcp k1 k2))
	 (k1-new (string-tail k1 (string-length k)))
	 (k2-new (string-tail k2 (string-length k))))
    (cond ((string=? k1 k) ;; e.g. insert "an" into "another"
	   (make-child k (make-trie x (list (make-child k2-new t2)))))
	  ((string=? k2 k) ;; e.g. insert "another" into "an"
	   (make-child k (insert t2 k1-new x)))
	  (else (make-child k (make-trie
			       '()
			       (list (make-child k1-new (make-leaf x))
				     (make-child k2-new t2))))))))
\end{lstlisting}

Where the longest common prefix is extracted as the following.

\begin{lstlisting}
(define (lcp x y)
  (let ((len (string-match-forward x y)))
    (string-head x len)))
\end{lstlisting}

We can reuse the list->trie and trie->string functions to test our
program.

\begin{lstlisting}
(define (test-patricia)
  (define t (list->trie (list '("a" 1) '("an" 2) '("another" 7)
                        '("boy" 3) '("bool" 4) '("zoo" 3))))
  (define t2 (list->trie (list '("zoo" 3) '("bool" 4) '("boy" 3)
                         '("another" 7) '("an" 2) '("a" 1))))
  (display (trie->string t)) (newline)
  (display (trie->string t2)) (newline))
\end{lstlisting}

Evaluate this function will print t and t2 as below.

\begin{lstlisting}
(test-patricia)
(., (a1, (an2, (another7))), (bo., (bool4), (boy3)), (zoo3))
(., (a1, (an2, (another7))), (bo., (bool4), (boy3)), (zoo3))
\end{lstlisting}

% ================================================================
%                 Look up in Alphabetic Patrica Tree
% ================================================================
\subsection{Look up in alphabetic Patricia tree}
Different with Trie, we can't take each character from the key to look up.
We need check each child to see if it is a prefix of the key to be found.
If there is such a child, we then remove the prefix from
the key, and search this updated key in that child. if we can't find any
children as a prefix of the key, it means the looking up failed.

\subsubsection{Iterative look up algorithm for alphabetic Patricia tree}

This algorithm can be described in pseudo code as below.
\begin{algorithmic}[1]
\Function{PATRICIA-LOOK-UP}{$T, key$}
  \If{$T = NIL$}
     \State \Return $NIL$ \EndIf

  \Repeat
    \State $match \leftarrow FALSE$
    \For{each $i$ in $CHILDREN(T)$}
      \If{$key = KEY(i)$}
        \State \Return $VALUE(TREE(i))$
      \EndIf
      \If{$KEY(i)$ IS-PREFIX-OF $key$}
        \State $match \leftarrow TRUE$
        \State $key \leftarrow key$ subtract $KEY(i)$
        \State $T \leftarrow TREE(i)$
        \State break
      \EndIf
    \EndFor
  \Until{$match = FALSE$}
  \State \Return $NIL$
\EndFunction
\end{algorithmic}

\subsubsection*{Look up in alphabetic Patrica Tree in C++}
In C++, we abstract the key type as a template parameter. By refer to
the KeyType defined in Patricia, we can get the support of implicitly
type conversion. If we can't find the key in a Patricia tree, the
below program returns default value of data type. One alternative is
to throw exception.

\lstset{language=C++}
\begin{lstlisting}
template<class K, class V>
V lookup(Patricia<K, V>* t, typename Patricia<K, V>::KeyType key){
  typedef typename Patricia<K, V>::Children::iterator Iterator;
  if(!t)
    return V(); //or throw exception
  for(;;){
    bool match(false);
    for(Iterator it=t->children.begin(); it!=t->children.end(); ++it){
      K k = it->first;
      if(key == k)
        return it->second->value;
      K prefix = lcp(key, k);
      if((!prefix.empty()) && k.empty()){
        match = true;
        t = it->second;
        break;
      }
    }
    if(!match)
      return V(); //or throw exception
  }
}
\end{lstlisting}

To verify the look up program, we test it with the following simple
test cases.

\begin{lstlisting}
Patricia<std::string, int>* t(0);
const char* keys[] = {"a", "an", "another", "boy", "bool", "home"};
const int vals[] = {1, 2, 7, 3, 4, 4};
for(unsigned int i=0; i<sizeof(keys)/sizeof(char*); ++i)
  t = insert(t, keys[i], vals[i]);
std::cout<<"\nlookup another: "<<lookup(t, "another")
         <<"\nlookup boo: "<<lookup(t, "boo")
         <<"\nlookup boy: "<<lookup(t, "boy")
         <<"\nlookup by: "<<lookup(t, "by")
         <<"\nlookup boolean: "<<lookup(t, "boolean")<<"\n";
delete t;
\end{lstlisting}

This program will output the the result like below.

\begin{verbatim}
lookup another: 7
lookup boo: 0
lookup boy: 3
lookup by: 0
lookup boolean: 0
\end{verbatim}

\subsubsection*{Look up in alphabetic Patricia Tree in Python}
The implementation of looking up in Python is similar to the pseudo code.
Because Python don't support repeat-until loop directly, a while loop is
used instead.

\lstset{language=Python}
\begin{lstlisting}
def lookup(t, key):
    if t is None:
        return None
    while(True):
        match = False
        for k, tr in t.children.items():
            if k == key:
                return tr.value
            (prefix, k1, k2) = lcp(key, k)
            if prefix != "" and k2 == "":
                match = True
                key = k1
                t = tr
                break
        if not match:
            return None
\end{lstlisting}

We can verify the looking up program as below.

\begin{lstlisting}
class PatriciaTest:
    # ...
    def test_lookup(self):
        t = map_to_patricia({"a":1, "an":2, "another":7, "b":1, "bob":3, \
                            "bool":4, "home":4})
        print "search t another", lookup(t, "another")
        print "search t boo", lookup(t, "boo")
        print "search t bob", lookup(t, "bob")
        print "search t boolean", lookup(t, "boolean")
\end{lstlisting}

The test result output in console is like the following.

\begin{verbatim}
search t another 7
search t boo None
search t bob 3
search t boolean None
\end{verbatim}

\subsubsection{Recursive look up algorithm for alphabetic Patricia
tree}

To implement the look up recursively, we just look up among the children
of the Patricia tree.

\begin{algorithmic}[1]
\Function{PATRICIA-LOOK-UP'}{$T, key$}
  \If{$T = NIL$}
    \State \Return $NIL$
  \Else
    \State \Return $FIND-IN-CHILDREN(CHILDREN(T), key)$
  \EndIf
\EndFunction
\end{algorithmic}

The real recursion happens in FIND-IN-CHILDREN call, we pass the
children list as an argument. If it is not empty, we take first child,
and check if the prefix of this child is equal to the key; the value
of this child will be returned if they are same; if the prefix of the
child is just a prefix of the key, we recursively find in this child
with updated key.

\begin{algorithmic}[1]
\Function{FIND-IN-CHILDREN}{$l, key$}
  \If{$l = NIL$}
    \State \Return $NIL$
  \ElsIf{$KEY(FIRST(l)) = key$}
    \State \Return $VALUE(FIRST(l))$
  \ElsIf{$KEY(FIRST(l))$ is prefix of $key$}
    \State $key \leftarrow key$ subtract $KEY(FIRST(l))$
    \State \Return $PATRICIA-LOOK-UP'(FIRST(l), key)$
  \Else
    \State \Return $FIND-IN-CHILDREN(REST(l), key)$
  \EndIf
\EndFunction
\end{algorithmic}

\subsubsection*{Look up in alphabetic Patrica Tree in Haskell}
In Haskell implementation, the above algorithm should be turned
into recursive way.

\lstset{language=Haskell}
\begin{lstlisting}
-- lookup
import qualified Data.List

find :: Patricia a -> Key -> Maybe a
find t k = find' (children t) k where
    find' [] _ = Nothing
    find' (p:ps) k
          | (fst p) == k = value (snd p)
          | (fst p) `Data.List.isPrefixOf` k = find (snd p) (diff (fst p) k)
          | otherwise = find' ps k
    diff k1 k2 = drop (length (lcp k1 k2)) k2
\end{lstlisting}

When we search a given key in a Patricia tree, we recursively check each
of the child. If there are no children at all, we stop the recursion and
indicate a look up failure. In other case, we pick the prefix-node pair
one by one. If the prefix is as same as the given key, it means the target
node is found and the value of the node is returned. If the key has common
prefix with the child, the key will be updated by removing the longest
common prefix and we performs looking up recursively.

We can verify the above Haskell program with the following simple cases.

\begin{lstlisting}
testPatricia = "t1=" ++ (toString t1) ++ "\n" ++
               "find t1 another =" ++ (show (find t1 "another")) ++ "\n" ++
               "find t1 bo = " ++ (show (find t1 "bo")) ++ "\n" ++
               "find t1 boy = " ++ (show (find t1 "boy")) ++ "\n" ++
               "find t1 boolean = " ++ (show (find t1 "boolean"))
    where
      t1 = fromList [("a", 1), ("an", 2), ("another", 7), ("boy", 3),
           ("bool", 4), ("zoo", 3)]

main = do
  putStrLn testPatricia
\end{lstlisting}

The output is as below.

\begin{verbatim}
t1=(, (a:1, (an:2, (another:7))), (bo, (bool:4), (boy:3)), (zoo:3))
find t1 another =Just 7
find t1 bo = Nothing
find t1 boy = Just 3
find t1 boolean = Nothing
\end{verbatim}


\subsubsection*{Look up in alphabetic Patricia Tree in Scheme/Lisp}

The Scheme/Lisp program is given as the following. The function
delegate the looking up to an inner function which will check each
child to see if the key binding to the child match the string we are
looking for.

\lstset{language=lisp}
\begin{lstlisting}
(define (lookup t k)
  (define (find lst k) ;; lst, [(k patricia)]
    (if (null? lst) '()
	(cond ((string=? (key (car lst)) k) (value (tree (car lst))))
	      ((string-prefix? (key (car lst)) k)
	       (lookup (tree (car lst))
		     (string-tail k (string-length (key (car lst))))))
	      (else (find (cdr lst) k)))))
  (find (children t) k))
\end{lstlisting}

In order to verify this program, some simple test cases are given to
search in the Patricia we created in previous section.

\begin{lstlisting}
(define (test-patricia)
  (define t (list->trie (list '("a" 1) '("an" 2) '("another" 7)
                        '("boy" 3) '("bool" 4) '("zoo" 3))))
  (display (trie->string t)) (newline)
  (display "lookup another: ") (display (lookup t "another")) (newline)
  (display "lookup bo: ") (display (lookup t "bo")) (newline)
  (display "lookup boy: ") (display (lookup t "boy")) (newline)
  (display "lookup by: ") (display (lookup t "by")) (newline)
  (display "lookup boolean: ") (display (lookup t "boolean")) (newline))
\end{lstlisting}

This program will output the same result as the Haskell one.

\begin{lstlisting}
(test-patricia)
(., (a1, (an2, (another7))), (bo., (bool4), (boy3)), (zoo3))
lookup another: 7
lookup bo: ()
lookup boy: 3
lookup by: ()
lookup boolean: ()
\end{lstlisting}

% ================================================================
%                 Trie and Patrica used in Industry
% ================================================================
\section{Trie and Patricia used in Industry}
Trie and Patricia are widely used in software industry. Integer based
Patricia tree is widely used in compiler. Some daily
used software has very interesting features can be realized with
Trie and Patricia. In the following sections, I'll list some of them,
including, e-dictionary, word auto-completion, t9 input method etc.
The commercial implementation typically doesn't adopt Trie or Patricia
directly. However, Trie and Patricia can be shown as a kind of
example realization.

\subsection{e-dictionary and word auto-completion}
Figure \ref{fig:e-dict} shows a screen shot of an English-Chinese dictionary.
In order to provide good user experience, when user input something,
the dictionary will search its word library, and list all candidate words and
phrases similar to what user have entered.

\begin{figure}[htbp]
       \begin{center}
	\includegraphics[scale=0.5]{img/ciba.eps}
        \caption{e-dictionary. All candidates starting with what user input are listed.}
        \label{fig:e-dict}
       \end{center}
\end{figure}

Typically such dictionary contains hundreds of thousands words, performs a whole
word search is expensive. Commercial software adopts complex approach, including
caching, indexing etc to speed up this process.

Similar with e-dictionary, figure \ref{fig:word-completion} shows a popular
Internet search engine, when user input something, it will provide a candidate
lists, with all items start with what user has entered. And these candidates
are shown in an order of popularity. The more people search for a word, the
upper position it will be shown in the list.

\begin{figure}[htbp]
       \begin{center}
	\includegraphics[scale=0.5]{img/adaptive-input.eps}
        \caption{Search engine. All candidates key words starting with what user input are listed.}
        \label{fig:word-completion}
       \end{center}
\end{figure}

In both case, we say the software provide a kind of word auto-completion support.
In some modern IDE, the editor can even helps user to auto-complete programmings.

In this section, I'll show a very simple implementation of e-dictionary with Trie
and Patricia.
To simplify the problem, let us assume the dictionary only support English - English
information.

Typically, a dictionary contains a lot of key-value pairs, the keys are English
words or phrases, and the relative values are the meaning of the words.

We can store all words and their meanings to a Trie, the drawback for this
approach is that it isn't space effective. We'll use Patricia as alternative
later on.

As an example, when user want to look up 'a', the dictionary does not only
return the meaning of then English word 'a', but also provide a list of
candidate words, which are all start with 'a', including 'abandon', 'about',
'accent', 'adam', ... Of course all these words are stored in Trie.

If there are too many candidates, one solution is only display the top 10
words for the user, and if he like, he can browse more.

Below pseudo code reuse the looking up program in previous sections and
Expand all potential top N candidates.

\begin{algorithmic}[1]
\Function{TRIE-LOOK-UP-TOP-N}{$T, key, N$}
  \State $p \leftarrow TRIE-LOOK-UP'(T, key)$
  \State \Return $EXPAND-TOP-N(p, key, N)$
\EndFunction
\end{algorithmic}

Note that we should modify the TRIE-LOOK-UP a bit, instead of return
the value of the node, TRIE-LOOK-UP' returns the node itself.

Another alternative is to use Patricia instead of Trie. It can save much
spaces.

\subsubsection{Iterative algorithm of search top N candidate in Patricia}

The algorithm is similar to the Patricia look up one, but when we found
a node which key start from the string we are looking for, we expand
all its children until we get N candidates.

\begin{algorithmic}[1]
\Function{PATRICIA-LOOK-UP-TOP-N}{$T, key, N$}
  \If{$T = NIL$}
     \State \Return $NIL$ \EndIf

  \State $prefix \leftarrow NIL$
  \Repeat
    \State $match \leftarrow FALSE$
    \For{each $i$ in $CHILDREN(T)$}
      \If{$key$ is prefix of $KEY(i)$}
        \State \Return $EXPAND-TOP-N(TREE(i), prefix, N)$
      \EndIf
      \If{$KEY(i)$ is prefix of $key$}
        \State $match \leftarrow TRUE$
        \State $key \leftarrow key$ subtract $KEY(i)$
        \State $T \leftarrow TREE(i)$
        \State $prefix \leftarrow prefix + KEY(i)$
        \State break
      \EndIf
    \EndFor
  \Until{$match = FALSE$}
  \State \Return $NIL$
\EndFunction
\end{algorithmic}

\subsubsection*{An e-dictionary in Python}
In Python implementation, a function trie\_lookup is provided to perform
search all top N candidate started with a given string.

\lstset{language=Python}
\begin{lstlisting}
def trie_lookup(t, key, n):
    if t is None:
        return None

    p = t
    for c in key:
        if not c in p.children:
            return None
        p=p.children[c]
    return expand(key, p, n)

def expand(prefix, t, n):
    res = []
    q = [(prefix, t)]
    while len(res)<n and len(q)>0:
        (s, p) = q.pop(0)
        if p.value is not None:
            res.append((s, p.value))
        for k, tr in p.children.items():
            q.append((s+k, tr))
    return res
\end{lstlisting}

Compare with the Trie look up function, the first part of this program is almost
same. The difference part is after we successfully located the node
which matches the key, all sub trees are expanded from this node in a
bread-first search manner, and the top n candidates are returned.

This program can be verified by below simple test cases.

\begin{lstlisting}
class LookupTest:
    def __init__(self):
        dict = {"a":"the first letter of English", \
           "an":"...same dict as in Haskell example"}
        self.tt = trie.map_to_trie(dict)

    def run(self):
        self.test_trie_lookup()

    def test_trie_lookup(self):
        print "test lookup top 5"
        print "search a ", trie_lookup(self.tt, "a", 5)
        print "search ab ", trie_lookup(self.tt, "ab", 5)
\end{lstlisting}

The test will output the following result.

\begin{verbatim}
test lookup to 5
search a  [('a', 'the first letter of English'), ('an', "used instead of 'a'
when the following word begins with a vowel sound"), ('adam', 'a character in
the Bible who was the first man made by God'), ('about', 'on the subject of;
connected with'), ('abandon', 'to leave a place, thing or person forever')]
search ab  [('about', 'on the subject of; connected with'), ('abandon', 'to
leave a place, thing or person forever')]
\end{verbatim}

To save the spaces, we can also implement such a dictionary search by using
Patricia.

\begin{lstlisting}
def patricia_lookup(t, key, n):
    if t is None:
        return None
    prefix = ""
    while(True):
        match = False
        for k, tr in t.children.items():
            if string.find(k, key) == 0: #is prefix of
                return expand(prefix+k, tr, n)
            if string.find(key, k) ==0:
                match = True
                key = key[len(k):]
                t = tr
                prefix += k
                break
        if not match:
            return None
\end{lstlisting}

In this program, we called Python string class to test if a string x is
prefix of a string y. In case we locate a node with the key we are looking
up is either equal of as prefix of the this sub tree, we expand it till
we find n candidates. Function expand() can be reused here.

We can test this program with the very same test cases and the results are
identical to the previous one.

\subsubsection*{An e-dictionary in C++}

In C++ implementation, we overload the look up function by providing an
extra integer n to indicate we want to search top n candidates. the
result is a list of key-value pairs,

\lstset{language=C++}
\begin{lstlisting}
//lookup top n candidate with prefix key in Trie
template<class K, class V>
std::list<std::pair<K, V> > lookup(Trie<K, V>* t,
                                   typename Trie<K, V>::KeyType key,
                                   unsigned int n)
{
  typedef std::list<std::pair<K, V> > Result;
  if(!t)
    return Result();

  Trie<K, V>* p(t);
  for(typename K::iterator it=key.begin(); it!=key.end(); ++it){
    if(p->children.find(*it) == p->children.end())
      return Result();
    p = p->children[*it];
  }
  return expand(key, p, n);
}
\end{lstlisting}

The program is almost same as the Trie looking up one, except it will
call expand function when it located the node with the key. Function
expand is as the following.

\begin{lstlisting}
template<class T>
std::list<std::pair<typename T::KeyType, typename T::ValueType> >
expand(typename T::KeyType prefix, T* t, unsigned int n)
{
  typedef typename T::KeyType KeyType;
  typedef typename T::ValueType ValueType;
  typedef std::list<std::pair<KeyType, ValueType> > Result;

  Result res;
  std::queue<std::pair<KeyType, T*> > q;
  q.push(std::make_pair(prefix, t));
  while(res.size()<n && (!q.empty())){
    std::pair<KeyType, T*> i = q.front();
    KeyType s = i.first;
    T* p = i.second;
    q.pop();
    if(p->value != ValueType()){
      res.push_back(std::make_pair(s, p->value));
    }
    for(typename T::Children::iterator it = p->children.begin();
        it!=p->children.end(); ++it)
      q.push(std::make_pair(s+it->first, it->second));
  }
  return res;
}
\end{lstlisting}

This function use a bread-first search approach to expand top N
candidates, it maintain a queue to store the node it is currently
dealing with. Each time the program picks a candidate node from the
queue, expands all its children and put them to the queue. the program
will terminate when the queue is empty or we have already found N
candidates.

Function expand is generic we'll use it in later sections.

Then we can provide a helper function to convert the candidate
list to readable string. Note that this list is actually a list of pairs so we can
provide a generic function.

\begin{lstlisting}
//list of pairs to string
template<class Container>
std::string lop_to_str(Container coll){
  typedef typename Container::iterator Iterator;
  std::ostringstream s;
  s<<"[";
  for(Iterator it=coll.begin(); it!=coll.end(); ++it)
    s<<"("<<it->first<<", "<<it->second<<"), ";
  s<<"]";
  return s.str();
}
\end{lstlisting}

After that, we can test the program with some simple test cases.

\begin{lstlisting}
Trie<std::string, std::string>* t(0);
const char* dict[] = {
  "a", "the first letter of English", \
  "an", "used instead of 'a' when the following word begins with a vowel sound", \
  "another", "one more person or thing or an extra amount", \
  "abandon", "to leave a place, thing or person forever", \
  "about", "on the subject of; connected with", \
  "adam", "a character in the Bible who was the first man made by God", \
  "boy", "a male child or, more generally, a male of any age", \
  "body", "the whole physical structure that forms a person or animal", \
  "zoo", "an area in which animals, especially wild animals, are kept" \
         "so that people can go and look at them, or study them"};

const char** first=dict;
const char** last =dict + sizeof(dict)/sizeof(char*);
for(;first!=last; ++first, ++first)
  t = insert(t, *first, *(first+1));
}

std::cout<<"test lookup top 5 in Trie\n"
         <<"search a "<<lop_to_str(lookup(t, "a", 5))<<"\n"
         <<"search ab "<<lop_to_str(lookup(t, "ab", 5))<<"\n";
delete t;
\end{lstlisting}

The result print to the console is something like this:

\begin{verbatim}
test lookup top 5 in Trie
search a [(a, the first letter of English), (an, used instead of 'a'
when the following word begins with a vowel sound), (adam, a character
in the Bible who was the first man made by God), (about, on the
subject of; connected with), (abandon, to leave a place, thing or
person forever), ]
search ab [(about, on the subject of; connected with), (abandon, to
leave a place, thing or person forever), ]
\end{verbatim}

To save the the space with Patricia, we provide a C++ program to
search top N candidate as below.

\begin{lstlisting}
template<class K, class V>
std::list<std::pair<K, V> > lookup(Patricia<K, V>* t,
                                   typename Patricia<K, V>::KeyType key,
                                   unsigned int n)
{
  typedef typename std::list<std::pair<K, V> > Result;
  typedef typename Patricia<K, V>::Children::iterator Iterator;
  if(!t)
    return Result();
  K prefix;
  for(;;){
    bool match(false);
    for(Iterator it=t->children.begin(); it!=t->children.end(); ++it){
      K k(it->first);
      if(is_prefix_of(key, k))
        return expand(prefix+k, it->second, n);
      if(is_prefix_of(k, key)){
        match = true;
        prefix += k;
        lcp<K>(key, k); //update key
        t = it->second;
        break;
      }
    }
    if(!match)
      return Result();
  }
}
\end{lstlisting}

The program iterate all children if the string we are looked up is
prefix of one child, we expand this child to find top N candidates; If
the in the opposite case, we update the string and go on examine into
this child Patricia tree.

Where the function is\_prefix\_of() is defined as below.

\begin{lstlisting}
// x `is prefix of` y?
template<class T>
bool is_prefix_of(T x, T y){
  if(x.size()<=y.size())
    return std::equal(x.begin(), x.end(), y.begin());
  return false;
}
\end{lstlisitng}

We use STL equal function to check if x is prefix of y.

The test case is nearly same as the one in Trie.

\begin{lstlisting}
Patricia<std::string, std::string>* t(0);
const char* dict[] = {
  "a", "the first letter of English", \
  "an", "used instead of 'a' when the following word begins with a vowel sound", \
  "another", "one more person or thing or an extra amount", \
  "abandon", "to leave a place, thing or person forever", \
  "about", "on the subject of; connected with", \
  "adam", "a character in the Bible who was the first man made by God", \
  "boy", "a male child or, more generally, a male of any age", \
  "body", "the whole physical structure that forms a person or animal", \
  "zoo", "an area in which animals, especially wild animals, are kept" \
         "so that people can go and look at them, or study them"};

const char** first=dict;
const char** last =dict + sizeof(dict)/sizeof(char*);
for(;first!=last; ++first, ++first)
  t = insert(t, *first, *(first+1));
}

std::cout<<"test lookup top 5 in Trie\n"
         <<"search a "<<lop_to_str(lookup(t, "a", 5))<<"\n"
         <<"search ab "<<lop_to_str(lookup(t, "ab", 5))<<"\n";
delete t;
\end{lstlisting}

This test case will output a very same result in console.

\subsubsection{Recursive algorithm of search top N candidate in
  Patricia}

This algorithm can also be implemented recursively, if the string we
are looking for is empty, we expand all children until we get N
candidates. else we recursively examine the children of the node to
see if we can find one has prefix as this string.

\begin{algorithmic}[1]
\Function{PATRICIA-LOOK-UP-TOP-N}{$T, key, N$}
  \If{$T = NIL$}
    \State \Return $NIL$
  \EndIf

  \If{$KEY = NIL$}
    \State \Return $EXPAND-TOP-N(T, NIL, N)$
  \Else
    \State \Return $FIND-IN-CHILDREN-TOP-N(CHILDREN(T), key, N)$
  \EndIf
\EndFunction

\Function{FIND-IN-CHILDREN-TOP-N}{$l, key, N$}
  \If{$l = NIL$}
    \State \Return $NIL$
  \ElsIf{$KEY(FIRST(l)) = key$}
    \State \Return $EXPAND-TOP-N(FIRST(l), key, N)$
  \ElsIf{$KEY(FIRST(l)$ is prefix of $key$}
    \State \Return $PATRICIA-LOOK-UP-TOP-N(FIRST(l), key$ subtract
    $KEY(FIRST(l)))$
  \ElsIf{$key$ is prefix of $KEY(FIRST(l)$}
    \State \Return $PATRICIA-LOOK-UP-TOP-N(FIRST(l), NIL, N)$
  \ElsIf
    \State \Return $FIND-IN-CHILDREN-TOP-N(REST(l), key, N)$
  \EndIf
\EndFunction
\end{algorithmic}

\subsubsection*{An e-dictionary in Haskell}
In Haskell implementation, we provide a function named as findAll.
Thanks for the lazy evaluation support, findAll won't produce all candidates
words until we need them. we can use something like 'take 10 findAll'
to get the top 10 words easily.

findAll is given as the following.

\lstset{language=Haskell}
\begin{lstlisting}
findAll:: Trie a -> String -> [(String, a)]
findAll t [] =
    case value t of
      Nothing -> enum (children t)
      Just x  -> ("", x):(enum (children t))
    where
      enum [] = []
      enum (p:ps) = (mapAppend (fst p) (findAll (snd p) [])) ++ (enum ps)
findAll t (k:ks) =
    case lookup k (children t) of
      Nothing -> []
      Just t' -> mapAppend k (findAll t' ks)

mapAppend x lst = map (\p->(x:(fst p), (snd p))) lst
\end{lstlisting}

function findAll take a Trie, a word to be looked up, it will output
a list of pairs, the first element of the pair is the candidate word,
the second element of the pair is the meaning of the word.

Compare with the find function of Trie, the none-trivial case is very similar.
We take a letter form the words to be looked up, if there is no child starting
with this letter, the program returns empty list. If there is such a child
starting with this letter, this child should be a candidate. We use function
mapAppend to add this letter in front of all elements of recursively founded
candidate words.

In case we consumed all letters, we next returns all potential words, which
means the program will traverse all children of the current node.

Note that only the node with value field not equal to 'None' is a meaningful
word in our dictionary. We need append the list with the right meaning.

With this function, we can construct a very simple dictionary and return
top 5 candidate to user. Here is the test program.

\begin{lstlisting}
testFindAll = "\nlook up a: " ++ (show $ take 5 $findAll t "a") ++
              "\nlook up ab: " ++ (show $ take 5 $findAll t "ab")
    where
      t = fromList [
        ("a", "the first letter of English"),
        ("an", "used instead of 'a' when the following word begins with"
               "a vowel sound"),
        ("another", "one more person or thing or an extra amount"),
        ("abandon", "to leave a place, thing or person forever"),
        ("about", "on the subject of; connected with"),
        ("adam", "a character in the Bible who was the first man made by God"),
        ("boy", "a male child or, more generally, a male of any age"),
        ("body", "the whole physical structure that forms a person or animal"),
        ("zoo", "an area in which animals, especially wild animals, are kept"
                " so that people can go and look at them, or study them")]

main = do
    putStrLn testFindAll
\end{lstlisting}

This program will out put a result like this:
\begin{verbatim}
look up a: [("a","the first letter of English"),("an","used instead of 'a'
when the following word begins with a vowel sound"),("another","one more
person or thing or an extra amount"),("abandon","to leave a place, thing
or person forever"),("about","on the subject of; connected with")]
look up ab: [("abandon","to leave a place, thing or person forever"),
("about","on the subject of; connected with")]
\end{verbatim}

The Trie solution wasts a lot of spaces. It is very easy to improve the above
program with Patricia. Below source code shows the Patricia approach.

\begin{lstlisting}
findAll' :: Patricia a -> Key -> [(Key, a)]
findAll' t [] =
    case value t of
      Nothing -> enum $ children t
      Just x  -> ("", x):(enum $ children t)
    where
      enum [] = []
      enum (p:ps) = (mapAppend' (fst p) (findAll' (snd p) [])) ++ (enum ps)
findAll' t k = find' (children t) k where
    find' [] _ = []
    find' (p:ps) k
          | (fst p) == k
              = mapAppend' k (findAll' (snd p) [])
          | (fst p) `Data.List.isPrefixOf` k
              = mapAppend' (fst p) (findAll' (snd p) (k `diff` (fst p)))
          | k `Data.List.isPrefixOf` (fst p)
              = findAll' (snd p) []
          | otherwise = find' ps k
    diff x y = drop (length y) x

mapAppend' s lst = map (\p->(s++(fst p), snd p)) lst
\end{lstlisting}

If compare this program with the one implemented by Trie, we can find they are
very similar to each other. In none-trivial case, we just examine each child
to see if any one match the key to be looked up. If one child is exactly equal
to the key, we then expand all its sub branches and put them to the candidate list.
If the child correspond to a prefix of the key, the program goes on find the
the rest part of the key along this child and concatenate this prefix to all
later results. If the current key is prefix to a child, the program will traverse
this child and return all its sub branches as candidate list.

This program can be tested with the very same case as above, and it will output
the same result.

\subsubsection*{An e-dictionary in Scheme/Lisp}

In Scheme/Lisp implementation with Trie, a function named find is used
to search all candidates start with a given string. If the string is
empty, the program will enumerate all sub trees as result; else the
program calls an inner function find-child to search a child which
matches the first character of the given string. Then the program
recursively apply the find function to this child with the rest
characters of the string to be searched.

\lstset{language=lisp}
\begin{lstlisting}
(define (find t k)
  (define (find-child lst k)
    (tree (find-matching-item lst (lambda (c) (string=? (key c) k)))))
  (if (string-null? k)
      (enumerate t)
      (let ((t-new (find-child (children t) (string-car k))))
	(if (null? t-new) '()
	  (map-string-append (string-car k) (find t-new (string-cdr k)))))))
\end{lstlisting}

Note that the map-string-append will insert the first character to all
the elements (more accurately, each element is a pair with a key and a
value, map-string-append insert the character in front of each key) in
the result returned by recursive call. It is defined like this.

\begin{lstlisting}
(define (map-string-append x lst) ;; lst: [(key value)]
  (map (lambda (p) (cons (string-append x (car p)) (cdr p))) lst))
\end{lstlisting}

The enumerate function which can expend all sub trees is implemented
as the following.

\begin{lstlisting}
(define (enumerate t) ;; enumerate all sub trees
  (if (null? t) '()
      (let ((res (append-map
		  (lambda (p)(map-string-append (key p)(enumerate (tree p))))
		  (children t))))
	(if (null? (value t)) res
	    (cons (cons "" (value t)) res)))))
\end{lstlisting}

The test case is a very simple list of word-meaning pairs.

\begin{lstlisting}
(define dict
  (list '("a" "the first letter of English")
	'("an" "used instead of 'a' when the following word begins with a vowel sound")
	'("another" "one more person or thing or an extra amount")
	'("abandon" "to leave a place, thing or person forever")
	'("about" "on the subject of; connected with")
	'("adam" "a character in the Bible who was the first man made by God")
	'("boy" "a male child or, more generally, a male of any age")
	'("body" "the whole physical structure that forms a person or animal")
	'("zoo" "an area in which animals, especially wild animals,
  are kept so that people can go and look at them, or study them")))
\end{lstlisting}

After feed this dict to a Trie, if user tries to find 'a*' or 'ab*'
like below.

\begin{lstlisting}
(define (test-trie-find-all)
  (define t (list->trie dict))
  (display "find a*: ") (display (find t "a")) (newline)
  (display "find ab*: ") (display (find t "ab")) (newline))
\end{lstlisting}

The result is a list with all candidates start with the given string.
\begin{verbatim}
(test-trie-find-all)
find a*: ((a . the first letter of English) (an . used instead of 'a'
when the following word begins with a vowel sound) (another . one more
person or thing or an extra amount) (abandon . to leave a place, thing
or person forever) (about . on the subject of; connected with) (adam
. a character in the Bible who was the first man made by God))
find ab*: ((abandon . to leave a place, thing or person forever)
(about . on the subject of; connected with))
\end{verbatim}

Trie approach isn't space effective. Patricia can be one alternative
to improve in terms of space.

We can fully reuse the function enumerate, map-string-append which are
defined for trie. the find function for Patricia is implemented as the
following.

\begin{lstlisting}
(define (find t k)
  (define (find-child lst k)
    (if (null? lst) '()
	(cond ((string=? (key (car lst)) k)
	       (map-string-append k (enumerate (tree (car lst)))))
	      ((string-prefix? (key (car lst)) k)
	       (let ((k-new (string-tail k (string-length (key (car lst))))))
		 (map-string-append (key (car lst)) (find (tree (car lst)) k-new))))
	      ((string-prefix? k (key (car lst))) (enumerate (tree (car lst))))
	      (else (find-child (cdr lst) k)))))
  (if (string-null? k)
      (enumerate t)
      (find-child (children t) k)))
\end{lstlisting}

If the same test cases of search all candidates of 'a*' and 'ab*' are
fed we can get a very same result.

%=====================================
% T9
%=====================================

\subsection{T9 input method}
Most mobile phones around year 2000 has a key pad. To edit a short message/email
with such key-pad, users typically have quite different experience from PC.
Because a mobile-phone key pad, or so called ITU-T key pad has few keys.
Figure {fig:itut-keypad} shows an example.

\begin{figure}[htbp]
       \begin{center}
	\includegraphics[scale=0.5]{img/itu-t.eps}
        \caption{an ITU-T keypad for mobile phone.}
        \label{fig:itut-keypad}
       \end{center}
\end{figure}

There are typical 2 methods to input an English word/phrase with ITU-T key pad.
For instance, if user wants to enter a word ``home'', He can press the key
in below sequence.

\begin{itemize}
\item Press key '4' twice to enter the letter 'h';
\item Press key '6' three times to enter the letter 'o';
\item Press key '6' twice to enter the letter 'm';
\item Press key '3' twice to enter the letter 'e';
\end{itemize}

Another high efficient way is to simplify the key press sequence like the
following.

\begin{itemize}
\item Press key '4', '6', '6', '3', word ``home'' appears on top of the candidate list;
\item Press key '*' to change a candidate word, so word ``good'' appears;
\item Press key '*' again to change another candidate word, next word ``gone'' appears;
\item ...
\end{itemize}

Compare these 2 method, we can see method 2 is much easier for the end user, and
it is operation efficient. The only overhead is to store a candidate words dictionary.

Method 2 is called as T9 input method, or predictive input method
\cite{wiki-t9}, \cite {wiki-predictive-text}. The abbreviation 'T9' stands
for 'textonym'. In this section, I'll show an example implementation of T9
by using Trie and Patricia.

In order to provide candidate words to user, a dictionary must be prepared
in advance. Trie or Patricia can be used to store the Dictionary. In the real
commercial software, complex indexing dictionary is used. We show the very
simple Trie and Patricia only for illustration purpose.

\subsubsection{Iterative algorithm of T9 looking up}

Below pseudo code shows how to realize T9 with Trie.

\begin{algorithmic}[1]
\Function{TRIE-LOOK-UP-T9}{$T, key$}
  \State $PUSH-BACK(Q, NIL, key, T)$
  \State $r \leftarrow NIL$
  \While{$Q$ is not empty}
    \State $p, k, t \leftarrow POP-FRONT(Q)$
    \State $i \leftarrow FIRST-LETTER(k)$
    \For{each $c$ in $T9-MAPPING(i)$}
      \If{$c$ is in $CHILDREN(t)$}
        \State $k' \leftarrow k$ subtract $i$
        \If{$k'$ is empty}
          \State $APPEND(r, p+c)$
        \Else
          \State $PUSH-BACK(Q, p+c, k', CHILDREN(t)[c])$
        \EndIf
      \EndIf
    \EndFor
  \EndWhile
  \State \Return $r$
\EndFunction
\end{algorithmic}

This is actually a bread-first search program. It utilizes a queue to store
the current node and key string we are examining. The algorithm takes the first
digit from the key, looks up it in T9 mapping to get all English letters
corresponding to this digit. For each letter, if it can be found in the children
of current node, the node along with the English string found so far are
push back to the queue. In case all digits are examined, a candidate is found.
We'll append this candidate to the result list. The loop will terminate when
the queue is empty.

Since Trie is not space effective, minor modification of the above program can
work with Patricia, which can help to save extra spaces.

\begin{algorithmic}[1]
\Function{PATRICIA-LOOK-UP-T9}{$T, key$}
  \State $PUSH-BACK(Q, NIL, key, T)$
  \State $r \leftarrow NIL$
  \While{$Q$ is not empty}
    \State $p, k, t \leftarrow POP-FRONT(Q)$
    \For{each $child$ in $CHILDREN(t)$}
      \State $k' \leftarrow CONVERT-T9(KEY(child))$
      \If{$k'$ IS-PREFIX-OF $k$}
        \If{$k' = k$}
          \State $APPEND(r, p+KEY(child))$
        \Else
          \State $PUSH-BACK(Q, p+KEY(child), k-k', child)$
        \EndIf
      \EndIf
    \EndFor
  \EndWhile
  \State \Return $r$
\EndFunction
\end{algorithmic}

\subsubsection*{T9 implementation in Python}

In Python implementation, T9 looking up is realized in a typical bread-first search algorithm as the following.

\lstset{language=Python}
\begin{lstlisting}
T9MAP={'2':"abc", '3':"def", '4':"ghi", '5':"jkl", \
       '6':"mno", '7':"pqrs", '8':"tuv", '9':"wxyz"}

def trie_lookup_t9(t, key):
    if t is None or key == "":
        return None
    q = [("", key, t)]
    res = []
    while len(q)>0:
        (prefix, k, t) = q.pop(0)
        i=k[0]
        if not i in T9MAP:
            return None #invalid input
        for c in T9MAP[i]:
            if c in t.children:
                if k[1:]=="":
                    res.append((prefix+c, t.children[c].value))
                else:
                    q.append((prefix+c, k[1:], t.children[c]))
    return res
\end{lstlisting}

Function trie\_lookup\_t9 check if the parameters are valid first. Then
it push the initial data into a queue. The program repeatedly pop the item
from the queue, including what node it will examine next, the number sequence
string, and the alphabetic string it has been searched.

For each popped item, the program takes the next digit from the number
sequence, and looks up in T9 map to find the corresponding English letters.
With all these letters, if they can be found in the children of the current
node, we'll push this child along with the updated number sequence string
and updated alphabetic string into the queue. In case we process all
numbers, we find a candidate result.

We can verify the above program with the following test cases.

\begin{lstlisting}
class LookupTest:
    def __init__(self):
        t9dict = ["home", "good", "gone", "hood", "a", "another", "an"]
        self.t9t = trie.list_to_trie(t9dict)

    def test_trie_t9(self):
        print "search 4 ", trie_lookup_t9(self.t9t, "4")
        print "search 46 ", trie_lookup_t9(self.t9t, "46")
        print "search 4663 ", trie_lookup_t9(self.t9t, "4663")
        print "search 2 ", trie_lookup_t9(self.t9t, "2")
        print "search 22 ", trie_lookup_t9(self.t9t, "22")
\end{lstlisting}

If we run the test, it will output a very same result as the above Haskell
program.

\begin{verbatim}
search 4  [('g', None), ('h', None)]
search 46  [('go', None), ('ho', None)]
search 4663  [('gone', None), ('good', None), ('home', None), ('hood', None)]
search 2  [('a', None)]
search 22  []
\end{verbatim}

To save the spaces, Patricia can be used instead of Trie.

\begin{lstlisting}
def patricia_lookup_t9(t, key):
    if t is None or key == "":
        return None
    q = [("", key, t)]
    res = []
    while len(q)>0:
        (prefix, key, t) = q.pop(0)
        for k, tr in t.children.items():
            digits = toT9(k)
            if string.find(key, digits)==0: #is prefix of
                if key == digits:
                    res.append((prefix+k, tr.value))
                else:
                    q.append((prefix+k, key[len(k):], tr))
    return res
\end{lstlisting}

Compare to the implementation with Trie, they are very similar. We also
used a bread-first search approach. The different part is that we convert
the string of each child to number sequence string according to T9 mapping.
if it is prefix of the key we are looking for, we push this child along with
updated key and prefix. In case we examined all digits, we find a candidate
result.

The convert function is a reverse mapping process as below.
\begin{lstlisting}
def toT9(s):
    res=""
    for c in s:
        for k, v in T9MAP.items():
            if string.find(v, c)>=0:
                res+=k
                break
        #error handling skipped.
    return res
\end{lstlisting}

For illustration purpose, the error handling for invalid letters is skipped.
If we feed the program with the same test cases, we can get a result as the
following.

\begin{verbatim}
search 4  []
search 46  [('go', None), ('ho', None)]
search 466  []
search 4663  [('good', None), ('gone', None), ('home', None), ('hood', None)]
search 2  [('a', None)]
search 22  []
\end{verbatim}

The result is slightly different from the one output by Trie. The reason is
as same as what we analyzed in Haskell implementation. It is easily to modify
the program to output a similar result.

\subsubsection*{T9 implemented in C++}

First we define T9 mapping as a Singleton object, this is because we
want to it can be used both in Trie look up and Patricia look up programs.

\lstset{language=C++}
\begin{lstlisting}
struct t9map{
  typedef std::map<char, std::string> Map;
  Map map;

  t9map(){
    map['2']="abc";
    map['3']="def";
    map['4']="ghi";
    map['5']="jkl";
    map['6']="mno";
    map['7']="pqrs";
    map['8']="tuv";
    map['9']="wxyz";
  }

  static t9map& inst(){
    static t9map i;
    return i;
  }
};
\end{lstlisting}

Note in other languages or keypad layout, we can define different
mappings and pass them as an argument to the looking up function.

With this mapping, the looking up in Trie can be given as
below. Although we want to keep the genericity of the program, for
illustration purpose, we just simply use the t9 mapping directly.

In order to keep the code as short as possible, a boost library tool,
boost::tuple is used. For more about boost::tuple, please refer to \cite{boost-book}.

\begin{lstlisting}
template<class K, class V>
std::list<std::pair<K, V> > lookup_t9(Trie<K, V>* t,
				      typename Trie<K, V>::KeyType key)
{
  typedef std::list<std::pair<K, V> > Result;
  typedef typename Trie<K, V>::KeyType Key;
  typedef typename Trie<K, V>::Char Char;

  if((!t) || key.empty())
    return Result();

  Key prefix;
  std::map<Char, Key> m = t9map::inst().map;
  std::queue<boost::tuple<Key, Key, Trie<K, V>*> > q;
  q.push(boost::make_tuple(prefix, key, t));
  Result res;
  while(!q.empty()){
    boost::tie(prefix, key, t) = q.front();
    q.pop();
    Char c = *key.begin();
    key = Key(key.begin()+1, key.end());
    if(m.find(c) == m.end())
      return Result();
    Key cs = m[c];
    for(typename Key::iterator it=cs.begin(); it!=cs.end(); ++it)
      if(t->children.find(*it)!=t->children.end()){
	if(key.empty())
	  res.push_back(std::make_pair(prefix+*it, t->children[*it]->value));
	else
	  q.push(boost::make_tuple(prefix+*it, key, t->children[*it]));
      }
  }
  return res;
}
\end{lstlisting}

This program will first check if the Patricia tree or the key are
empty to deal with with trivial case. It next initialize a queue, and
push one tuple to it. the tuple contains 3 elements, a prefix to
represent a string the program has been searched, current key it need
look up, and a node it will examine.

Then the program repeatedly pops the tuple from the queue, takes the
first character from the key, and looks up in T9 map to get a
candidate English letter list. With each letter in this list, the
program examine if it exists in the children of current node. In case
it find such a child, if there is no left letter to look up, it means
we found a candidate result, we push it to the result list. Else, we
create a new tuple with updated prefix, key and this child; the push
it to the queue for later process.

Below are some simple test cases for verification.

\begin{lstlisting}
Trie<std::string, std::string>* t9trie(0);
const char* t9dict[] = {"home", "good", "gone", "hood", "a", "another", "an"};
t9trie = list_to_trie(t9dict, t9dict+sizeof(t9dict)/sizeof(char*), t9trie);
std::cout<<"test t9 lookup in Trie\n"
         <<"search 4 "<<lop_to_str(lookup_t9(t9trie, "4"))<<"\n"
	 <<"serach 46 "<<lop_to_str(lookup_t9(t9trie, "46"))<<"\n"
	 <<"serach 4663 "<<lop_to_str(lookup_t9(t9trie, "4663"))<<"\n"
	 <<"serach 2 "<<lop_to_str(lookup_t9(t9trie, "2"))<<"\n"
	 <<"serach 22 "<<lop_to_str(lookup_t9(t9trie, "22"))<<"\n\n";
delete t9trie;
\end{lstlisting}

It will output the same result as the Python program.

\begin{verbatim}
test t9 lookup in Trie
search 4 [(g, ), (h, ), ]
serach 46 [(go, ), (ho, ), ]
serach 4663 [(gone, ), (good, ), (home, ), (hood, ), ]
serach 2 [(a, ), ]
serach 22 []
\end{verbatim}

In order to save space, a looking up program for Patricia is also
provided.

\begin{lstlisting}
template<class K, class V>
std::list<std::pair<K, V> > lookup_t9(Patricia<K, V>* t,
				      typename Patricia<K, V>::KeyType key)
{
  typedef std::list<std::pair<K, V> > Result;
  typedef typename Patricia<K, V>::KeyType Key;
  typedef typename Key::value_type Char;
  typedef typename Patricia<K, V>::Children::iterator Iterator;

  if((!t) || key.empty())
    return Result();

  Key prefix;
  std::map<Char, Key> m = t9map::inst().map;
  std::queue<boost::tuple<Key, Key, Patricia<K, V>*> > q;
  q.push(boost::make_tuple(prefix, key, t));
  Result res;
  while(!q.empty()){
    boost::tie(prefix, key, t) = q.front();
    q.pop();
    for(Iterator it=t->children.begin(); it!=t->children.end(); ++it){
      Key digits = t9map::inst().toT9(it->first);
      if(is_prefix_of(digits, key)){
	if(digits == key)
	  res.push_back(std::make_pair(prefix+it->first, it->second->value));
	else{
	  key =Key(key.begin()+it->first.size(), key.end());
	  q.push(boost::make_tuple(prefix+it->first, key, it->second));
	}
      }
    }
  }
  return res;
}
\end{lstlisting}

The program is similar to the one with Trie very much. This is a
typical bread-first search approach. Note that we added a member
function to\_t9() to convert a English word/phrase back to digit
number string. This member function is implemented as the following.

\begin{lstlisting}
struct t9map{
  //...
  std::string to_t9(std::string s){
    std::string res;
    for(std::string::iterator c=s.begin(); c!=s.end(); ++c){
      for(Map::iterator m=map.begin(); m!=map.end(); ++m){
	std::string val = m->second;
	if(std::find(val.begin(), val.end(), *c)!=val.end()){
	  res.push_back(m->first);
	  break;
	}
      }
    } // skip error handling.
    return res;
  }
\end{lstlisting}

The error handling for invalid letters is omitted in order to keep the
code short for easy understanding. We can use the very similar test
cases as above except we need change the Trie to Patrica. It will
output as below.

\begin{verbatim}
test t9 lookup in Patricia
search 4 []
serach 46 [(go, ), (ho, ), ]
serach 466 []
serach 4663 [(gone, ), (good, ), ]
serach 2 [(a, ), ]
serach 22 []
\end{verbatim}

The result is slightly different, please refer to the Haskell section
for the reason of this difference. It is very easy to modify the
program to output the very same result as Trie's one.

\subsubsection{Recursive algorithm of T9 looking up}

\subsubsection*{T9 implemented in Haskell}

In Haskell, we first define a map from key pad to English letter. When user
input a key pad number sequence, we take each number and check from the Trie.
All children match the number should be investigated. Below is a Haskell
program to realize T9 input.

\lstset{language=Haskell}
\begin{lstlisting}
mapT9 = [('2', "abc"), ('3', "def"), ('4', "ghi"), ('5', "jkl"),
         ('6', "mno"), ('7', "pqrs"), ('8', "tuv"), ('9', "wxyz")]

lookupT9 :: Char -> [(Char, b)] -> [(Char, b)]
lookupT9 c children = case lookup c mapT9 of
        Nothing -> []
        Just s  -> foldl f [] s where
             f lst x = case lookup x children of
                 Nothing -> lst
                 Just t  -> (x, t):lst

-- T9-find in Trie
findT9:: Trie a -> String -> [(String, Maybe a)]
findT9 t [] = [("", Trie.value t)]
findT9 t (k:ks) = foldl f [] (lookupT9 k (children t))
    where
      f lst (c, tr) = (mapAppend c (findT9 tr ks)) ++ lst
\end{lstlisting}

findT9 is the main function, it takes 2 parameters, a Trie and a number
sequence string. In non-trivial case, it calls lookupT9 function to
examine all children which match the first number.

For each matched child, the program recursively calls findT9 on it with
the left numbers, and we use mapAppend to insert the currently finding letter
in front of all results. The program use foldl to combine all these together.

Function lookupT9 is used to filtered all possible children who match a
number. It first call lookup function on mapT9, so that a string of possible
English letters can be identified. Next we call lookup for each candidate
letter to see if there is a child can match the letter. We use foldl to
collect all such child together.

This program can be verified by using some simple test cases.

\begin{lstlisting}
testFindT9 = "press 4: " ++ (show $ take 5 $ findT9 t "4")++
             "\npress 46: " ++ (show $ take 5 $ findT9 t "46")++
             "\npress 4663: " ++ (show $ take 5 $ findT9 t "4663")++
             "\npress 2: " ++ (show $ take 5 $ findT9 t "2")++
             "\npress 22: " ++ (show $ take 5 $ findT9 t "22")
    where
      t = Trie.fromList lst
      lst = [("home", 1), ("good", 2), ("gone", 3), ("hood", 4),
             ("a", 5), ("another", 6), ("an", 7)]
\end{lstlisting}

The program will output below result.

\begin{verbatim}
press 4: [("g",Nothing),("h",Nothing)]
press 46: [("go",Nothing),("ho",Nothing)]
press 4663: [("gone",Just 3),("good",Just 2),("home",Just 1),("hood",Just 4)]
press 2: [("a",Just 5)]
press 22: []
\end{verbatim}

The value of each child is just for illustration, we can put empty value instead
and only returns candidate keys for a real input application.

Tries consumes to many spaces, we can provide a Patricia version as alternative.

\begin{lstlisting}
findPrefixT9' :: String -> [(String, b)] -> [(String, b)]
findPrefixT9' s lst = filter f lst where
    f (k, _) = (toT9 k) `Data.List.isPrefixOf` s

toT9 :: String -> String
toT9 [] = []
toT9 (x:xs) = (unmapT9 x mapT9):(toT9 xs) where
    unmapT9 x (p:ps) = if x `elem` (snd p) then (fst p) else unmapT9 x ps

findT9' :: Patricia a -> String -> [(String, Maybe a)]
findT9' t [] = [("", value t)]
findT9' t k = foldl f [] (findPrefixT9' k (children t))
    where
      f lst (s, tr) = (mapAppend' s (findT9' tr (k `diff` s))) ++ lst
      diff x y = drop (length y) x
\end{lstlisting}

In this program, we don't check one digit at a time, we take all the digit
sequence, and we examine all children of the Patricia node. For each child,
the program convert the prefix string to number sequence by using function
toT9, if the result is prefix of what user input, we go on search in this
child and append the prefix in front of all further results.

If we tries the same test case, we can find the result is a bit different.

\begin{verbatim}
press 4: []
press 46: [("go",Nothing),("ho",Nothing)]
press 466: []
press 4663: [("good",Just 2),("gone",Just 3),("home",Just 1),("hood",Just 4)]
press 2: [("a",Just 5)]
press 22: []
\end{verbatim}

If user press key '4', because the dictionary (represent by Patricia) doesn't
contain any candidates matches it, user will get an empty candidates list.
The same situation happens when he enters ``466''. In real input method
implementation, such user experience isn't good, because it displays nothing
although user presses the key several times. One improvement is to
predict what user will input next by display a partial result. This can be
easily achieved by modify the above program. (Hint: not only check
\begin{lstlisting}
findPrefixT9' s lst = filter f lst where
    f (k, _) = (toT9 k) `Data.List.isPrefixOf` s
\end{lstlisting}
but also check
\begin{lstlisting}
    f (k, _) = s `Data.List.isPrefixOf` (toT9 k)
\end{lstlisting}
)

\subsubsection*{T9 implemented in Scheme/Lisp}

In Scheme/Lisp, T9 map is defined as a list of pairs.

\lstset{language=lisp}
\begin{lstlisting}
(define map-T9 (list '("2" "abc") '("3" "def") '("4" "ghi") '("5" "jkl")
		     '("6" "mno") '("7" "pqrs") '("8" "tuv") '("9" "wxyz")))
\end{lstlisting}

The main searching function is implemented as the following.

\begin{lstlisting}
(define (find-T9 t k) ;; return [(key value)]
  (define (accumulate-find lst child)
    (append (map-string-append (key child) (find-T9 (tree child) (string-cdr k)))
	    lst))
  (define (lookup-child lst c) ;; lst, list of childen [(key tree)], c, char
    (let ((res (find-matching-item map-T9 (lambda (x) (string=? c (car x))))))
      (if (not res) '()
	  (filter (lambda (x) (substring? (key x) (cadr res))) lst))))
  (if (string-null? k) (list (cons k (value t)))
      (fold-left accumulate-find '() (lookup-child (children t) (string-car k)))))
\end{lstlisting}

This function contains 2 inner functions. If the string is empty, the
program returns a one element list. The element is a string value pair.
For the none trivial case, the program will call inner function to
find in each child and then put them together by using fold-left high
order function.

To test this T9 search function, a very simple dictionary is
established by using Trie insertion. Then we test by calling find-T9
function on several digits sequences.

\begin{lstlisting}
(define dict-T9 (list '("home" ()) '("good" ()) '("gone" ()) '("hood" ())
		      '("a" ()) '("another" ()) '("an" ())))

(define (test-trie-T9)
  (define t (list->trie dict-T9))
  (display "find 4: ") (display (find-T9 t "4")) (newline)
  (display "find 46: ") (display (find-T9 t "46")) (newline)
  (display "find 4663: ") (display (find-T9 t "4663")) (newline)
  (display "find 2: ") (display (find-T9 t "2")) (newline)
  (display "find 22: ") (display (find-T9 t "22")) (newline))
\end{lstlisting}

Evaluate this test function will output below result.

\begin{verbatim}
find 4: ((g) (h))
find 46: ((go) (ho))
find 4663: ((gone) (good) (hood) (home))
find 2: ((a))
find 22: ()
\end{verbatim}

In order to be more space effective, Patricia can be used to replace
Trie. The search program modified as the following.

\begin{lstlisting}
(define (find-T9 t k)
  (define (accumulate-find lst child)
    (append (map-string-append (key child) (find-T9 (tree child) (string- k (key child))))
	    lst))
  (define (lookup-child lst k)
    (filter (lambda (child) (string-prefix? (str->t9 (key child)) k)) lst))
  (if (string-null? k) (list (cons "" (value t)))
      (fold-left accumulate-find '() (lookup-child (children t) k))))
\end{lstlisting}

In this program a string helper function 'string-' is defined to get
the different part of two strings. It is defined like below.

\begin{lstlisting}
(define (string- x y)
  (string-tail x (string-length y)))
\end{lstlisting}

Another function is 'str->t9' it will convert a alphabetic string back
to digit sequence base on T9 mapping.

\begin{lstlisting}
(define (str->t9 s)
  (define (unmap-t9 c)
    (car (find-matching-item map-T9 (lambda (x) (substring? c (cadr x))))))
  (if (string-null? s) ""
      (string-append (unmap-t9 (string-car s)) (str->t9 (string-cdr s)))))
\end{lstlisting}

We can feed the almost same test cases, and the result is output as
the following.

\begin{verbatim}
find 4: ()
find 46: ((go) (ho))
find 466: ()
find 4663: ((good) (gone) (home) (hood))
find 2: ((a))
find 22: ()
\end{verbatim}

Note the result is a bit different, the reason is described in Haskell
section. It is easy to modify the program, so that Trie and Patricia
approach give the very same result.

% ================================================================
%                 Short summary
% ================================================================
\section{Short summary}

In this post, we start from the Integer base trie and Patricia, the
map data structure based on integer patricia plays an important role
in Compiler implementation. Next, alphabetic Trie and Patricia are
given, and I provide a example implementation to illustrate how to
realize a predictive e-dictionary and a T9 input method. Although they
are far from the real implementation in commercial software. They show
a very simple approach of manipulating text. There are still some
interesting problem can not be solved by Trie or Patrica directly, how
ever, some other data structures such as suffix tree have close
relationship with them. I'll note something about suffix tree in other post.

% ================================================================
%                 Appendix
% ================================================================
\section{Appendix} \label{appendix}
%\appendix
All programs provided along with this article are free for
downloading.

\subsection{Prerequisite software}
GNU Make is used for easy build some of the program. For C++ and ANSI C programs,
GNU GCC and G++ 3.4.4 are used. I use boost triple to reduce the
amount of our code lines, boost library version I am using is
1.33.1. The path is in CXX variable in Makefile, please change it to
your path when compiling.
For Haskell programs GHC 6.10.4 is used
for building. For Python programs, Python 2.5 is used for testing, for
Scheme/Lisp program, MIT Scheme 14.9 is used.

all source files are put in one folder. Invoke 'make' or 'make all'
will build C++ and Haskell program.

Run 'make Haskell' will separate build Haskell program. There will be
two executable file generated one is htest the other is happ (with .exe
in Window like OS). Run htest will test functions in IntTrie.hs, IntPatricia.hs,
Trie.hs and Patricia.hs. Run happ will execute the editionary and T9
test cases in EDict.hs.

Run 'make cpp' will build c++ program. It will create a executable
file named cpptest (with .exe in Windows like OS). Run this program
will test inttrie.hpp, intpatricia.hpp, trie.hpp, patricia.hpp, and
edict.hpp.

Run 'make c' will build the ANSI C program for Trie. It will create a
executable file named triec (with .exe in Windows like OS).

Python programs can run directly with interpreter.

Scheme/Lisp program need be loaded into Scheme evaluator and evaluate
the final function in the program. Note that patricia.scm will hide
some functions defined in trie.scm.

Here is a detailed list of source files

\subsection{Haskell source files}

\begin{itemize}
\item IntTrie.hs, Haskell version of little-endian integer Trie.
\item IntPatricia.hs, integer Patricia tree implemented in Haskell.
\item Trie.hs, Alphabetic Trie, implemented in Haskell.
\item Patricia.hs, Alphabetic Patricia, implemented in Haskell.
\item TestMain.hs, main module to test the above 4 programs.
\item EDict.hs, Haskell program for e-dictionary and T9.
\end{itemize}

\subsection{C++/C source files}

\begin{itemize}
\item inttrie.hpp, Integer base Trie;
\item intpatricia.hpp, Integer based Patricia tree;
\item trie.c, Alphabetic Trie only for lowercase English language,
implemented in ANSI C.
\item trie.hpp, Alphabetic Trie;
\item patricia.hpp, Alphabetic Patricia;
\item trieutil.hpp, Some generic utilities;
\item edit.hpp, e-dictionary and T9 implemented in C++;
\item test.cpp, main program to test all above programs.
\end{itemize}

\subsection{Python source files}

\begin{itemize}
\item inttrie.py, Python version of little-endian integer Trie, with
test cases;
\item intpatricia.py, integer Patricia tree implemented in Python;
\item trie.py, Alphabetic Trie, implemented in Python;
\item patricia.py, Alphabetic Patricia implemented in Python;
\item trieutil.py, Common utilities;
\item edict.py, e-dictionary and T9 implemented in Python.
\end{itemize}

\subsection{Scheme/Lisp source files}

\begin{itemize}
\item inttrie.scm, Little-endian integer Trie, implemented in Scheme/Lisp;
\item intpatricia.scm, Integer based Patricia tree;
\item trie.scm, Alphabetic Trie;
\item patricia.scm, Alphabetic Patricia, reused many definitions in
Trie;
\item trieutil.scm, common functions and utilities.
\end{itemize}

\subsection{Tools}

Besides them, I use graphviz to draw most of the figures in this post. In order to
translate the Trie, Patrica and Suffix Tree output to dot language scripts. I wrote a python program.
it can be used like this.

\begin{verbatim}
trie2dot.py -o foo.dot -t patricia "1:x, 4:y, 5:z"
trie2dot.py -o foo.dot -t trie "001:one, 101:five, 100:four"
\end{verbatim}

This helper scripts can also be downloaded with this article.

download position: http://sites.google.com/site/algoxy/trie/trie.zip

\begin{thebibliography}{99}

\bibitem{CLRS}
Thomas H. Cormen, Charles E. Leiserson, Ronald L. Rivest and Clifford Stein.
``Introduction to Algorithms, Second Edition''. Problem 12-1. ISBN:0262032937. The MIT Press. 2001

\bibitem{okasaki-int-map}
Chris Okasaki and Andrew Gill. ``Fast Mergeable Integer
Maps''. Workshop on ML, September 1998, pages 77-86, http://www.cse.ogi.edu/~andy/pub/finite.htm

\bibitem{patricia-morrison}
D.R. Morrison, ``PATRICIA -- Practical Algorithm To Retrieve  Information Coded In Alphanumeric", Journal of the ACM, 15(4), October 1968, pages 514-534.

\bibitem{wiki-suffix-tree}
Suffix Tree, Wikipedia. http://en.wikipedia.org/wiki/Suffix\_tree

\bibitem{wiki-trie}
Trie, Wikipedia. http://en.wikipedia.org/wiki/Trie

\bibitem{wiki-t9}
T9 (predictive text), Wikipedia. http://en.wikipedia.org/wiki/T9\_(predictive\_text)

\bibitem{wiki-predictive-text}
Predictive text,
Wikipedia. http://en.wikipedia.org/wiki/Predictive\_text

\bibitem{boost-book}
Bjorn Karlsson. ``Beyond the C++ Standard Library: An Introduction to
Boost''. Addison Wesley Professional, August 31, 2005, ISBN: 0321133544

\end{thebibliography}

\ifx\wholebook\relax\else
\end{document}
\fi
