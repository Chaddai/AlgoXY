\ifx\wholebook\relax \else
% ------------------------

\documentclass{article}
%------------------- Other types of document example ------------------------
%
%\documentclass[twocolumn]{IEEEtran-new}
%\documentclass[12pt,twoside,draft]{IEEEtran}
%\documentstyle[9pt,twocolumn,technote,twoside]{IEEEtran}
%
%-----------------------------------------------------------------------------
%\input{../../../common.tex}
\input{../../../common-en.tex}

\setcounter{page}{1}

\begin{document}

\fi
%--------------------------

% ================================================================
%                 COVER PAGE
% ================================================================

\title{B-Trees}

\author{Larry~LIU~Xinyu
\thanks{{\bfseries Larry LIU Xinyu } \newline
  Email: liuxinyu95@gmail.com \newline}
  }

\markboth{B-Trees}{AlgoXY}

\maketitle

\ifx\wholebook\relax
\chapter{B-Trees}
\numberwithin{Exercise}{chapter}
\section{abstract}
\fi

%{\bfseries Corresponding Author:} Larry LIU Xinyu

% ================================================================
%                 Introduction
% ================================================================
\section{Introduction}
\label{introduction}

B-Tree is important data structure.
It is widely used in modern file systems. Some
are implemented based on B+ tree, which is extended from B-tree.
B-tree is also widely used in database systems.

Some textbooks introduce B-tree with the the problem of how to access a
large block of data on magnetic disks or secondary storage devices\cite{CLRS}.
It is also helpful to understand B-tree as a generalization of balanced binary
search tree\cite{wiki-b-tree}.

Refer to the Figure \ref{fig:btree-example}, It is easy to find the difference
and similarity of B-tree regarding to binary search tree.

\begin{figure}[htbp]
   \begin{center}
	\includegraphics[scale=0.5]{img/btree-example.ps}
   \caption{Example B-Tree} \label{fig:btree-example}
   \end{center}
\end{figure}

Remind the definition of binary search tree. A binary search tree is
\begin{itemize}
\item either an empty node;
\item or a node contains 3 parts, a value, a left child and a right child.
Both children are also binary search trees.
\end{itemize}

The binary search tree satisfies the constraint that.
\begin{itemize}
\item all the values on the left child are not greater than the value of of this node;
\item the value of this node is not greater than any values on the right child.
\end{itemize}

For non-empty binary tree $(L, k, R)$, where $L$, $R$ and $k$
are the left, right children, and the key. Function $Key(T)$ accesses
the key of tree $T$.
The constraint can be represented as the following.

\begin{equation}
\forall x \in L, \forall y \in R \\
\Rightarrow Key(x) \leq k \leq Key(y)
\end{equation}

If we extend this definition to allow multiple keys and children, we get the
B-tree definition.

A B-tree
\begin{itemize}
\item is either empty;
\item or contains $n$ keys, and $n+1$ children, each child is
also a B-Tree, we denote these keys and children as $k_1, k_2, ..., k_n$
and $c_1, c_2, ..., c_n, c_{n+1}$.
\end{itemize}

Figure \ref{fig:btree-node} illustrates a B-Tree node.

\begin{figure}[htbp]
  \centering
	\includegraphics[scale=0.5]{img/btree-node.ps}
  \caption{A B-Tree node} \label{fig:btree-node}
\end{figure}

The keys and children in a node satisfy the following order constraints.

\begin{itemize}
\item Keys are stored in non-decreasing order. that $k_1 \leq k_2 \leq ... \leq k_n$;
\item for each $k_i$, all elements stored in child $c_i$ are not greater
than $k_i$, while $k_i$ is not greater than any values stored in child $c_{i+1}$.
\end{itemize}

The constraints can be represented as in equation (\ref{eq:btree-order})
as well.

\begin{equation}
\forall x_i \in c_i, i=0, 1, ..., n, \Rightarrow x_1 \leq k_1 \leq
x_2 \leq k_2 \leq ... \leq x_n \leq k_n \leq x_{n+1}
\label{eq:btree-order}
\end{equation}

Finally, after adding some constraints to make the tree balanced, we get the
complete B-tree definition.

\begin{itemize}
\item All leaves have the same depth;
\item We define integral number, $t$, as the {\em minimum degree} of
B-tree;
    \begin{itemize}
        \item each node can have at most $2t-1$ keys;
        \item each node can have at least $t-1$ keys, except the root;
    \end{itemize}
\end{itemize}

Consider a B-tree holds $n$ keys. The minimum degree $t \geq 2$.
The height is $h$. All the nodes have at least $t-1$ keys except the
root. The root contains at least 1 key. There are at least 2 nodes
at depth 1, at least $2t$ nodes at depth 2, at least $2t^2$ nodes
at depth 3, ..., finally, there are at least $2t^{h-1}$ nodes at
depth $h$. Times all nodes with $t-1$ except for root,
the total number of keys satisifes the following inequality.

\be
\begin{array}{rl}
n & \geq 1 + (t-1)(2 + 2t + 2t^2 + ... + 2t^{h-1}) \\
  & = 1 + 2(t-1) \displaystyle \sum_{k=0}^{h-1} t^k \\
  & = 1 + 2(t-1) \displaystyle \frac{t^h-1}{t-1} \\
  & = 2t^h - 1
\end{array}
\ee

Thus we have the inequality between the height and the number
of keys.

\be
h \leq \log_t \frac{n+1}{2}
\ee

This is the reason why B-tree is balanced. The simplest B-tree
is so called 2-3-4 tree, where $t=2$, that every node except
root contains 2 or 3 or 4 keys. red-black tree can be mapped
to 2-3-4 tree essentially.

The following Python code shows example B-tree definition.
It explicitly pass $t$ when create a node.

\lstset{language=Python}
\begin{lstlisting}
class BTree:
    def __init__(self, t):
        self.t = t
        self.keys = []
        self.children = []
\end{lstlisting}

B-tree nodes commonly have satellite data as well. We ignore
satellite data for illustration purpose.

In this chapter, we will firstly introduce how to generate B-tree by insertion.
Two different methods will be explained. One is the classic method
as in \cite{CLRS}, that we split the node before insertion if it's full;
the other is the modify-fix approach which is quite similar to the
red-black tree solution \cite{okasaki-rbtree} \cite{wiki-b-tree}.
We will next explain how to delete
key from B-tree and how to look up a key.


% ================================================================
%                 Insertion
% ================================================================
\section{Insertion}
\label{btree-insertion}

B-tree can be created by inserting
keys repeatedly. The basic idea is similar to the binary
search tree. When insert key $x$, from the tree root, we examine all the
keys in the node to find a position where all the keys on the left are
less than $x$, while all the keys on the right are greater than $x$.
If the current node is a leaf node, and it is not full (there are
less then $2t-1$ keys in this node), $x$ will be insert at this position.
Otherwise, the position points to a child node.
We need recursively insert $x$ to it.

\begin{figure}[htbp]
  \centering
  \subfloat[Insert key 22 to the 2-3-4 tree. $22 > 20$, go to the right child; $22 < 26$ go to the first child.]{\includegraphics[scale=0.5]{img/btree-insert-simple1.ps}} \\
  \subfloat[$21 < 22 < 25$, and the leaf isn't full.]{\includegraphics[scale=0.5]{img/btree-insert-simple2.ps}}
  \caption{Insertion is similar to binary search tree.} \label{fig:btree-insert-simple}
\end{figure}

Figure \ref{fig:btree-insert-simple} shows one example. The B-tree illustrated
is 2-3-4 tree. When insert key $x=22$, because it's greater than the root,
the right child contains key 26, 38, 45 is examined next; Since $22 < 26$,
the first child contains key 21 and 25 are examined. This is a leaf
node, and it is not full, key 22 is inserted to this node.

However, if there are $2t-1$ keys in the leaf, the new key $x$ can't
be inserted, because this node is 'full'. When try to insert key 18
to the above example B-tree will meet this problem. There are 2 methods to
solve it.

%=========================================================================
%       Splitting
%=========================================================================
\subsection{Splitting}
\label{split}

\subsubsection{Split before insertion}

If the node is full, one method to solve the problem is to
split to node before insertion.

For a node with $at-1$ keys, it can be divided into 3 parts as shown in
Figure \ref{fig:node-split}. the left part contains the first $t-1$ keys
and $t$ children. The right part contains the rest $t-1$ keys
and $t$ children. Both left part and right part are valid B-tree
nodes. the middle part is the $t$-th key. We can push it up
to the parent node (if the current node is root, then the this key,
with the two children will be the new root).

\begin{figure}[htbp]
  \centering
  \subfloat[Before split]{\includegraphics[scale=0.5]{img/split-node-before.ps}} \\
  \subfloat[After split]{\includegraphics[scale=0.5]{img/split-node-after.ps}}
  \caption{Split node}
  \label{fig:node-split}
\end{figure}

For node $x$, denote $\mathbb{K}(x)$
as keys, $\mathbb{C}(x)$ as children. The $i$-th key as $K_i(x)$, the $j$-th child
as $C_j(x)$. Below algorithm describes how to split the $i$-th child for a given node.

\begin{algorithmic}[1]
\Procedure{Split-Child}{$node, i$}
  \State $x \gets C_i(node)$
  \State $y \gets$ \Call{CREATE-NODE}{}
  \State \Call{Insert}{$\mathbb{K}(node), i, K_t(x)$}
  \State \Call{Insert}{$\mathbb{C}(node), i + 1, y$}
  \State $\mathbb{K}(y) \gets \{K_{t+1}(x), K_{t+2}(x), ..., K_{2t-1}(x)\}$
  \State $\mathbb{K}(x) \gets \{K_1(x), K_2(x), ..., K_{t-1}(x)\}$
  \If{$y$ is not leaf}
    \State $\mathbb{C}(y) \gets \{C_{t+1}(x), C_{t+2}(x), ..., C_{2t}(x)\}$
    \State $\mathbb{C}(x) \gets \{C_1(x), C_2(x), ..., C_t(x)\}$
  \EndIf
\EndProcedure
\end{algorithmic}

The following example Python program implements this child spliting algorithm.

\lstset{language=Python}
\begin{lstlisting}
def split_child(node, i):
    t = node.t
    x = node.children[i]
    y = BTree(t)
    node.keys.insert(i, x.keys[t-1])
    node.children.insert(i+1, y)
    y.keys = x.keys[t:]
    x.keys = x.keys[:t-1]
    if not is_leaf(x):
        y.children = x.children[t:]
        x.children = x.children[:t]
\end{lstlisting}

Where function \texttt{is\_leaf} test if a node is leaf.

\lstset{language=Python}
\begin{lstlisting}
def is_leaf(t):
    return t.children == []
\end{lstlisting}

%=========================================================================
%       Split before insertion
%=========================================================================

After splitting, a key is pushed up to its parent node.
It is quite possible that the parent node has already been full.
And this pushing violates the B-tree property.

In order to solve this problem, we can check from the root along
the path of insertion traversing till the leaf. If there is any node
in this path is full, the splitting is applied. Since the parent
of this node has been examined, it is ensured that there are
less than $2t-1$ keys in the parent. It won't make the
parent full if pushing up one key. This approach only need one single
pass down the tree without any back-tracking.

If the root need splitting, a new node is created as the new root.
There is no keys in this new created root, and the previous root is
set as the only child. After that, splitting is performed top-down.
And we can insert the new key finally.

\begin{algorithmic}[1]
\Function{Insert}{$T, k$}
  \State $r \gets T$
  \If{$r$ is full} \Comment{root is full}
    \State $s \gets$ \Call{CREATE-NODE}{}
    \State $\mathbb{C}(s) \gets \{r\}$
    \State \Call{Split-Child}{$s, 1$}
    \State $r \gets s$
  \EndIf
  \State \Return \Call{Insert-Nonfull}{$r, k$}
\EndFunction
\end{algorithmic}

Where algorithm \textproc{Insert-Nonfull} assumes the node passed in
is not full. If it is a leaf node, the new key is inserted to
the proper position based on the order; Otherwise, the algorithm
finds a proper child node to which the new key will be inserted.
If this child is full, splitting will be performed.

\begin{algorithmic}[1]
\Function{Insert-Nonfull}{$T, k$}
  \If{$T$ is leaf}
    \State $i \gets 1$
    \While{$i \leq |\mathbb{K}(T)| \land k > K_i(T)$}
      \State $i \gets i+1$
    \EndWhile
    \State \Call{Insert}{$\mathbb{K}(T), i, k$}
  \Else
    \State $i \gets |\mathbb{K}(T)|$
    \While{$i>1 \land k < K_i(T)$}
      \State $i \gets i-1$
    \EndWhile
    \If{$C_i(T)$ is full}
      \State \Call{Split-Child}{$T, i$}
      \If{$k > K_i(T)$}
        \State $i \gets i+1$
      \EndIf
    \EndIf
    \State \Call{Insert-Nonfull}{$C_i(T), k$}
  \EndIf
  \State \Return $T$
\EndFunction
\end{algorithmic}

This algorithm is recursive. In B-tree,
the minimum degree $t$ is typically relative to magnetic disk structure.
Even small depth can support huge amount of data
(with $t=10$, maximum to 10 billion data can be stored in a B-tree with height of 10).
The recursion can also be eleminated. This is left as exercise to the reader.

Figure \ref{fig:btree-insert} shows the result of continously inserting
keys G, M, P, X, A, C, D, E, J, K, N, O, R, S, T, U, V, Y, Z to the empty tree.
The first result is the 2-3-4 tree ($t=2$). The second result shows how
it varies when $t=3$.

\begin{figure}[htbp]
  \centering
  \subfloat[2-3-4 tree.]{\includegraphics[scale=0.5]{img/btree-insert-2-3-4.ps}}\\
  \subfloat[$t=3$]{\includegraphics[scale=0.5]{img/btree-insert-3.ps}}
  \caption{Insertion result} \label{fig:btree-insert}
\end{figure}

Below example Python program implements this algorithm.
\lstset{language=Python}
\begin{lstlisting}
def insert(tr, key):
    root = tr
    if is_full(root):
        s = BTree(root.t)
        s.children.insert(0, root)
        split_child(s, 0)
        root = s
    return insert_nonfull(root, key)
\end{lstlisting}

And the insertion to non-full node is implemented as the following.

\begin{lstlisting}
def insert_nonfull(tr, key):
    if is_leaf(tr):
        ordered_insert(tr.keys, key)
    else:
        i = len(tr.keys)
        while i>0 and key < tr.keys[i-1]:
            i = i-1
        if is_full(tr.children[i]):
            split_child(tr, i)
            if key>tr.keys[i]:
                i = i+1
        insert_nonfull(tr.children[i], key)
    return tr
\end{lstlisting}

Where function \texttt{ordered\_insert} is used to insert an element
to an ordered list. Function \texttt{is\_full} tests if a node contains
$2t-1$ keys.

\begin{lstlisting}
def ordered_insert(lst, x):
    i = len(lst)
    lst.append(x)
    while i>0 and lst[i]<lst[i-1]:
        (lst[i-1], lst[i]) = (lst[i], lst[i-1])
        i=i-1

def is_full(node):
    return len(node.keys) >= 2 * node.t - 1
\end{lstlisting}

For the array based collection, append on the tail is much more effective than
insert in other position, because the later takes $O(n)$ time, if the length
of the collection is $n$. The \texttt{ordered\_insert} program firstly appends
 the new element at the end of the existing collection, then iterates from the
last element to the first one, and checks if the current two elements next to each other
are ordered. If not, these two elements will be swapped.

% ================================================================
%               Insert and fix method
% ================================================================

\subsubsection{Insert then fixing}

In functional settings, B-tree insertion can be realized in a way similar to
red-black tree. When insert a key to red-black tree, it is firstly inserted
as in the normal binary search tree, then recursive fixing is performed to
resume the balance of the tree. B tree can be viewed as extension to the
binary search tree, that each node contains multiple keys and children.
We can firstly insert the key without considering if the node is full.
Then perform fixing to satisfy the minimum degree constraint.

\be
insert(T, k) = fix(ins(T, k))
\ee

Function $ins(T, k)$ traverse the B tree $T$ from root to find a proper
position where key $k$ can be inserted. After that, function $fix$ is
applied to resume the B tree properties. Denote B-tree in a form of
$T = (K, C, t)$, where $K$ represents keys, $C$ represents children,
and $t$ is the minimum degree.

Below is the Haskell definition of B-tree.

\lstset{language=Haskell}
\begin{lstlisting}
data BTree a = Node{ keys :: [a]
                   , children :: [BTree a]
                   , degree :: Int} deriving (Eq)
\end{lstlisting}

The insertion function can be provided based on this definition.

\lstset{language=Haskell}
\begin{lstlisting}
insert tr x = fixRoot $ ins tr x
\end{lstlisting} %$

There are two cases when realize $ins(T, k)$ function. If the tree $T$ is
leaf, $k$ is inserted to the keys; Otherwise if $T$ is the branch node, we
need recursively insert $k$ to the proper child.

Figure \ref{fig:recursive-insert} shows the branch case. The
algorithm first locates the position. for certain key $k_i$,
if the new key $k$ to be inserted satisfy $k_{i-1}<k<k_i$,
Then we need recursively insert $k$ to child $c_i$.

This position divides the node into 3 parts, the left part,
the child $c_i$ and the right part.

\begin{figure}[htbp]
  \centering
  \subfloat[Locate the child to insert.]{\includegraphics[scale=0.5]{img/insert-before.ps}} \\
  \subfloat[Recursive insert.]{\includegraphics[scale=0.5]{img/insert-after.ps}}
  \caption{Insert a key to a branch node} \label{fig:recursive-insert}
\end{figure}

\be
ins(T, k) = \left \{
  \begin{array}
  {r@{\quad:\quad}l}
  (K' \cup \{k\} \cup K'', \Phi, t) & C = \Phi, (K', K'') = divide(K, k) \\
  make((K', C_1), ins(c, k), (K'', C_2')) & (C_1, C_2) = split(|K'|, C)
  \end{array}
\right.
\ee

The first clause deals with the leaf case.
Function $divide(K, k)$ devide keys into two parts, all keys in the first
part are not greater than $k$, and all rest keys are not less than $k$.

\[
K = K' \cup K'' \land \forall k' \in K, k'' \in K'' \Rightarrow k' \leq k \leq k''
\]

The second clause handle the branch case.
Function $split(n, C)$ splits children in two parts, $C_1$ and $C_2$.
$C_1$ contains the first $n$ children; and $C_2$ contains the rest.
Among $C_2$, the first child is denoted as $c$, and others are
represented as $C_2'$.

Here the key $k$ need be recursively inserted into child $c$. Function
$make$ takes 3 parameter. The first and the third are pairs of
key and children; the second parameter is a child node. It examine
if a B tree node made from these keys and children violates the
minimum degree constraint and performs fixing if necessary.

\be
make((K', C'), c, (K'', C'')) = \left \{
  \begin{array}
  {r@{\quad:\quad}l}
  fixFull((K', C'), c, (K'', C'')) & full(c) \\
  (K' \cup K'', C' \cup \{c\} \cup C'', t) & otherwise
  \end{array}
\right.
\ee

Where function $full(c)$ tests if the child $c$ is full.
Function $fixFull$ splits the the child $c$, and forms a new B-tree node
with the pushed up key.

\be
fixFull((K', C'), c, (K'', C'')) = (K' \cup \{k\} \cup K'', C' \cup \{c_1, c_2\} \cup C'', t)
\ee

Where $(c_1, k, c_2) = split(c)$. During splitting, the first $t-1$ keys and $t$ childrens
are extract to one new child, the last $t-1$ keys and $t$ children form another child.
The $t$-th key is pushed up.

With all the above functions defined, we can realize $fix(T)$ to complete the functional B-tree
insertion algorithm. It firstly checks if the root contains too
many keys. If it exceeds the limit, splitting will be applied.
The split result will be used to make a new node, so the total
height of the tree increases by one.

\be
fix(T) = \left \{
  \begin{array}
  {r@{\quad:\quad}l}
  c & T = (\Phi, \{c\}, t) \\
  (\{k\}, \{c_1, c_2\}, t) & full(T), (c_1, k, c_2) = split(T) \\
  T & otherwise
  \end{array}
\right.
\ee

The following Haskell example code implements the B-tree insertion.

\lstset{language=Haskell}
\begin{lstlisting}
import qualified Data.List as L

ins (Node ks [] t) x = Node (L.insert x ks) [] t
ins (Node ks cs t) x = make (ks', cs') (ins c x) (ks'', cs'')
    where
      (ks', ks'') = L.partition (<x) ks
      (cs', (c:cs'')) = L.splitAt (length ks') cs

fixRoot (Node [] [tr] _) = tr -- shrink height
fixRoot tr = if full tr then Node [k] [c1, c2] (degree tr)
             else tr
    where
      (c1, k, c2) = split tr

make (ks', cs') c (ks'', cs'')
    | full c = fixFull (ks', cs') c (ks'', cs'')
    | otherwise = Node (ks'++ks'') (cs'++[c]++cs'') (degree c)

fixFull (ks', cs') c (ks'', cs'') = Node (ks'++[k]++ks'')
                                         (cs'++[c1,c2]++cs'') (degree c)
    where
      (c1, k, c2) = split c

full tr = (length $ keys tr) > 2*(degree tr)-1
\end{lstlisting}

Figure \ref{fig:btree-insert-fp} shows the varies of results of building B-trees
by continously inserting keys "GMPXACDEJKNORSTUVYZ".

\begin{figure}[htbp]
  \centering
  \subfloat[Insert result of a 2-3-4 tree.]{\includegraphics[scale=0.5]{img/btree-insert-fp-234.ps}} \\
  \subfloat[Insert result of a B-tree with $t = 3$]{\includegraphics[scale=0.5]{img/btree-insert-fp-3.ps}}
    \caption{Insert then fixing results} \label{fig:btree-insert-fp}
\end{figure}

Compare to the imperative insertion result
as shown in figure \ref{fig:btree-insert-fp}
we can found that thet are different. However, they are all valid
because all B-tree properties
are satisfied.


% ================================================================
%               Deletion
% ================================================================
\section{Deletion}

Deleting a key from
B-tree may violate balance properties. Except the root, a node shouldn't
contain too few keys less than $t-1$, where $t$ is the
minimum degree.

Similar to the approaches for insertion, we can either do some preparation
so that the node from where the key being deleted contains enough
keys; or do some fixing after the deletion if the node has too few keys.


% ================================================================
%               Merge before delete method
% ================================================================
\subsection{Merge before delete method}

We start from the easiest case. If the key $k$ to be deleted
can be located in node $x$, and $x$ is a leaf node,
we can directly remove $k$ from $x$. If $x$ is the root (the only
node of the tree), we needn't worry about there are too few
keys after deletion. This case is named as case 1 later.

In most cases, we start from the root, along a path to locate
where is the node contains $k$. If $k$ can be located in the
internal node $x$, there are three sub cases.

\begin{itemize}
\item Case 2a, If the child $y$ precedes $k$ contains enough keys (more than $t$).
We replace $k$ in node $x$ with $k'$, which is
the predecessor of $k$ in child $y$. And recursively remove $k'$
from $y$.

The predecessor of $k$ can be easily located as the last key of child
$y$.

This is shown in figure \ref{fig:btree-del-case2a}.

\begin{figure}[htbp]
  \centering
    \includegraphics[scale=0.5]{img/btree-del-case2a.eps}
    \caption{Replace and delete from predecessor.} \label{fig:btree-del-case2a}
\end{figure}

\item Case 2b, If $y$ doesn't contain enough keys, while the child $z$
follows $k$ contains more than $t$ keys. We replace $k$ in node $x$
with $k''$, which is the successor of $k$ in child $z$. And recursively
remove $k''$ from $z$.

The successor of $k$ can be easily located as the first key of child $z$.

This sub-case is illustrated in figure \ref{fig:btree-del-case2b}.

\begin{figure}[htbp]
  \centering
    \includegraphics[scale=0.5]{img/btree-del-case2b.eps}
    \caption{Replace and delete from successor.} \label{fig:btree-del-case2b}
\end{figure}

\item Case 2c, Otherwise, if neither $y$, nor $z$ contains enough keys, we
can merge $y$, $k$ and $z$ into one new node, so that this new node
contains $2t-1$ keys. After that, we can then recursively do the removing.

Note that after merge, if the current node doesn't contain any keys,
which means $k$ is the only key in $x$. $y$ and $z$ are the only two
children of $x$. we need shrink the tree height by one.
\end{itemize}

Figure \ref{fig:btree-del-case2c} illustrates this sub-case.

\begin{figure}[htbp]
  \centering
    \includegraphics[scale=0.5]{img/btree-del-case2c.eps}
    \caption{Merge and delete.} \label{fig:btree-del-case2c}
\end{figure}

the last case states that, if $k$ can't be located in node $x$, the algorithm
need find a child node $c_i$ in $x$, so that the sub-tree $c_i$
contains $k$. Before the deletion is recursively applied in $c_i$, we
need make sure that there are at least $t$ keys in $c_i$. If there are
not enough keys, the following adjustment is performed.

\begin{itemize}
\item Case 3a, We check the two sibling of $c_i$, which are $c_{i-1}$ and $c_{i+1}$.
If either one contains enough keys (at least $t$ keys), we move
one key from $x$ down to $c_i$, and move one key from the sibling up to
$x$. Also we need move the relative child from the sibling to $c_i$.

This operation makes $c_i$ contains enough keys for deletion. we can
next try to delete $k$ from $c_i$ recursively.

Figure \ref{fig:btree-del-case3a} illustrates this case.

\begin{figure}[htbp]
  \centering
    \includegraphics[scale=0.5]{img/btree-del-case3a.eps}
    \caption{Borrow from the right sibling.}
    \label{fig:btree-del-case3a}
\end{figure}

\item Case 3b, In case neither one of the two siblings contains enough keys, we then
merge $c_i$, a key from $x$, and either one of the sibling into a new
node. Then do the deletion on this new node.
\end{itemize}

Figure \ref{fig:btree-del-case3b} shows this case.

\begin{figure}[htbp]
  \centering
    \includegraphics[scale=0.5]{img/btree-del-case3b.eps}
    \caption{Merge $c_i$, $k$, and $c_{i+1}$ to a new node.}
    \label{fig:btree-del-case3b}
\end{figure}

Before define the B-tree delete algorithm, we need provide some auxiliary
functions. Function \textproc{Can-Del} tests if a node contains enough keys
for deletion.

\begin{algorithmic}[1]
\Function{Can-Del}{$T$}
  \State \Return $|\mathbb{K}(T)| \ge t$
\EndFunction
\end{algorithmic}

Procedure \textproc{Merge-Children}($T, i$) merges child $C_i(T)$, key $K_i(T)$,
and child $C_{i+1}(T)$ into one big node.

\begin{algorithmic}[1]
\Procedure{Merge-Children}{$T, i$} \Comment{Merge $C_i(T)$, $K_i(T)$, and $C_{i+1}(T)$}
  \State $x \gets C_i(T)$
  \State $y \gets C_{i+1}(T)$
  \State $\mathbb{K}(x) \gets \mathbb{K}(x) \cup \{K_i(T)\} \cup \mathbb{K}(y)$
  \State $\mathbb{C}(x) \gets \mathbb{C}(x) \cup \mathbb{C}(y)$
  \State \Call{Remove-At}{$\mathbb{K}(T), i$}
  \State \Call{Remove-At}{$\mathbb{C}(T), i+1$}
\EndProcedure
\end{algorithmic}

Procedure \textproc{Merge-Children} merges the $i$-th child, the $i$-th key,
and $i+1$-th child of node $T$ into a new child, and remove the
$i$-th key and $i+1$-th child from $T$ after merging.

With these functions defined, the B-tree deletion algorithm can be given by
realizing the above 3 cases.

\begin{algorithmic}[1]
\Function{Delete}{$T, k$}
  \State $i \gets 1$
  \While{$i \leq |\mathbb{K}(T)|$}
    \If{$k = K_i(T)$}
      \If{$T$ is leaf} \Comment{case 1}
        \State \Call{Remove}{$\mathbb{K}(T), k$}
      \Else \Comment{case 2}
        \If{\Call{Can-Del}{$C_i(T)$}} \Comment{case 2a}
          \State $K_i(T) \gets$ \Call{Last-Key}{$C_i(T)$}
          \State \Call{Delete}{$C_i(T), K_i(T)$}
        \ElsIf{\Call{Can-Del}{$C_{i+1}(T)$}} \Comment{case 2b}
          \State $K_i(T) \gets$ \Call{First-Key}{$C_{i+1}(T)$}
          \State \Call{Delete}{$C_{i+1}(T), K_i(T)$}
        \Else \Comment{case 2c}
          \State \Call{Merge-Children}{$T, i$}
          \State \Call{Delete}{$C_i(T), k$}
          \If{$\mathbb{K}(T) = NIL$}
            \State $T \gets C_i(T)$ \Comment{Shrinks height}
          \EndIf
        \EndIf
      \EndIf
      \State \Return $T$
    \ElsIf{$k < K_i(T)$}
      \State Break
    \Else
      \State $i \gets i+1$
    \EndIf
  \EndWhile
  \Statex
  \If{$T$ is leaf}
    \State \Return $T$ \Comment{$k$ doesn't exist in $T$.}
  \EndIf
  \If{$\lnot$ \Call{Can-Del}{$C_i(T)$}}  \Comment{case 3}
    \If{$i>1 \land$ \Call{Can-Del}{$C_{i-1}(T)$}} \Comment{case 3a: left sibling}
      \State \Call{Insert}{$\mathbb{K}(C_i(T)), K_{i-1}(T)$}
      \State $K_{i-1}(T) \gets$ \Call{Pop-Back}{$\mathbb{K}(C_{i-1}(T))$}
      \If{$C_i(T)$ isn't leaf}
        \State $c \gets$ \Call{Pop-Back}{$\mathbb{C}(C_{i-1}(T))$}
        \State \Call{Insert}{$\mathbb{C}(C_i(T)), c$}
      \EndIf
    \ElsIf{$i \leq |\mathbb{C}(T)| \land$ \Call{Can-Del}{$C_{i_1}(T)$}} \Comment{case 3a: right sibling}
      \State \Call{Append}{$\mathbb{K}(C_i(T)), K_i(T)$}
      \State $K_i(T) \gets$ \Call{Pop-Front}{$\mathbb{K}(C_{i+1}(T))$}
      \If{$C_i(T)$ isn't leaf}
        \State $c \gets$ \Call{Pop-Front}{$\mathbb{C}(C_{i+1}(T))$}
        \State \Call{Append}{$\mathbb{C}(C_i(T)), c$}
      \EndIf
    \Else \Comment{case 3b}
      \If{$i>1$}
        \State \Call{Merge-Children}{$T, i-1$}
      \Else
        \State \Call{Merge-Children}{$T, i$}
      \EndIf
    \EndIf
  \EndIf
  \State \Call{Delete}{$C_i(T), k$} \Comment {recursive delete}
  \If{$\mathbb{K}(T) = NIL$} \Comment {Shrinks height}
    \State $T \gets C_1(T)$
  \EndIf
  \State \Return $T$
\EndFunction
\end{algorithmic}

Figure \ref{fig:result-del1}, \ref{fig:result-del2}, and \ref{fig:result-del3}
show the deleting process step by step. The nodes modified are shaded.

\begin{figure}[htbp]
  \centering
  \subfloat[A B-tree before deleting.]{\includegraphics[scale=0.5]{img/btree-del-before.ps}} \\
  \subfloat[After delete key 'F', case 1.]{\includegraphics[scale=0.5]{img/btree-del-F.ps}}
  \caption{Result of B-tree deleting (1).} \label{fig:result-del1}
\end{figure}

\begin{figure}[htbp]
  \centering
  \subfloat[After delete key 'M', case 2a.]{\includegraphics[scale=0.5]{img/btree-del-M.ps}} \\
  \subfloat[After delete key 'G', case 2c.]{\includegraphics[scale=0.5]{img/btree-del-G.ps}}
  \caption{Result of B-tree deleting program (2)} \label{fig:result-del2}
\end{figure}

\begin{figure}[htbp]
  \centering
  \subfloat[After delete key 'D', case 3b, and height is shrunk.]{\includegraphics[scale=0.5]{img/btree-del-D.ps}} \\
  \subfloat[After delete key 'B', case 3a, borrow from right sibling.]{\includegraphics[scale=0.5]{img/btree-del-B.ps}} \\
  \subfloat[After delete key 'U', case 3a, borrow from left sibling.]{\includegraphics[scale=0.5]{img/btree-del-U.ps}}
  \caption{Result of B-tree deleting program (3)} \label{fig:result-del3}
\end{figure}

The following example Python program implements the B-tree deletion algorithm.

TODO: Change the member function to global function.
\lstset{language=Python}
\begin{lstlisting}
class BTreeNode:
    def merge_children(self, i):
        self.children[i].keys += [self.keys[i]]+self.children[i+1].keys
        self.children[i].children += self.children[i+1].children
        self.keys.pop(i)
        self.children.pop(i+1)

    def replace_key(self, i, key):
        self.keys[i] = key
        return key

    def can_remove(self):
        return len(self.keys) >= self.t
\end{lstlisting}

\begin{lstlisting}
def B_tree_delete(tr, key):
    i = len(tr.keys)
    while i>0:
        if key == tr.keys[i-1]:
            if tr.leaf:  # case 1 in CLRS
                tr.keys.remove(key)
                #disk_write(tr)
            else: # case 2 in CLRS
                if tr.children[i-1].can_remove(): # case 2a
                    key = tr.replace_key(i-1, tr.children[i-1].keys[-1])
                    B_tree_delete(tr.children[i-1], key)
                elif tr.children[i].can_remove(): # case 2b
                    key = tr.replace_key(i-1, tr.children[i].keys[0])
                    B_tree_delete(tr.children[i], key)
                else: # case 2c
                    tr.merge_children(i-1)
                    B_tree_delete(tr.children[i-1], key)
                    if tr.keys==[]: # tree shrinks in height
                        tr = tr.children[i-1]
            return tr
        elif key > tr.keys[i-1]:
            break
        else:
            i = i-1
    # case 3
    if tr.leaf:
        return tr #key doesn't exist at all
    if not tr.children[i].can_remove():
        if i>0 and tr.children[i-1].can_remove(): #left sibling
            tr.children[i].keys.insert(0, tr.keys[i-1])
            tr.keys[i-1] = tr.children[i-1].keys.pop()
            if not tr.children[i].leaf:
                tr.children[i].children.insert(0, tr.children[i-1].children.pop())
        elif i<len(tr.children) and tr.children[i+1].can_remove(): #right sibling
            tr.children[i].keys.append(tr.keys[i])
            tr.keys[i]=tr.children[i+1].keys.pop(0)
            if not tr.children[i].leaf:
                tr.children[i].children.append(tr.children[i+1].children.pop(0))
        else: # case 3b
            if i>0:
                tr.merge_children(i-1)
            else:
                tr.merge_children(i)
    B_tree_delete(tr.children[i], key)
    if tr.keys==[]: # tree shrinks in height
        tr = tr.children[0]
    return tr
\end{lstlisting}

% ================================================================
%               Delete and fix method
% ================================================================

\subsection{Delete and fix method}

From previous sub-sections, we see how complex is the deletion algorithm,
There are several cases, and in each case, there are sub cases to deal.

Another approach to design the deleting algorithm is a kind of delete-then-fix
way. It is similar to the insert-then-fix strategy.

When we need delete a key from a B-tree, we firstly try to locate
which node this key is contained. This will be a traverse process
from the root node towards leaves. We start from root node, If the
key doesn't exist in the node, we'll traverse deeper and deeper
until we rich a node.

If this node is a leaf node, we can remove the key directly, and then
examine if the deletion makes the node contains too few keys to
maintain the B-tree balance properties.

If it is a branch node, removing the key will break the node into
two parts, we need merge them together. The merging is a recursive
process which can be shown in figure \ref{fig:del-fp-merge}.

\begin{figure}[htbp]
    \begin{center}
      \includegraphics[scale=0.5]{img/btree-del-fp-merge.eps}
      \caption{Delete a key from a branch node. Removing $k_i$ breaks
the node into 2 parts, left part and right part. Merging these 2 parts
is a recursive process. When the two parts are leaves, the merging
terminates.} \label{fig:del-fp-merge}
    \end{center}
\end{figure}

When do merging, if the two nodes are not leaves, we merge the keys
together, and recursively merge the last child of the left part
and the first child of the right part as one new child node. Otherwise,
if they are leaves, we merely put all keys together.

Till now, we do the deleting in straightforward way. However, deleting
will decrease the number of keys of a node, and it may result in
violating the B-tree balance properties. The solution is to perform a
fixing along the path we traversed from root.

\begin{figure}[htbp]
    \begin{center}
      \includegraphics[scale=0.5]{img/btree-del-fp-make.eps}
      \caption{Denote $c_i'$ as the result of recursively deleting
key $k$, from child $c_i$, we should do fixing when making the
left part, $c_i'$ and right part together to a new node.} \label{fig:del-fp-make}
    \end{center}
\end{figure}

When we do recursive deletion, the branch node is broken into 3 parts.
The left part contains all keys less than $k$, say $k_1, k_2, ..., k_{i-1}$,
and children $c_1, c_2, ..., c_{i-1}$, the right part contains all keys
greater than $k$, say $k_i, k_{i+1}, ..., k_{n+1}$, and children
$c_{i+1}, c_{i+2}, ..., c_{n+1}$, the child $c_i$ which recursive deleting
applied becomes $c_i'$. We need make these 3 parts to a new node
as shown in figure \ref{fig:del-fp-make}.

At this time point, we can examine if $c_i'$ contains enough keys,
it the number of keys is to less (less than $t-1$, but not $t$ in
contrast to merge and delete approach), we can either borrow a key-child
pair from left part or right part, and do a inverse operation of
splitting. Figure \ref{fig:del-fp-fixlow} shows an example of borrow from left part.

\begin{figure}[htbp]
    \begin{center}
      \includegraphics[scale=0.5]{img/btree-del-fp-fixlow.eps}
      \caption{Borrow a key-child pair from left part and
un-split to a new child.} \label{fig:del-fp-fixlow}
    \end{center}
\end{figure}

In case both left part and right part is empty, we can simply
push $c_i'$ up.

\subsubsection{Delete and fix algorithm implemented functionally}

By summarize all above analysis, we can draft the delete and fix
algorithm.

\begin{algorithmic}[1]
\Function{B-TREE-DELETE'}{$T, k$}
  \State \Return $FIX-ROOT(DEL(T, k))$
\EndFunction

\Function{DEL}{$T, k$}
  \If{$CHILDREN(T) = NIL$}\Comment{leaf node}
    \State $DELETE(KEYS(T), k)$
    \State \Return $T$
  \Else \Comment{branch node}
    \State $n \gets LENGTH(KEYS(T))$
    \State $i \gets LOWER-BOUND(KEYS(T), k)$
    \If{$KEYS(T)[i] = k$}
      \State $k_l \gets KEYS(T)[1, ..., i-1]$
      \State $k_r \gets KEYS(T)[i+1, ..., n]$
      \State $c_l \gets CHILDREN(T)[1, ..., i]$
      \State $c_r \gets CHILDREN(T)[i+1, ..., n+1]$
      \State \Return $MERGE(CREATE-B-TREE(k_l, c_l), CREATE-B-TREE(k_r, c_r))$
    \Else
      \State $k_l \gets KEYS(T)[1, ..., i-1]$
      \State $k_r \gets KEYS(T)[i, ..., n]$
      \State $c \gets CHILDREN(T)[i]$
      \State $c_l \gets CHILDREN(T)[1, ..., i-1]$
      \State $c_r \gets CHILDREN(T)[i+1, ..., n+1]$
      \State \Return $MAKE((k_l, c_l), c, (k_r, c_r))$
    \EndIf
  \EndIf
\EndFunction
\end{algorithmic}

The main delete function will call an internal $DEL$ function to
performs the work, after that, it will apply $FIX-ROOT$ to check
if need to shrink the tree height. So the $FIX-ROOT$ function we
defined in insertion section should be updated as the following.

\begin{algorithmic}[1]
\Function{FIX-ROOT}{$T$}
  \If{$KEYS(T) = NIL$} \Comment{Single child, shrink the height}
    \State $T \gets CHILDREN(T)[1]$
  \ElsIf{$FULL?(T)$}
    \State $T \gets B-TREE-SPLIT(T)$
  \EndIf
  \State \Return $T$
\EndFunction
\end{algorithmic}

For the recursive merging, the algorithm is given as below.
The left part and right part are passed as parameters. If
they are leaves, we just put all keys together. Otherwise,
we recursively merge the last child of left and the first child
of right to a new child, and make this new merged child and
the other two parts it breaks into a new node.

\begin{algorithmic}[1]
\Function{MERGE}{$L, R$}
  \If{$L, R$ are leaves}
    \State $T \gets CREATE-NEW-NODE()$
    \State $KEYS(T) \gets KEYS(L)+KEYS(R)$
    \State \Return $T$
  \Else
    \State $m gets LENGTH(KEYS(L))$
    \State $n gets LENGTH(KEYS(R))$
    \State $k_l \gets KEYS(L)$
    \State $k_r \gets KEYS(R)$
    \State $c_l \gets CHILDREN(L)[1, ..., m-1]$
    \State $c_r \gets CHILDREN(R)[2, ..., n]$
    \State $c \gets MERGE(CHILDREN(L)[m], CHILDREN(R)[1])$
    \State \Return $MAKE-B-TREE((k_l, c_l), c, (k_r, c_r))$
  \EndIf
\EndFunction
\end{algorithmic}

In order to make the three parts, the left $L$, the right $R$ and
the child $c_i'$ into a node, we need examine if $c_i$ contains
enough keys, together with the process of ensure it contains not too
much keys during insertion, we updated the algorithm like the following.

\begin{algorithmic}[1]
\Function{MAKE-B-TREE}{$L, C, R$}
  \If{$FULL?(C)$}
    \State \Return $FIX-FULL(L, C, R)$
  \ElsIf{$LOW?(C)$}
    \State \Return $FIX-LOW(L, C, R)$
  \Else
    \State $T \leftarrow CREATE-NEW-NODE()$
    \State $KEYS(T) \leftarrow KEYS(L) + KEYS(R)$
    \State $CHILDREN(T) \leftarrow CHILDREN(L)+[C]+CHILDREN(R)$
    \State \Return $T$
  \EndIf
\EndFunction
\end{algorithmic}

Where $FIX-LOW$ is defined as the following. In case the left part
isn't empty, it will borrow a key-child pair from the left, and
do un-split to make the child contains enough keys, then recursively
call $MAKE-B-TREE$; If the left part is empty, it will try to borrow
key-child pair from the right part, and if both sides are empty, it
will returns the child node as result, so that the height shrinks.

\begin{algorithmic}[1]
\Function{FIX-LOW}{$L, C, R$}
  \State $k_l, c_l \gets L$
  \State $k_r, c_r \gets R$
  \State $m \gets LENGTH(k_l)$
  \State $n \gets LENGTH(k_r)$
  \If{$k_l \ne NIL$}
    \State $k_l' \gets k_l[1, ..., m-1]$
    \State $c_l' \gets c_l[1, ..., m-1]$
    \State $C' \gets UN-SPLIT(c_l[m], k_l[m], C)$
    \State \Return $MAKE-B-TREE((k_l', c_l'), C', R)$
  \ElsIf{$k_r \ne NIL$}
    \State $k_r' \gets k_r[2, ..., n]$
    \State $c_r' \gets c_r[2, ..., n]$
    \State $C' \gets UN-SPLIT(C, k_r[1], c_r[1])$
    \State \Return $MAKE-B-TREE(L, C', (k_r', c_r'))$
  \Else
    \State \Return $C$
  \EndIf
\EndFunction
\end{algorithmic}

Function $UN-SPLIT$ defines as the inverses operation of splitting.

\begin{algorithmic}[1]
\Function{UN-SPLIT}{$L, k, R$}
  \State $T \gets CREATE-B-TREE-NODE()$
  \State $KEYS(T) \gets KEYS(L)+[k]+KEYS(R)$
  \State $CHILDREN(T) \gets CHILDREN(L)+CHILDREN(R)$
  \State \Return $T$
\EndFunction
\end{algorithmic}

\subsubsection{Delete and fix algorithm implemented in Haskell}
Based on the analysis of delete-then-fixing approach, a Haskell
program can be provided accordingly.

The core deleting function is simple, it just call an internal
removing function, then examine the root node to see if the
height of the tree can be shrunk.

\lstset{language=Haskell}
\begin{lstlisting}
import qualified Data.List as L

delete :: (Ord a) => BTree a -> a -> BTree a
delete tr x = fixRoot $ del tr x

del:: (Ord a) => BTree a -> a -> BTree a
del (Node ks [] t) x = Node (L.delete x ks) [] t
del (Node ks cs t) x =
    case L.elemIndex x ks of
      Just i -> merge (Node (take i ks) (take (i+1) cs) t)
                      (Node (drop (i+1) ks) (drop (i+1) cs) t)
      Nothing -> make (ks', cs') (del c x) (ks'', cs'')
    where
      (ks', ks'') = L.partition (<x) ks
      (cs', (c:cs'')) = L.splitAt (length ks') cs
\end{lstlisting} %$

Let's focus on the `del' function, if try to delete a key from
a leaf node, it just calls delete function defined in Data.List
library. If the key doesn't exist at all, the pre-defined
delete function will simply return the list without any modification.
For the case of deleting a key from a branch node, it will first examine
if the key can be located in this node, and apply recursive merge
after remove this key. Otherwise, it will locate the proper child
and do recursive delete-then-fixing on this child.

Note that `partition' and 'splitAt' functions defined in Data.List
can help to split the key and children list at the position that
all elements on the left is less than the key while the right part
are greater than the key.

The recursive merge program has two patterns, merge two leaves and
merge two branches. It is given as the following.

\begin{lstlisting}
merge :: BTree a -> BTree a -> BTree a
merge (Node ks [] t) (Node ks' [] _) = Node (ks++ks') [] t
merge (Node ks cs t) (Node ks' cs' _) = make (ks, init cs)
                                             (merge (last cs) (head cs'))
                                             (ks', tail cs')
\end{lstlisting}

Where `init', `last', 'tail' functions are used to manipulate
list which are defined in Haskell prelude.

The fixing part of delete-then-fixing is defined inside 'make'
function.

\begin{lstlisting}
make :: ([a], [BTree a]) -> BTree a -> ([a], [BTree a]) -> BTree a
make (ks', cs') c (ks'', cs'')
    | full c = fixFull (ks', cs') c (ks'', cs'')
    | low c  = fixLow  (ks', cs') c (ks'', cs'')
    | otherwise = Node (ks'++ks'') (cs'++[c]++cs'') (degree c)
\end{lstlisting}

Where function `low' is used to test if a node contains too few keys.

\begin{lstlisting}
low::BTree a -> Bool
low tr = (length $ keys tr) < (degree tr)-1
\end{lstlisting} %$

The real fixing is implemented by try to borrow keys either from
left sibling or right sibling as the following.

\begin{lstlisting}
fixLow :: ([a], [BTree a]) -> BTree a -> ([a], [BTree a]) -> BTree a
fixLow (ks'@(_:_), cs') c (ks'', cs'') = make (init ks', init cs')
                                              (unsplit (last cs') (last ks') c)
                                              (ks'', cs'')
fixLow (ks', cs') c (ks''@(_:_), cs'') = make (ks', cs')
                                              (unsplit c (head ks'') (head cs''))
                                              (tail ks'', tail cs'')
fixLow _ c _ = c
\end{lstlisting}

Note that by using `x@(\_:\_)' like pattern can help to ensure 'x' is
not empty. Here function `unsplit' is used which will do inverse
splitting operation like below.

\begin{lstlisting}
unsplit :: BTree a -> a -> BTree a -> BTree a
unsplit c1 k c2 = Node ((keys c1)++[k]++(keys c2))
                       ((children c1)++(children c2)) (degree c1)
\end{lstlisting}

In order to verify the Haskell program, we can provide some simple
test cases.

\begin{lstlisting}
import Control.Monad (foldM_, mapM_)

testDelete = foldM_ delShow (listToBTree "GMPXACDEJKNORSTUVYZBFHIQW" 3) "EGAMU"
  where
    delShow tr x = do
      let tr' = delete tr x
      putStrLn $ "delete "++(show x)
      putStrLn $ toString tr'
      return tr'
\end{lstlisting}

Where function `listToBTree' and `toString' are defined in previous section when we
explain insertion algorithm.

Run this function will generate the following result.

\begin{verbatim}
delete 'E'
((('A', 'B'), 'C', ('D', 'F'), 'G', ('H', 'I', 'J', 'K')), 'M',
(('N', 'O'), 'P', ('Q', 'R', 'S'), 'T', ('U', 'V'), 'W', ('X', 'Y', 'Z')))
delete 'G'
((('A', 'B'), 'C', ('D', 'F'), 'H', ('I', 'J', 'K')), 'M',
(('N', 'O'), 'P', ('Q', 'R', 'S'), 'T', ('U', 'V'), 'W', ('X', 'Y', 'Z')))
delete 'A'
(('B', 'C', 'D', 'F'), 'H', ('I', 'J', 'K'), 'M', ('N', 'O'),
'P', ('Q', 'R', 'S'), 'T', ('U', 'V'), 'W', ('X', 'Y', 'Z'))
delete 'M'
(('B', 'C', 'D', 'F'), 'H', ('I', 'J', 'K', 'N', 'O'), 'P',
('Q', 'R', 'S'), 'T', ('U', 'V'), 'W', ('X', 'Y', 'Z'))
delete 'U'
(('B', 'C', 'D', 'F'), 'H', ('I', 'J', 'K', 'N', 'O'), 'P',
('Q', 'R', 'S', 'T', 'V'), 'W', ('X', 'Y', 'Z'))
\end{verbatim}

If we try to delete the same key from the same B-tree as in merge and fixing
approach, we can found that the result is different by using delete-then-fixing
methods. Although the results are not as same as each other, both satisfy
the B-tree properties, so they are all correct.

\begin{figure}[htbp]
    \begin{center}
      \includegraphics[scale=0.5]{img/btree-del-fp-before.ps}

      a. A B-tree before performing deleting;

      \includegraphics[scale=0.5]{img/btree-del-fp-E.ps}

      b. After delete key 'E'
      \caption{Result of delete-then-fixing (1)} \label{fig:result-del-fp1}
    \end{center}
\end{figure}

\begin{figure}[htbp]
    \begin{center}
      \includegraphics[scale=0.5]{img/btree-del-fp-G.ps}

      c. After delete key 'G';

      \includegraphics[scale=0.5]{img/btree-del-fp-A.ps}

      d. After delete key 'A';
      \caption{Result of delete-then-fixing (2)} \label{fig:result-del-fp2}
    \end{center}
\end{figure}

\begin{figure}[htbp]
    \begin{center}
      \includegraphics[scale=0.5]{img/btree-del-fp-M.ps}

      e. After delete key 'M';

      \includegraphics[scale=0.5]{img/btree-del-fp-U.ps}

      f. After delete key 'U';
      \caption{Result of delete-then-fixing (3)} \label{fig:result-del-fp3}
    \end{center}
\end{figure}


\subsubsection{Delete and fix algorithm implemented in Scheme/Lisp}
In order to implement delete program in Scheme/Lisp, we provide an
extra function to test if a node contains too few keys after deletion.

\lstset{language=lisp}
\begin{lstlisting}
(define (low? tr t) ;; t: minimum degree
  (< (length (keys tr))
     (- t 1)))
\end{lstlisting}

And some general purpose list manipulation functions are defined.

\begin{lstlisting}
(define (rest lst k)
  (list-tail lst (- (length lst) k)))

(define (except-rest lst k)
  (list-head lst (- (length lst) k)))

(define (first lst)
  (if (null? lst) '() (car lst)))

(define (last lst)
  (if (null? lst) '() (car (last-pair lst))))

(define (inits lst)
  (if (null? lst) '() (except-last-pair lst)))
\end{lstlisting}

Function `rest' can extract the last $k$ elements from a list, while
`except-rest' used to extract all except the last $k$ elements.
`first' can be treat as a safe `car', it will return empty list but not
throw exception when the list is empty. Function `last' returns the
last element of a list, and if the list is empty, it will return
empty result. Function `inits' returns all excluding the last element.

And a inversion operation of splitting is provided.

\begin{lstlisting}
(define (un-split lst)
  (let ((c1 (car lst))
	(k (cadr lst))
	(c2 (caddr lst)))
    (append c1 (list k) c2)))
\end{lstlising}

The main function of deletion is defined as the following.

\begin{lstlisting}
(define (btree-delete tr x t)
  (define (del tr x)
    (if (leaf? tr)
	(delete x tr)
	(let* ((res (partition-by tr x))
	       (left (car res))
	       (c (cadr res))
	       (right (caddr res)))
	  (if (equal? (first right) x)
	      (merge-btree (append left (list c)) (cdr right) t)
	      (make-btree left (del c x) right t)))))
  (fix-root (del tr x) t))
\end{lstlisting}

It is implemented in a similar way as the insertion, call an internal
defined `del' function then apply fixing process on it. In the internal
deletion fiction, if the B-tree is a leaf node, the standard list
deleting function defined in standard library is applied. If it is
a branch node, we call the `partition-by' function defined previously.
This function will divide the node into 3 parts, all children and
keys less than $x$ as the left part, a child node next, all keys not less
than (greater than or equal to) $x$ and children s the right part.

If the first key in right part is equal to $x$, it means $x$ can
be located in this node, we remove $x$ from right and then
call `merge-btree' to merge left+c, right-$x$ to one new node.

\begin{lstlisting}
(define (merge-btree tr1 tr2 t)
  (if (leaf? tr1)
      (append tr1 tr2)
      (make-btree (inits tr1)
		  (merge-btree (last tr1) (car tr2) t)
		  (cdr tr2)
		  t)))
\end{lstlisting}

Otherwise, $x$ may be located in c, so we need recursively try
to delete $x$ from c.

Function `fix-root' is updated to handle the cases for deletion
as below.

\begin{lstlisting}
(define (fix-root tr t)
  (cond ((null? tr) '()) ;; empty tree
	((full? tr t) (split tr t))
	((null? (keys tr)) (car tr)) ;; shrink height
	(else tr)))
\end{lstlisting}

We added one case to handle if a node contains too few keys
after deleting in `make-btree'.

\begin{lstlisting}
(define (make-btree l c r t)
  (cond ((full? c t) (fix-full l c r t))
	((low? c t) (fix-low l c r t))
	(else (append l (cons c r)))))
\end{lstlisting}

Where `fix-low' is defined to try to borrow a key and a child
either from left sibling or right sibling.

\begin{lstlisting}
(define (fix-low l c r t)
  (cond ((not (null? (keys l)))
	 (make-btree (except-rest l 2)
		     (un-split (append (rest l 2) (list c)))
		     r t))
	((not (null? (keys r)))
	 (make-btree l
		     (un-split (cons c (list-head r 2)))
		     (list-tail r 2) t))
	(else c)))
\end{lstlisting}

In order to verify the the deleting program, a simple test is
fed to the above defined function.

\begin{lstlisting}
(define (test-delete)
  (define (del-and-show tr x)
    (let ((r (btree-delete tr x 3)))
      (begin (display r) (display "\n") r)))
  (fold-left del-and-show
	     (list->btree (str->slist "GMPXACDEJKNORSTUVYZBFHIQW") 3)
	     (str->slist "EGAMU")))
\end{lstlisting}

Run the test will generate the following result.

\begin{lstlisting}
(((A B) C (D F) G (H I J K)) M ((N O) P (Q R S) T (U V) W (X Y Z)))
(((A B) C (D F) H (I J K)) M ((N O) P (Q R S) T (U V) W (X Y Z)))
((B C D F) H (I J K) M (N O) P (Q R S) T (U V) W (X Y Z))
((B C D F) H (I J K N O) P (Q R S) T (U V) W (X Y Z))
((B C D F) H (I J K N O) P (Q R S T V) W (X Y Z))
\end{lstlisting}

Compare with the output by the Haskell program in previous section,
it can be found they are same.

% ================================================================
%               Searching
% ================================================================
\section{Searching}

Although searching in B-tree can be considered as a generalized
form of tree search which extended from binary search tree, it's good
to mention that in disk access case, instead of just returning the
satellite data corresponding to the key, it's more meaningful to
return the whole node, which contains the key.

\subsection{Imperative search algorithm}

When searching in Binary tree, there are only 2 different directions,
left and right to go further searching, however, in B-tree, we need
extend the search directions to cover the number of children in a
node.

\begin{algorithmic}[1]
\Function{B-TREE-SEARCH}{$T, k$}
  \Loop
    \State $i \gets 1$
    \While{$i \leq LENGTH(KEYS(T))$ and $k > KEYS(T)[i]$}
      \State $k \gets k+1$
    \EndWhile
    \If{$i \leq LENGTH(KEYS(T))$ and $k = KEYS(T)[i]$}
      \State \Return $(T, i)$
    \EndIf
    \If{$T$ is leaf}
      \State \Return $NIL$ \Comment{$k$ doesn't exist at all}
    \Else
      \State $T \gets CHILDREN(T)[i]$
    \EndIf
  \EndLoop
\EndFunction
\end{algorithmic}

When doing search, the program examine each key from the root node by
traverse from the smallest towards the biggest one. in case it find a
matched key, it returns the current node as well as the index of this
keys. Otherwise, if it finds this key satisfying $k_i < k < k_{i+1}$,
The program will update the current node to be examined as child node
$c_{i+1}$. If it
fails to find this key in a leaf node, empty value is returned to
indicate the fail case.

Note that in ``Introduction to Algorithm'', this program is described
with recursion, Here the recursion is eliminated.

\subsubsection*{search program in C++}
In C++ implementation, we can use pair provided in STL library as
the return type.

\lstset{language=C++}
\begin{lstlisting}
template<class T>
std::pair<T*, unsigned int> search(T* t, typename T::key_type k){
  for(;;){
    unsigned int i(0);
    for(; i < t->keys.size() && k > t->keys[i]; ++i);
    if(i < t->keys.size() && k == t->keys[i])
      return std::make_pair(t, i);
    if(t->leaf())
      break;
    t = t->children[i];
  }
  return std::make_pair((T*)0, 0); //not found
}
\end{lstlisting}

And the test cases are given as below.

\begin{lstlisting}
  void test_search(){
    std::cout<<"test search...\n";
    const char* ss[] = {"G", "M", "P", "X", "A", "C", "D", "E", "J", "K", \
                        "N", "O", "R", "S", "T", "U", "V", "Y", "Z"};
    BTree<std::string, 3>* tr = list_to_btree(ss, ss+sizeof(ss)/sizeof(char*),
                                              new BTree<std::string, 3>);
    std::cout<<"\n"<<btree_to_str(tr)<<"\n";
    for(unsigned int i=0; i<sizeof(ss)/sizeof(char*); ++i)
      __test_search(tr, ss[i]);
    __test_search(tr, "W");
    delete tr;
  }

  template<class T>
  void __test_search(T* t, typename T::key_type k){
    std::pair<T*, unsigned int> res = search(t, k);
    if(res.first)
      std::cout<<"found "<<res.first->keys[res.second]<<"\n";
    else
      std::cout<<"not found "<<k<<"\n";
  }
\end{lstlisting}

Run `test\_search' function will generate the following result.

\begin{verbatim}
test search...
((A, C), D, (E, G, J, K), M, (N, O), P, (R, S), T, (U, V, X, Y, Z))
found G
found M
...
found Z
not found W
\end{verbatim}

Here the program can find all keys we inserted.

\subsubsection*{search program in Python}

Change a bit the above algorithm in Python gets the program corresponding
to the pseudo code mentioned in ``Introduction to Algorithm'' textbook.

\lstset{language=Python}
\begin{lstlisting}
def B_tree_search(tr, key):
    for i in range(len(tr.keys)):
        if key<= tr.keys[i]:
            break
    if key == tr.keys[i]:
        return (tr, i)
    if tr.leaf:
        return None
    else:
        if key>tr.keys[-1]:
            i=i+1
        #disk_read
        return B_tree_search(tr.children[i], key)
\end{lstlisting}

There is a minor modification from the original pseudo code.
We uses for-loop to iterate the keys, the the boundary check is done
by compare the last key in the node and adjust the index if
necessary.

Let's feed some simple test cases to this program.

\begin{lstlisting}
   def test_search():
        lst = ["G", "M", "P", "X", "A", "C", "D", "E", "J", "K", \
               "N", "O", "R", "S", "T", "U", "V", "Y", "Z"]
        tr = list_to_B_tree(lst, 3)
        print "test search\n", B_tree_to_str(tr)
        for i in lst:
            __test_search__(tr, i)
        __test_search__(tr, "W")

    def __test_search__(tr, k):
        res = B_tree_search(tr, k)
        if res is None:
            print k, "not found"
        else:
            (node, i) = res
            print "found", node.keys[i]
\end{lstlisting}

Run the function `test\_search' will generate the following result.

\begin{verbatim}
found G
found M
...
found Z
W not found
\end{verbatim}

\subsection{Functional search algorithm}
The imperative algorithm can be turned into Functional by performing
recursive search on a child in case key can't be located in current
node.

\begin{algorithmic}[1]
\Function{B-TREE-SEARCH}{$T, k$}
  \State $i \gets FIND-FIRST( \lambda_xx>=k, KEYS(T))$
  \If{$i$ exists and $k = KEYS(T)[i]$}
    \State \Return $(T, i)$
  \EndIf
  \If{$T$ is leaf}
    \State \Return $NIL$ \Comment{$k$ doesn't exist at all}
  \Else
    \State \Return $B-TREE-SEARCH(CHILDREN(T)[i], k)$
  \EndIf
\EndFunction
\end{algorithmic}

\subsubsection*{Search program in Haskell}
In Haskell program, we first filter out all keys less than the
key to be searched. Then check the first element in the result.
If it matches, we return the current node along with the index
as a tuple. Where the index start from `0'. If it doesn't match,
We then do recursive search till leaf node.

\lstset{language=C++}
\begin{lstlisting}
search :: (Ord a)=> BTree a -> a -> Maybe (BTree a, Int)
search tr@(Node ks cs _) k
    | matchFirst k $ drop len ks = Just (tr, len)
    | otherwise = if null cs then Nothing
                  else search (cs !! len) k
    where
      matchFirst x (y:_) = x==y
      matchFirst x _ = False
      len = length $ filter (<k) ks
\end{lstlisting}

The verification test cases are provided as the following.

\begin{lstlisting}
testSearch = mapM_ (showSearch (listToBTree lst 3)) $ lst++"L"
    where
      showSearch tr x = do
        case search tr x of
          Just (_, i) -> putStrLn $ "found" ++ (show x)
          Nothing -> putStrLn $ "not found" ++ (show x)
      lst = "GMPXACDEJKNORSTUVYZBFHIQW"
\end{lstlisting} %$

Here we construct a B-tree from a series of string, then
we check if each element in this string can be located.
Finally, an non-existed element ``L'' is fed to verify the
failure case.

Run this test function generates the following results.

\begin{verbatim}
found'G'
found'M'
...
found'W'
not found'L'
\end{verbatim}

\subsubsection*{Search program in Scheme/Lisp}
Because we intersperse children and keys in one list in
Scheme/Lisp B-tree definition, the search function just
move one step a head to locate the key in a node.

\lstset{language=lisp}
\begin{lstlisting}
(define (btree-search tr x)
  ;; find the smallest index where keys[i]>= x
  (define (find-index tr x)
    (let ((pred (if (string? x) string>=? >=)))
      (if (null? tr)
	  0
	  (if (and (not (list? (car tr))) (pred (car tr) x))
	      0
	      (+ 1 (find-index (cdr tr) x))))))
  (let ((i (find-index tr x)))
    (if (and (< i (length tr)) (equal? x (list-ref tr i)))
	(cons tr i)
	(if (leaf? tr) #f (btree-search (list-ref tr (- i 1)) x)))))
\end{lstlisting}

The program defines an inner function to find the index of the
first element which is greater or equal to the key we are searching.

If the key pointed by this index matches, we are done. Otherwise,
this index points to a child which may contains this key. The program
will return false result in case the current node is a leaf node.

We can run the below testing function to verify this searching
program.

\begin{lstlisting}
(define (test-search)
  (define (search-and-show tr x)
    (if (btree-search tr x)
	(display (list "found " x))
	(display (list "not found " x))))
  (let* ((lst (str->slist "GMPXACDEJKNORSTUVYZBFHIQW"))
	 (tr (list->btree lst 3)))
    (map (lambda (x) (search-and-show tr x)) (cons "L" lst))))
\end{lstlisitng}

A non-existed key ``L'' is firstly fed, and then all elements
which used to form the B-tree are looked up for verification.

\begin{lstlisting}
(not found  L)(found  G)(found  M) ... (found  W)
\end{lstlisting}

% ================================================================
%                 Short summary
% ================================================================
\section{Notes and short summary}
In this post, we explained the B-tree data structure as a kind of
extension from binary search tree. The background knowledge of
magnetic disk access is skipped, user can refer to \cite{CLRS}
for detail. For the three main operations, insertion, deletion,
and searching, both imperative and functional algorithms are
illustrated. The complexity isn't discussed here, However, since
B-tree are defined to maintain the balance properties, all operations
mentioned here perform $O(lgN)$ where $N$ is the number of the
keys in a B-tree.

\begin{Exercise}
\begin{itemize}
\item Eleminate the recursion in imperative B-tree insertion algorithm.
\end{itemize}
\end{Exercise}
% ================================================================
%                 Appendix
% ================================================================
\section{Appendix} \label{appendix}
%\appendix
All programs provided along with this article are free for
downloading.

\subsection{Prerequisite software}
GNU Make is used for easy build some of the program. For C++ and ANSI C programs,
GNU GCC and G++ 3.4.4 are used.
For Haskell programs GHC 6.10.4 is used
for building. For Python programs, Python 2.5 is used for testing, for
Scheme/Lisp program, MIT Scheme 14.9 is used.

all source files are put in one folder. Invoke 'make' or 'make all'
will build C++ and Haskell program.

Run 'make Haskell' will separate build Haskell program. the executable
file is ``htest'' (with .exe
in Window like OS). It is also possible to run the program in GHCi.

\subsection{Tools}

Besides them, I use graphviz to draw most of the figures in this post. In order to
translate the B-tree output to dot script. A Haskell tool is provided.
It can be used like this.

\begin{verbatim}
bt2dot filename.dot "string"
\end{verbatim}

Where filename.dot is the output file for the dot script. It can
parse the string which describes B-tree content and translate it
into dot script.

This source code of this tool is BTr2dot.hs, it can also be downloaded
with this article.

download position: http://sites.google.com/site/algoxy/btree/btree.zip

\begin{thebibliography}{99}

\bibitem{CLRS}
Thomas H. Cormen, Charles E. Leiserson, Ronald L. Rivest and Clifford Stein. ``Introduction to Algorithms, Second Edition''. The MIT Press, 2001. ISBN: 0262032937.

\bibitem{wiki-b-tree}
B-tree, Wikipedia. http://en.wikipedia.org/wiki/B-tree

\bibitem{lxy-bst}
Liu Xinyu. ``Comparison of imperative and functional implementation of
binary search tree''. http://sites.google.com/site/algoxy/bstree

\bibitem{okasaki-rbtree}
Chris Okasaki. ``FUNCTIONAL PEARLS Red-Black Trees in a Functional Setting''. J. Functional Programming. 1998

\end{thebibliography}

\ifx\wholebook\relax \else
\end{document}
\fi
